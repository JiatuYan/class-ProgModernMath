\begin{rem}
  In this appendix, we follow
  \cite{axler15:_linear_algeb_done_right}
  for most contents.
  % who advertises postponing the learning of determinants
  % until the students get a solid grasp
  % on linear spaces and inner product spaces.
  % This compares drastically to the traditional pedagogy
  % where determinant is the starting point. % of linear algebra.
  See the book by \cite{strang16:_introd_linear_algeb}
  for a more elementary text on linear algebra.
\end{rem}


\section{Vector spaces}
\label{sec:vector-spaces}

\begin{defn}
  \label{def:field}
  % A \emph{field} is a commutative division ring.
  % More commonly,
  A \emph{field} $\mathbb{F}$ is a set together with two binary operations,
  usually called ``addition'' and ``multiplication''
  and denoted by ``$+$'' and ``$*$'',
  such that $\forall a,b,c\in\mathbb{F}$,
  the following axioms hold,
  \begin{itemize}\itemsep0em
  \item commutativity: $a+b=b+a$, $ab=ba$;
  \item associativity: $a+(b+c)=(a+b)+c$, $a(bc)=(ab)c$;
  \item identity: $a+0=a$, $a1=a$; % $0\ne 1$;
  \item invertibility: $a+(-a)=0$, $a a^{-1}=1$ ($a\ne 0$);
  \item distributivity: $a(b+c)=ab+ac$.
  \end{itemize}
\end{defn}

\begin{defn}
  \label{def:vectorSpace}
  A \emph{vector space} or \emph{linear space}
  over a field $\mathbb{F}$ is a set ${\cal V}$
  together with two binary operations
  ``$+$'' and ``$\times$''
  respectively called vector addition and scalar multiplication
  that satisfy the following axioms:
  \begin{enumerate}[({VSA}-1)]\itemsep0em
  \item commutativity\\
    $\forall \mathbf{u},\mathbf{v}\in {\cal V}$,\ 
    $\mathbf{u}+\mathbf{v}=\mathbf{v}+\mathbf{u}$;
  \item associativity\\
    $\forall \mathbf{u},\mathbf{v}, \mathbf{w}\in {\cal V}$,\ 
    $(\mathbf{u}+\mathbf{v})+\mathbf{w}
    =\mathbf{u}+(\mathbf{v}+\mathbf{w})$;
  \item compatibility\\
    $\forall \mathbf{u}\in {\cal V}$, $\forall a,b\in \mathbb{F}$, 
    $(ab)\mathbf{u} = a(b\mathbf{u})$;
  \item additive identity\\
    $\exists \mathbf{0}\in {\cal V}$,
    $\forall \mathbf{u}\in{\cal V}$,
    s.t. $\mathbf{u}+\mathbf{0}=\mathbf{u}$;
  \item additive inverse\\
    $\forall \mathbf{u}\in{\cal V}$,
    $\exists \mathbf{v}\in {\cal V}$,
    s.t. $\mathbf{u}+\mathbf{v}=\mathbf{0}$;
  \item multiplicative identity\\
    $\exists 1 \in \mathbb{F}$, s.t.
    $\forall \mathbf{u}\in{\cal V}$,\ 
    $1\mathbf{u}=\mathbf{u}$;
  \item distributive laws\\
    \begin{equation*}
      \forall \mathbf{u},\mathbf{v}\in {\cal V},\ 
      \forall a,b\in \mathbb{F},\ 
      \left\{
        \begin{array}{l}
          (a+b)\mathbf{u}=a\mathbf{u}+b\mathbf{u}, \\
          a(\mathbf{u}+\mathbf{v})=a\mathbf{u}+a\mathbf{v}.
        \end{array}
      \right.
    \end{equation*}
  \end{enumerate}
  The elements of ${\cal V}$ are called \emph{vectors}
  and the elements of $\mathbb{F}$ are called \emph{scalars}.
\end{defn}

\begin{defn}
  \label{def:vectorSpacesRandC}
  A \emph{real vector space}
  or a \emph{complex vector space} is
  a vector space with $\mathbb{F}=\mathbb{R}$
  or $\mathbb{F}=\mathbb{C}$,
  respectively.
\end{defn}

\begin{exc}
  \label{exc:complexVecSpaceIsRealVecSpace}
  Show that a complex vector space
  always induces another real vector space.
\end{exc}
\begin{solution}
  % By Definitions \ref{def:vectorSpace}
  % and \ref{def:vectorSpacesRandC}, 
  $\mathbb{R}$, as a subspace of $\mathbb{C}$, 
  is closed under multiplication
  and we can thus restrict the multiplication
  in Definitions \ref{def:vectorSpace} to $\mathbb{R}$.
\end{solution}

\begin{rem}
  Clearly the converse of Exercise
  \ref{exc:complexVecSpaceIsRealVecSpace}
  does not hold.
  We will limit ourselves to these two vector spaces.
\end{rem}

\begin{exm}
  The simplest vector space is $\{\mathbf{0}\}$.
  Another simple example of a vector space over a field $\mathbb{F}$
  is $\mathbb{F}$ itself,
  equipped with its standard addition and multiplication.
\end{exm}

\begin{defn}
  \label{def:algebra}
  An \emph{algebra} is a vector space ${\cal V}$
  with an associative and distributive multiplication
  ${\cal V}\times {\cal V} \rightarrow {\cal V}$, 
  \begin{align}\nonumber
    &\forall u,v,w\in {\cal V},\ \forall \alpha\in \mathbb{F},
    \\ \label{eq:algebra}
    &\left\{
      \begin{array}{l}
        u(vw) = (uv)w,
        \\
        (u+v)w = uw + vw,\ \ 
        u(v+w) = uv + uw,
        \\
        \alpha(uv) = u(\alpha v) = (\alpha u) v.
      \end{array}
    \right.
  \end{align}
  The \emph{multiplicative identity}
  is the element $e\in {\cal V}$ such that
  $\forall v\in {\cal V}$, $ev=v=ve$.
\end{defn}

\begin{rem}
  Different from a normed space such as $\mathbb{R}^n$,
  a linear spaces of functions has 
  composition a natural \emph{vector} multiplication
  with the identity function as the multiplicative identity.
  In contrast to the vector addition,
   this vector multiplication needs not be commutative.
\end{rem}


\subsection{Subspaces}

\begin{defn}
  A subset ${\cal U}$ of ${\cal V}$ is called a \emph{subspace}
  of ${\cal V}$ if ${\cal U}$ is also a vector space
  when equipped with the same addition and scalar multiplication on $V$.
\end{defn}

\begin{defn}
  Suppose ${\cal U}_1, \ldots, {\cal U}_m$ are subsets of ${\cal V}$.
  The \emph{sum} of ${\cal U}_1, \ldots, {\cal U}_m$
  is the set of all possible sums of elements
  of ${\cal U}_1, \ldots, {\cal U}_m$:
  \begin{equation}
    \label{eq:sumOfSubsets}
    {\cal U}_1 + \ldots + {\cal U}_m
    := \left\{
      \sum_{j=1}^m\mathbf{u}_j : \mathbf{u}_j \in {\cal U}_j
      \right\}.
  \end{equation}
\end{defn}

\begin{exm}
  For
  ${\cal U}=\{(x,x,y,y)\in \mathbb{F}^4: x,y\in \mathbb{F}\}$
  and 
  ${\cal W}=\{(x,x,x,y)\in \mathbb{F}^4: x,y\in \mathbb{F}\}$,
  we have
  \begin{displaymath}
    {\cal U}+{\cal W}=\{(x,x,z,y)\in \mathbb{F}^4: x,y,z\in \mathbb{F}\}.
  \end{displaymath}
\end{exm}

\begin{lem}
  Suppose ${\cal U}_1, \ldots, {\cal U}_m$ are subspaces of ${\cal V}$.
  Then ${\cal U}_1 + \ldots + {\cal U}_m$ is the smallest
  subspace of ${\cal V}$ that contains
  ${\cal U}_1, \ldots, {\cal U}_m$.
\end{lem}

\begin{defn}
  \label{def:directSum}
  Suppose ${\cal U}_1, \ldots, {\cal U}_m$ are subspaces of ${\cal V}$.
  The sum ${\cal U}_1 + \ldots + {\cal U}_m$
  is called a \emph{direct sum}
  if each element in ${\cal U}_1 + \ldots + {\cal U}_m$
  can be written in only one way as a sum
  $\sum_{j=1}^m\mathbf{u}_j$ with $\mathbf{u}_j \in {\cal U}_j$
  for each $j=1, \ldots, m$.
  In this case we write the direct sum as
  ${\cal U}_1 \oplus \ldots \oplus {\cal U}_m$.
\end{defn}

\begin{exc}
  Show that ${\cal U}_1 + {\cal U}_2 + {\cal U}_3$
  is not a direct sum:
  \begin{align*}
    {\cal U}_1 &= \{(x,y,0)\in \mathbb{F}^3: x,y\in \mathbb{F}\},
    \\
    {\cal U}_2 &= \{(0,0,z)\in \mathbb{F}^3: z\in \mathbb{F}\},
    \\
    {\cal U}_3 &= \{(0,y,y)\in \mathbb{F}^3: y\in \mathbb{F}\}.
  \end{align*}
\end{exc}

\begin{lem}
  Suppose ${\cal U}_1, \ldots, {\cal U}_m$ are subspaces of ${\cal V}$.
  Then ${\cal U}_1 + \ldots + {\cal U}_m$
  is a direct sum if and only if
  the only way to write $\mathbf{0}$ as a sum 
  $\sum_{j=1}^m\mathbf{u}_j$,
  where $\mathbf{u}_j \in {\cal U}_j$ for each $j=1, \ldots, m$,
  is by taking each $\mathbf{u}_j$ equal to $\mathbf{0}$.
\end{lem}

\begin{thm}
  \label{thm:directSumIFFintersection0}
  Suppose ${\cal U}$ and ${\cal W}$ are subspaces of ${\cal V}$.
  Then ${\cal U}+{\cal W}$ is a direct sum
  if and only if ${\cal U}\cap {\cal W}=\{\mathbf{0}\}$.
\end{thm}

\subsection{Span and linear independence}
\label{sec:span-line-indep}

\begin{defn}
  \label{def:listN}
  % For $n\in\mathbb{N}$, 
  A \emph{list of length $n$} or \emph{$n$-tuple}
  is an ordered collection of $n$ elements
  (which might be numbers, other lists, or more abstract entities)
  separated by commas and surrounded by parentheses:
  % \begin{equation*}
  $\mathbf{x} = (x_1, x_2, \ldots, x_n)$.
  % \end{equation*}
\end{defn}

\begin{rem}
  According to Definition \ref{def:listN},
  two lists are equal iff
  they have the same length and the same elements in the same order.
  Hence the two lists $(3,5)$ and $(5,3)$ are not equal
  while the two sets $\{3,5\}$ and $\{5,3\}$ are equal.
  Different from a set,
  a list must have finite length and linear ordering;
  in addition, elements of a list may be repeated.
\end{rem}

\begin{defn}
  \label{def:coordinateSpace}
  A vector space composed of all the $n$-tuples of a field
  $\mathbb{F}$
  is known as a \emph{coordinate space},
  denoted by $\mathbb{F}^n$ ($n\in \mathbb{N}^+$).
\end{defn}

\begin{exm}
  The properties of forces or velocities in the real world
  can be captured by a coordinate space $\mathbb{R}^2$ or $\mathbb{R}^3$.
\end{exm}

\begin{exm}
  The set of continuous real-valued functions
  on the interval $[a,b]$ forms a real vector space.
\end{exm}

\begin{ntn}
  \label{ntn:powerSetNotation}
  For a set ${\cal S}$, define a vector space
  \begin{equation*}
    \mathbb{F}^{{\cal S}} := \{ f:{\cal S}\rightarrow \mathbb{F}\}.
  \end{equation*}
  $\mathbb{F}^n$ is a special case of $\mathbb{F}^{{\cal S}}$
  because $n$ can be regarded as the set $\{1,2,\ldots,n\}$
  and each element in $\mathbb{F}^n$
  can be considered as a function $\{1,2,\ldots,n\}\mapsto \mathbb{F}$.
\end{ntn}

\begin{defn}
  A \emph{linear combination} of a list
  of vectors $\{\mathbf{v}_i\}$ is a vector of the form
  $\sum_ia_i\mathbf{v}_i$ where $a_i\in \mathbb{F}$.
\end{defn}

\begin{exm}
  $(17,-4,2)$ is a linear combination of $(2,1,-3),(1,-2,4)$ because
  \begin{equation*}
    (17,-4,2) = 6(2,1,-3)+5(1,-2,4).
  \end{equation*}
\end{exm}

\begin{exm}
  $(17,-4,5)$ is not a linear combination of $(2,1,-3),(1,-2,4)$
  because there do not exist numbers $a_1, a_2$ such that
  \begin{equation*}
    (17,-4,5) = a_1(2,1,-3)+a_2(1,-2,4).
  \end{equation*}
  Solving from the first two equations yields
  $a_1=6$, $a_2=5$, but $5\ne -3\times 6 + 4\times 5$.
\end{exm}

\begin{defn}
  \label{def:span}
  The \emph{span} of a list of vectors $(\mathbf{v}_i)$
  is the set of all linear combinations of $(\mathbf{v}_i)$,
  \begin{equation}
    \label{eq:span}
    \Span(\mathbf{v}_1, \mathbf{v}_2, \ldots, \mathbf{v}_m)
    = \left\{\sum_{i=1}^m a_i\mathbf{v}_i :\  a_i\in \mathbb{F}\right\}.
  \end{equation}
  In particular, the span of the empty set is $\{\mathbf{0}\}$.
  We say that $(\mathbf{v}_1, \mathbf{v}_2, \ldots, \mathbf{v}_m)$
  \emph{spans} ${\cal V}$
  if ${\cal V}=\Span(\mathbf{v}_1, \mathbf{v}_2, \ldots, \mathbf{v}_m)$.
\end{defn}

\begin{exm}
  \begin{align*}
    (17,-4,2) &\in \Span((2,1,-3), (1,-2,4)) \\
    (17,-4,5) &\not\in \Span((2,1,-3), (1,-2,4))
  \end{align*}
\end{exm}

\begin{defn}
  A vector space ${\cal V}$ is called \emph{finite dimensional}
  if some list of vectors span ${\cal V}$; otherwise
  it is \emph{infinite dimensional}.
\end{defn}

\begin{exm}
  \label{exm:polynomial-spaces}
  Let $\mathbb{P}_m(\mathbb{F})$ denote the set of all polynomials
  with coefficients in $\mathbb{F}$ and degree at most $m$,
  \begin{equation}
    \label{eq:polynomialSpace}
    \mathbb{P}_m(\mathbb{F}) = \left\{
      p:\mathbb{F}\rightarrow\mathbb{F};\ 
      p(z) = \sum_{i=0}^m a_i z^i, a_i\in \mathbb{F} \right\}.
  \end{equation}
  Then $\mathbb{P}_m(\mathbb{F})$ is a finite-dimensional vector space
  for each non-negative integer $m$.
  The set of all polynomials with coefficients in $\mathbb{F}$,
  % In particular,
  denoted by $\mathbb{P}(\mathbb{F}):=\mathbb{P}_{+\infty}(\mathbb{F})$, 
  is infinite-dimensional.
  Both are subspaces of $\mathbb{F}^{\mathbb{F}}$
  for $\mathbb{F}=\mathbb{R}$ or $\mathbb{C}$.
  % 
  % see Example \ref{ntn:powerSetNotation}.
\end{exm}

\begin{defn}
  \label{def:linearIndependence}
  A list of vectors $(\mathbf{v}_1, \mathbf{v}_2, \ldots,
  \mathbf{v}_m)$
  in ${\cal V}$ is called \emph{linearly independent}
  iff
  \begin{equation}
    \label{eq:linearIndependent}
    a_1 \mathbf{v}_1 +\ldots + a_m \mathbf{v}_m = \mathbf{0}
    \ \Rightarrow\  a_1=\cdots=a_m=0.
    % \forall a_i\in \mathbb{F},\ 
    % \sum_{i=1}^m a_i \mathbf{v}_i = \mathbf{0}
    % \ \Rightarrow\ \forall i, a_i=0.
  \end{equation}
  Otherwise the list of vectors is called \emph{linearly dependent}.
\end{defn}

\begin{exm}
  The empty list is declared to be linearly independent.
  A list of one vector ($\mathbf{v}$) is linearly independent
  iff $\mathbf{v}\ne \mathbf{0}$.
  A list of two vectors is linearly independent iff neither vector
  is a scalar multiple of the other.
\end{exm}

\begin{exm}
  The list $(1,z,\ldots,z^m)$ is linearly independent in 
  $\mathbb{P}_{m}(\mathbb{F})$ for each $m\in \mathbb{N}$.
\end{exm}

\begin{exm}
  $(2,3,1)$, $(1,-1,2)$, and $(7,3,8)$ is linearly dependent in
  $\mathbb{R}^3$
  because
  \begin{equation*}
    2(2,3,1)+3(1,-1,2)+(-1)(7,3,8)=(0,0,0).
  \end{equation*}
\end{exm}

\begin{exm}
  Every list of vectors containing the $\mathbf{0}$ vector
  is linearly dependent.
\end{exm}

\begin{lem}[Linear dependence lemma]
  Suppose \mbox{$V=(\mathbf{v}_1,\mathbf{v}_2,\cdots,\mathbf{v}_m)$}
  is a linearly dependent list in ${\cal V}$.
  Then there exists $j\in \{1,2,\ldots,m\}$ such that
  \begin{itemize}\itemsep0em
  \item $\mathbf{v}_j\in \Span(\mathbf{v}_1,\mathbf{v}_2,\ldots,\mathbf{v}_{j-1})$;
  \item if the $j$th term is removed from $V$,
    the span of the remaining list equals $\Span(\mathbf{v}_1,\mathbf{v}_2,\ldots,\mathbf{v}_{m})$.
  \end{itemize}
\end{lem}

\begin{lem}
  \label{lem:lengthOfIndependentAndSpanningLists}
  In a finite-dimensional vector space, the length
  of every linearly independent list of vectors
  is less than or equal to the length of every
  spanning list of vectors.
\end{lem}


\subsection{Bases}
\label{sec:bases}

\begin{defn}
  \label{defn:basis}
  A \emph{basis} of a vector space ${\cal V}$ 
  is a list of vectors in ${\cal V}$ 
  that is linearly independent
  and spans ${\cal V}$.
\end{defn}

\begin{defn}
  \label{def:standardBasis}
  The \emph{standard basis} of $\mathbb{F}^n$ is the list of vectors
  \begin{equation}
    \label{eq:standardBasis}
    (1,0,\cdots,0)^T,\  (0,1,0,\cdots,0)^T,\ 
    \ldots,\ (0,\cdots,0, 1)^T.
  \end{equation}
\end{defn}

\begin{exm}
  $(z^0, z^1, \ldots, z^m)$ is
  a basis of $\mathbb{P}_m(\mathbb{F})$
  in (\ref{eq:polynomialSpace}).
\end{exm}

\begin{lem}
  A list of vectors ($\mathbf{v}_1, \ldots, \mathbf{v}_n$) is
  a basis of ${\cal V}$ iff every vector $\mathbf{u}\in {\cal V}$
  can be written uniquely as
  \begin{equation}
    \label{eq:basis}
    \mathbf{u} = \sum_{i=1}^n a_i \mathbf{v}_i,
  \end{equation}
  where $a_i\in \mathbb{F}$.
\end{lem}

\begin{lem}
  Every spanning list in a vector space ${\cal V}$ can be reduced
  to a basis of ${\cal V}$.
\end{lem}

\begin{lem}
  \label{lem:linearIndependentListExtendableToBasis}
  Every linearly independent list of vectors
  in a finite-dimensional vector space
  can be extended to a basis of that vector space.
\end{lem}


\subsection{Dimension}

\begin{lem}
  Any two bases of a finite-dimensional vector space
  have the same length.
\end{lem}
\begin{proof}
  Suppose $B_1$ and $B_2$ are two bases of $V$.
  Then $B_1$ is linearly independent in $V$
  and $B_2$ spans $V$.
  By Lemma \ref{lem:lengthOfIndependentAndSpanningLists},
  the length of $B_1$ is no greater than $B_2$.
  The proof is completed by switching
  the roles of $B_1$ and $B_2$.
\end{proof}

\begin{defn}
  The \emph{dimension} of a finite-dimensional vector space
  ${\cal V}$,
  denoted $\dim {\cal V}$,
  is the length of any basis of the vector space.
\end{defn}

\begin{rem}
  The above definition is well defined because
  any two bases of a finite-dimensional vector space
  have the same length.
\end{rem}

\begin{lem}
  \label{lem:spanningListAndDimImpliesBasis}
  If ${\cal V}$ is finite-dimensional,
  then every spanning list of vectors in ${\cal V}$
  with length $\dim {\cal V}$
  is a basis of ${\cal V}$.
\end{lem}

\begin{lem}
  \label{lem:linearIndependenceAndDimImpliesBasis}
  If ${\cal V}$ is finite-dimensional,
  then every linearly independent list of vectors in ${\cal V}$
  with length $\dim {\cal V}$
  is a basis of ${\cal V}$.
\end{lem}

\begin{rem}
  Why do we care so much about bases of a vector space?
  Because a basis greatly reduces the complexity
  of dealing with a vector space.
  In fact, it reduces the $O(\infty)$ complexity
  of the vector space
  to the $O(N)$ complexity
  of the basis.
  As another example,
  the Riesz representation theorem \ref{thm:RieszRep}
  reduces the $O(\infty)$ complexity
  to the $O(1)$ complexity of taking
  inner product to a single element.
\end{rem}


\section{Linear maps}
\label{sec:linear-maps}

\begin{defn}
  \label{def:linearMap}
  A \emph{linear map} or \emph{linear transformation}
  between two vector spaces ${\cal V}$ and ${\cal W}$
  is a function $T: {\cal V}\rightarrow{\cal W}$
  that satisfies
  \begin{enumerate}[(LNM-1)]\itemsep0em
    \itemsep0em
  \item additivity\\
    $\forall \mathbf{u}, \mathbf{v}\in {\cal V}$,\ 
    $T(\mathbf{u}+\mathbf{v}) = T\mathbf{u} + T\mathbf{v}$;
  \item homogeneity\\
    $\forall a\in \mathbb{F}$, $\forall\mathbf{v}\in{\cal V}$,\ 
    $T(a\mathbf{v})=a(T\mathbf{v})$,
  \end{enumerate}
  where $\mathbb{F}$ is a scalar field.
%  of ${\cal V}$ and ${\cal W}$.
  In particular, a linear map $T: {\cal V}\rightarrow{\cal W}$
  is called a \emph{(linear) operator} if ${\cal W}={\cal V}$.
\end{defn}

\begin{ntn}
  \label{ntn:spaceOfLinearMaps}  
  The set of all linear maps from ${\cal V}$ to ${\cal W}$
  is denoted by ${\cal L}({\cal V}, {\cal W})$.
  The set of all linear operators from ${\cal V}$ to itself
  is denoted by ${\cal L}({\cal V})$.
\end{ntn}

\begin{exm}
  \label{exm:polynomialDiffIsLinearMap}
  The differentiation map 
  $D\in {\cal L}(\mathbb{P}(\mathbb{R}))$
  given by $D p = p'$
  is a linear map. 
\end{exm}

\begin{exm}
  ${\cal L}(\mathbb{F}^n, \mathbb{F}^m)\simeq \mathbb{F}^{m\times n}$
  is a vector space with the zero map $\mathbf{0}$
  as the additive identity.
\end{exm}

% \begin{exm}
%   \label{exm:homomorphismAsLinearMaps}
%   For two groups $(G_1,+)$, $(G_2,+)$
%   as described in Example \ref{exm:groupAsVectorSpace},
%   a homomorphism $f:G_1\rightarrow G_2$
%   can be considered as a linear map from $G_1$ to $G_2$.
%   In this context,
%   the homomorphism condition implies (LNM-1,2).
% \end{exm}

\begin{lem}
  The set ${\cal L}({\cal V}, {\cal W})$,
  equipped with
  scalar multiplication
  $(aT)\mathbf{v}= a(T\mathbf{v})$
  and vector addition
  \mbox{$(S+T)\mathbf{v} = S\mathbf{v}+ T\mathbf{v}$},
  is a vector space.
\end{lem}
\begin{proof}
  The scalar field $\mathbb{F}$ of ${\cal L}({\cal V}, {\cal W})$
  is the same as that of ${\cal V}$ and ${\cal W}$.
  So multiplicative identity is still 1,
  the same as that of $\mathbb{F}$.
  However, the additive identity is the zero map
  $\mathbf{0} \in {\cal L}({\cal V}, {\cal W})$.
\end{proof}

\begin{defn}
  The \emph{identity map}, denoted $I$,
  is the function on a vector space
  that assigns to each element the same element:
  \begin{equation}
    \label{eq:identityMap}
    I\mathbf{v} = \mathbf{v}.
  \end{equation}
\end{defn}

\begin{defn}
  \label{def:linearFunctionals}
  A \emph{complex linear functional} is
  a linear map $T: {\cal V}\rightarrow \mathbb{C}$
  with $\mathbb{C}$ being the underlying field of ${\cal V}$. 
  A \emph{real linear functional}
  is a map $T: {\cal V}\rightarrow \mathbb{R}$
  such that (LNM-1) and (LNM-2) in Definition \ref{def:linearMap}
  hold for $\mathbb{F}=\mathbb{R}$.
\end{defn}

\begin{rem}
  By Definition \ref{def:linearFunctionals},
  the domain ${\cal V}$ of a complex linear functional $T: {\cal V}\rightarrow \mathbb{C}$
  must be a complex vector space. 
  In contrast,
  the domain ${\cal V}$ of a real linear functional
  $T: {\cal V}\rightarrow \mathbb{R}$, 
  can be a complex vector space
  or a real vector space,
  i.e., the underlying field of ${\cal V}$
  can be $\mathbb{C}$ or $\mathbb{R}$.
  This flexibility is a consequence of the fact
  that a complex vector space is also a real vector space.
\end{rem}

\begin{lem}
  \label{lem:complexLinearFuncRealPart}
  Let $V$ be a complex vector space
  and $f$ a complex linear functional on $V$.
  Then the real part $\Rez f (x) = u(x)$ is related to $f$
  by
  \begin{equation}
    \label{eq:complexLinearFuncRealPart}
    \forall x\in V, \quad f(x) = u(x) - \ii u(\ii x).
  \end{equation}
\end{lem}
\begin{proof}
  Any $\alpha,\beta\in \mathbb{R}$
  and $z=\alpha+\ii \beta\in \mathbb{C}$ satisfy
  \begin{displaymath}
    z = \Rez z - \ii \Rez (\ii z).
  \end{displaymath}
  Set $z=f(x)$ and we have
  \begin{align*}
    f(x) &= \Rez f(x) - \ii \Rez (\ii f(x))
    \\
    &= u(x) - \ii \Rez (f(\ii x))
    = u(x) - \ii u(\ii x). \qedhere
  \end{align*}
\end{proof}

\begin{lem}
  \label{lem:realLinearFuncToComplex}
  Let $V$ be a complex vector space
  and $u: V\rightarrow \mathbb{R}$
  a real linear functional on $V$.
  Then the function $f: V \rightarrow \mathbb{C}$
  defined by (\ref{eq:complexLinearFuncRealPart})
  is a complex linear functional. % on $V$.
\end{lem}
\begin{proof}
  The additivity (LNM-1) of $f$ follows from
  the additivity of $u$ and (\ref{eq:complexLinearFuncRealPart}).
  For any $c\in \mathbb{R}$, we have
  $f(cx)=cf(x)$ from
  (\ref{eq:complexLinearFuncRealPart}).
  The rest follows from the additivity of $f$
  and 
  \begin{align*}
    f(\ii x) &= u(\ii x) - \ii u(\ii^2 x)
    \\
    &= u(\ii x) + \ii u( x)
      = \ii (u(x) - \ii u(\ii x)) = \ii f(x).
      \qedhere
  \end{align*}
\end{proof}

\subsection{Null spaces and ranges}
\label{sec:null-spaces-ranges}

\begin{defn}
  \label{def:nullSpace}
  The \emph{null space} or \emph{kernel} of a linear map
  \mbox{$T\in {\cal L}({\cal V}, {\cal W})$}
  is the subset of ${\cal V}$ consisting of those vectors
  that $T$ maps to the additive identity $\mathbf{0}$:
  \begin{equation}
    \label{eq:nullSpace}
    \Null\,T = \{\mathbf{v}\in {\cal V} : T\mathbf{v}=\mathbf{0}\}.
  \end{equation}
\end{defn}

\begin{exm}
  The null space of the differentiation map
  in Example \ref{exm:polynomialDiffIsLinearMap} is $\mathbb{R}$.
\end{exm}

% \begin{rem}
%   The null space is a special type of kernel
%   in Definition \ref{def:kernel}.
% \end{rem}

\begin{thm}
  \label{thm:injectiveIsNull0}
  A linear map $T\in {\cal L}(V,W)$ is injective
  if and only if $\Null\,T=\{\mathbf{0}\}$.
\end{thm}

\begin{defn}
  \label{def:range}
  The \emph{range} of a linear map
  \mbox{$T\in {\cal L}({\cal V}, {\cal W})$}
  is the subset of ${\cal W}$ consisting of those vectors
  that are of the form $T\mathbf{v}$ for some $\mathbf{v}\in{\cal V}$:
  \begin{equation}
    \label{eq:range}
    \Range\ T = \{T\mathbf{v}: \mathbf{v}\in {\cal V}\}.
  \end{equation}
\end{defn}

\begin{exm}
  The range of $A\in \mathbb{C}^{m\times n}$
  is the span of its column vectors.
\end{exm}

\begin{thm}
  \label{thm:RangeIsAsubspace}
  The range of a linear map $T\in {\cal L}(V,W)$
  is a subspace of $W$.
\end{thm}

\begin{thm}[The counting theorem or the fundamental theorem of linear maps]
  \label{thm:dimensionsOfNullAndRange}
  If ${\cal V}$ is a finite-dimensional vector space
  and $T\in {\cal L}({\cal V}, {\cal W})$,
  then $\Range\ T$ is a finite-dimensional subspace of ${\cal W}$
  and
  \begin{equation}
    \label{eq:dimensionsOfNullAndRange}
    \dim {\cal V} = \dim \Null\ T + \dim \Range\ T.
  \end{equation}
\end{thm}

\begin{thm}
  \label{thm:finiteDimVecSpaceInjecSurjecInvert}
  For an operator $T\in{\cal L}({\cal V})$ on a finite-dimensional vector
  space ${\cal V}$,
  the following are equivalent:
  \begin{enumerate}[(a)]\itemsep0em
  \item $T$ is invertible;
  \item $T$ is injective;
  \item $T$ is surjective.
  \end{enumerate}
\end{thm}

\subsection{The matrix of a linear map}
\label{sec:matrices-linear-maps}

\begin{defn}
  \label{def:matrixOfLinearMap}
  The \emph{matrix of a linear map
    \mbox{$T\in {\cal L}({\cal V}, {\cal W})$}
    with respect to the bases}
  $(\mathbf{v}_1, \mathbf{v}_2, \ldots, \mathbf{v}_n)$
  of ${\cal V}$
  and $(\mathbf{w}_1, \mathbf{w}_2, \ldots, \mathbf{w}_m)$
  of ${\cal W}$, denoted by
  \begin{equation}
    \label{eq:matrixOfLinearMap}
    M_T:=M(T, (\mathbf{v}_1, \ldots, \mathbf{v}_n),
    (\mathbf{w}_1, \ldots, \mathbf{w}_m)),
  \end{equation}
  is the $m\times n$ matrix $A(T)$
  whose entries $a_{i,j}\in \mathbb{F}$
  satisfy the linear system
  \begin{equation}
    \label{eq:matrixOfLinearMapRelation}
    \forall j=1,2,\ldots,n, \qquad
    T\mathbf{v}_j = \sum_{i=1}^m a_{i,j}\mathbf{w}_i.
    % = a_{1,j}\mathbf{w}_1 %+ a_{2,j}\mathbf{w}_2
    % + \cdots + a_{m,j}\mathbf{w}_m,
  \end{equation}
\end{defn}

\begin{rem}
  There are $m\times n$ equations and  $m\times n$ variables
  in the linear system (\ref{eq:matrixOfLinearMap}).
  In $\sum_{i=1}^m a_{i,j}\mathbf{w}_i$,
  we index $a$ as $a_{i,j}$, not $a_{j,i}$, why?
  Because when we write it in matrix product form,
  $a_{i,j}$ naturally corresponds to the $(i,j)$ entry of the matrix.
\end{rem}

\begin{coro}
  \label{coro:matrixOfLinearMap}
  The matrix $M_T$ in (\ref{eq:matrixOfLinearMap})
  of a linear map
  \mbox{$T\in {\cal L}({\cal V}, {\cal W})$}
  satisfies
  \begin{equation}
    \label{eq:relationOfLinearMapToItsMatrix}
    T [\mathbf{v}_1, \mathbf{v}_2, \ldots, \mathbf{v}_n]
    =[\mathbf{w}_1, \mathbf{w}_2, \ldots, \mathbf{w}_m] M_T.
    % =: W M_T.
  \end{equation}
\end{coro}
\begin{proof}
  This follows directly from (\ref{eq:matrixOfLinearMapRelation}).
\end{proof}

\begin{rem}
  Definition \ref{def:matrixOfLinearMap} 
  packages an essential idea:
  if we express any vector as a linear combination
  of the basis
  and insist on the conditions in Definition \ref{def:linearMap},
  the action of a linear map on an arbitrary vector 
  can be characterized by the effects on the two bases
  via the corresponding matrix of the linear map.
  More precisely,
  for $\mathbf{u}=\sum_i c_i\mathbf{v}_i$,
  we have 
  \begin{align*}
    T\mathbf{u} = \sum_i c_iT\mathbf{v}_i = T [\mathbf{v}_1, \ldots, \mathbf{v}_n] \mathbf{c}
    = [\mathbf{w}_1, \ldots, \mathbf{w}_m] M_T \mathbf{c}, 
  \end{align*}
  i.e., the matrix $M_T$ maps the coordinates of $\mathbf{u}$
  with respect to  $(\mathbf{v}_1, \mathbf{v}_2, \ldots, \mathbf{v}_n)$
  in ${\cal V}$
  to the coordinates of $T\mathbf{u}$
  with respect to $(\mathbf{w}_1, \mathbf{w}_2, \ldots, \mathbf{w}_m)$
  in ${\cal W}$.
  Finally, the matrix of a linear map is dependent on
  the choice of basis while the linear map itself is not.
\end{rem}


\subsection{Duality}
\label{sec:duality}

\subsubsection{Dual vector spaces}
\label{sec:dual-space}

\begin{defn}
  \label{def:dualSpace}
  The \emph{dual space} of a vector space $V$
   is the vector space of all linear functionals on $V$,
   \begin{equation}
     \label{eq:dualSpace}
     V'={\cal L}(V,\mathbb{F}).
   \end{equation}
\end{defn}

\begin{defn}
  \label{def:dualBasis}
  For a basis $\mathbf{v}_1,\ldots, \mathbf{v}_n$ of a vector space $V$,
   its \emph{dual basis}
   is the list $\varphi_1, \ldots, \varphi_n$
   where each $\varphi_j\in V'$ is
   \begin{equation}
     \label{eq:dualBasis}
     \varphi_j(\mathbf{v}_k) = 
     \begin{cases}
       1 & \textrm{if } k=j,
       \\
       0 & \textrm{if } k\ne j.
     \end{cases}
   \end{equation}
\end{defn}

\begin{rem}
  Equation (\ref{eq:dualBasis}) does not entirely define
   the behavior of $\varphi_j$;
   we have to add the fact of $\varphi_j$
   being a linear functional
   to complete graph of $\varphi_j$.
\end{rem}

\begin{rem}
  \label{rem:standardProjections}
  What is the dual basis of the standard basis
   $(\mathbf{e}_1, \ldots, \mathbf{e}_n)$
   of $\mathbb{F}^n$?
%
  By (\ref{eq:dualBasis}), we have
  \begin{displaymath}
     \varphi_j(\mathbf{e}_k) = 
     \begin{cases}
       1 & \textrm{if } k=j,
       \\
       0 & \textrm{if } k\ne j.
     \end{cases}
  \end{displaymath}
  Hence we have
  \begin{displaymath}
    \forall \mathbf{v}=(v_1,\ldots,v_n)\in \mathbb{F}^n,\ \ 
    \varphi_j(\mathbf{v}) = v_j,
  \end{displaymath}
  i.e., the dual basis of the standard basis
   consists of standard projections.
\end{rem}

\begin{exc}
  \label{exc:basicChainsCochainsAreBasis}
  Show that the dual basis is a basis of the dual space.
  % basic chains form a basis for the vector space $C_k$
  % and basic cochains form a basis for the vector space $C^k$.
\end{exc}
\begin{solution}
  This follows from the definition of linear independence
   and Definition \ref{def:dualBasis}.
\end{solution}
 
\begin{lem}
  \label{lem:dimV=dimV'}
  A finite-dimensional vector space $V$ satisfies
  \begin{equation}
    \label{eq:dimV=dimV'}
     \dim V' = \dim V.
  \end{equation}
\end{lem}
\begin{proof}
  This follows from Definition \ref{def:dualSpace}
  and the identity
  $\dim {\cal L}(V,W) = \dim(V) \dim(W)$.
\end{proof}

\begin{defn}
  \label{def:doubleDualSpace}
  The \emph{double dual space} of a vector space $V$,
   denoted by $V''$, 
   is the dual space of $V'$.
\end{defn}

\begin{rem}
  The signature of any linear map $\theta\in V''$ is
   \mbox{$\theta: V'\rightarrow \mathbb{F}$}.
\end{rem}

\begin{lem}
  \label{lem:dualDualAuxFunc}
  The function $\Lambda: V\rightarrow V''$
   defined as
   \begin{equation}
     \label{eq:dualDualAuxFunc}
     \forall v\in V, \forall \varphi\in V',\qquad
     (\Lambda v) (\varphi) = \varphi(v)
   \end{equation}
   is a linear bijection.
%   and hence an isomorphism between $V$ and $V''$.
\end{lem}
\begin{proof}
  It is easily verified that
   $\Lambda$ is a linear map.
  The rest follows from
   Definitions \ref{def:dualSpace}, \ref{def:doubleDualSpace},
   and Lemma \ref{lem:dimV=dimV'}.
\end{proof}

\begin{rem}
  As a mnemonic device,
   think of the triplet $v,\varphi, \Lambda$
   as that of a bullet, a gun, and a cowboy:
   the cowboy put the bullet into a gun and pull the trigger.
\end{rem}

\begin{rem}
 If $V$ is finite-dimensional,
  $V'$ and $V$ are isomorphic,
  but finding an isomorphism from $V$ onto $V'$
  generally requires choosing a basis of $V$.
 In contrast, the isomorphism $\Lambda$ from $V$ onto $V''$
  does not require a choice of basis and
  is considered more natural.  
\end{rem}


\subsubsection{Dual linear maps}

\begin{defn}
  \label{def:dualMap}
  The \emph{dual map} of a linear map 
   \mbox{$T: V\rightarrow W$}
   is the linear map $T': W'\rightarrow V'$
   defined as
   \begin{equation}
     \label{eq:dualMap}
     \forall\varphi\in W',\qquad
     T'(\varphi) = \varphi\circ T.
   \end{equation}
\end{defn}

\begin{exc}
  % Denote by $D$ the linear map
  %  in Example \ref{exm:polynomialDiffIsLinearMap}.
   % of differentiation
   % $D p = p'$ on 
   % the vector space $\mathbb{P}(\mathbb{R})$
   % of polynomials with real coefficients.
  Under the dual map of the linear map $D$
   in Example \ref{exm:polynomialDiffIsLinearMap}, 
  what is the image of
   the linear functional $\varphi(p)=\int_0^1 p$
   on $\mathbb{P}(\mathbb{R})$?   
\end{exc}
\begin{solution}
  The second fundamental theorems of calculus
  (Theorem \ref{thm:fundamentalThmCalculus2}) yields
  \begin{displaymath}
     (D' \varphi)(p) = (\varphi\circ D)(p)
     = \varphi(p') = \int_0^1 p' = p(1) - p(0).
  \end{displaymath}
\end{solution}

\begin{thm}
  \label{thm:matrixOfDualMap}
  The matrix of $T'$ is the transpose of the matrix of $T$.
\end{thm}
\begin{proof}
  Let $(\mathbf{v}_1, \cdots, \mathbf{v}_n)$,
   $(\varphi_1, \cdots, \varphi_n)$,
   $(\mathbf{w}_1, \cdots, \mathbf{w}_m)$,
   $(\psi_1, \cdots, \psi_m)$,
   be bases of $V$, $V'$, $W$, $W'$,
   respectively.
  Denote by $A$ and $C$ the matrices of $T: V\rightarrow W$
   and $T': W'\rightarrow V'$,
   respectively.
  We have
  \begin{displaymath}
    \psi_j\circ T = T'(\psi_j) = \sum_{r=1}^n c_{r,j} \varphi_r.
  \end{displaymath}
  By Corollary \ref{coro:matrixOfLinearMap},
   applying this equation to $\mathbf{v}_k$ yields
  \begin{displaymath}
    (\psi_j\circ T)(\mathbf{v}_k)
    = \sum_{r=1}^n c_{r,j} \varphi_r (\mathbf{v}_k)
    = c_{k,j}.
  \end{displaymath}
  On the other hand, we have
  \begin{align*}
    (\psi_j\circ T)(\mathbf{v}_k)
    &= \psi_j(T \mathbf{v}_k)
    = \psi_j\left(\sum_{r=1}^m a_{r,k} w_r \right)
    \\
    &= \sum_{r=1}^m a_{r,k} \psi_j(w_r)
    = a_{j,k}. \qedhere
  \end{align*}
\end{proof}

\begin{defn}
  \label{def:doubleDualMap}
  The \emph{double dual map} of a linear map $T: V\rightarrow W$
   is the linear map $T'': V''\rightarrow W''$
   defined as $T'' = (T')'$.
\end{defn}

\begin{thm}
  \label{thm:doubleDualMapCommute}
  For $T\in {\cal L}(V)$ and $\Lambda$ in (\ref{eq:dualDualAuxFunc}),
   we have
   \begin{equation}
     \label{eq:doubleDualMapCommute}
     T''\circ \Lambda = \Lambda\circ T.
   \end{equation}
\end{thm}
\begin{proof}
  Definition \ref{def:doubleDualMap} and
   equation (\ref{eq:dualDualAuxFunc}) yields
   \begin{align*}
     \forall v\in V,&\ \ \forall \varphi\in V',
     \\
     (T''\circ \Lambda) v \varphi
     &= ((T')' \Lambda v) \varphi
     = (\Lambda v \circ T')\varphi
     = \Lambda v ( T'\varphi)
     \\
     &= (T' \varphi) (v)
     = \varphi(Tv)
     = \Lambda(Tv)(\varphi)
       \\
     &= (\Lambda\circ T) v \varphi,
   \end{align*}
   where the third step is natural
   since $T'$ send $V'$ to $V'$.
\end{proof}

\begin{coro}
  \label{coro:doubleDualMapFiniteDim}
  For $T\in {\cal L}(V)$ where $V$ is finite-dimensional,
   the double dual map is
   \begin{equation}
     \label{eq:doubleDualMapFiniteDim}
     T'' = \Lambda\circ T \circ \Lambda^{-1}.
   \end{equation}
\end{coro}
\begin{proof}
  This follows directly from Theorem \ref{thm:doubleDualMapCommute}
   and Lemma \ref{lem:dualDualAuxFunc}.
\end{proof}

% \begin{exc}
%   Draw commutative diagrams to illustrate
%    Lemma \ref{lem:dualDualAuxFunc},
%    Theorem \ref{thm:doubleDualMapCommute},
%    and Corollary \ref{coro:doubleDualMapFiniteDim}.
% \end{exc}

\begin{rem}
For $T\in {\cal L}(V)$,
   the linear map $\Lambda$ in Lemma \ref{lem:dualDualAuxFunc}
   can be used to construct a formula of $T''$.
%   By the signature of $T'':V''\rightarrow V''$,
%    the input of $T''$ is a linear functional
%    $\pi:V''\rightarrow \mathbb{F}$.
%   By sending $V$ to $V''$,
%    $\Lambda$ satisfies
%    $\varphi = \pi\circ \Lambda$, 
%    as shown below.
%    \begin{displaymath}
%      \begin{tikzcd}[column sep=3em]
%        V'' \ar{dr}{\pi} 
%        & 
%        \\
%        V \ar{u}{\Lambda} \ar{r}{\varphi}
%        & \mathbb{F}
%      \end{tikzcd}
%    \end{displaymath}
  The commutative diagram for Theorem \ref{thm:doubleDualMapCommute}
   is as follows.
   \begin{displaymath}
     \begin{tikzcd}[column sep=3em]
       V'' \ar{r}{T''} 
       & V''
       \\
       V \ar{u}{\Lambda} \ar{r}{T}
       & V \ar{u}{\Lambda} 
     \end{tikzcd}
   \end{displaymath}
  Flip the direction of the left arrow
   and we have the diagram
   for Corollary \ref{coro:doubleDualMapFiniteDim}.
\end{rem}


\subsubsection{The null space and range of the dual of a linear map}

\begin{defn}
  \label{def:annihilator}
  For $U\subset V$,
   the \emph{annihilator} of $U$, denoted $U^0$,
   is defined by
   \begin{equation}
     \label{eq:annihilator}
     U^0 := \{\varphi \in V':\ \forall \mathbf{u}\in U,\ 
     \varphi(\mathbf{u})=0\}.
   \end{equation}
\end{defn}

\begin{rem}
  Clearly, the concept of annihilators depends on the space $V$.
  So the complete notation would be $U_V^0$.
  However, we use $U^0$ if $V$ is clear from the context.
\end{rem}

\begin{exc}
  Let $\mathbf{e}_1, \mathbf{e}_2, \mathbf{e}_3, \mathbf{e}_4, \mathbf{e}_5$
   denote the standard basis of $V=\mathbb{R}^5$,
   and $\varphi_1, \varphi_2, \varphi_3, \varphi_4, \varphi_5$
   its dual basis of $V'$.
  Suppose 
  \begin{displaymath}
    U = \Span(\mathbf{e}_1, \mathbf{e}_2) = \{(x_1,x_2,0,0,0)\in
    \mathbb{R}^5: x_1, x_2\in \mathbb{R}\}.
  \end{displaymath}
  Show that $U^0 = \Span(\varphi_3, \varphi_4, \varphi_5)$.
\end{exc}
\begin{solution}
  First suppose $\varphi\in \Span(\varphi_3, \varphi_4, \varphi_5)$.
  \begin{displaymath}
    \forall u\in U, \exists c_1,c_2\in \mathbb{R} \text{ s.t. }
    u=a_1\mathbf{e}_1 + a_2\mathbf{e}_2,
  \end{displaymath}
  Then $\varphi=\sum_{i=3}^5 c_i \varphi_i$ implies $\varphi(u)=0$,
   i.e. $\varphi\in U^0$.

  Conversely, for any $\varphi\in V'$ 
   we have $\varphi=\sum_{i=1}^5 c_i \varphi_i$. 
  Any $\varphi\in U^0$ satisfies
  \begin{displaymath}
    \forall a_1,a_2\in \mathbb{R}, u:=a_1\mathbf{e}_1 + a_2\mathbf{e}_2 \in U, 
    \varphi(u) = 0,
  \end{displaymath}
  which implies $a_1=0$, $a_2=0$,
  i.e., $\varphi\in \Span(\mathbf{e}_3, \mathbf{e_4}, \mathbf{e}_5)$.
\end{solution}

\begin{exc}
  \label{exc:nullspaceOfInclusionDual}
  Let $i: U\hookrightarrow V$ be an inclusion.
  Show that $\Null i' = U^0$.
\end{exc}
\begin{solution}
  The signature of $i'$ is $i': V'\rightarrow U'$.
  Then we have
  \begin{align*}
    \Null i' &= \{\varphi\in V': i'(\varphi)= \varphi\circ i =0\}
               \\
             &= \{\varphi\in V': \forall u\in U,
               \varphi(u) =0\} = U^0,
  \end{align*}
  where the first line follows
  from Definitions \ref{def:nullSpace} and \ref{def:dualMap}.
  % where the last line does not change the fact
  %  that $\varphi$ acts on $V$.
% \begin{CJK*}{UTF8}{gbsn}
%   ``小样儿,你穿个马甲我就不认识你啦?''
% \end{CJK*}
\end{solution}

\begin{lem}
  \label{lem:annihilatorSubspace}
  Suppose $U\subset V$.
  Then $U^0$ is a subspace of $V'$.
\end{lem}

\begin{exc}
  \label{exc:linearMapExtension}
  Suppose $V$ is finite-dimensional.
  Prove every linear map on a subspace of $V$
   can be extended to a linear map on $V$.
\end{exc}

\begin{lem}
  \label{lem:dimAnnihilator}
  Suppose $V$ is finite-dimensional and $U$
  is a subspace of $V$.
  Then 
  \begin{equation}
    \label{eq:dimAnnihilator}
    \dim U + \dim U^0 = \dim V.
  \end{equation}
\end{lem}
\begin{proof}
  Apply Theorem \ref{thm:dimensionsOfNullAndRange}
   to the dual of an inclusion \mbox{$i': V'\rightarrow U'$}
   and we have
   \begin{align*}
     &\dim \Range i' + \dim \Null i' = \dim V'
       \\
     \Rightarrow & 
      \dim \Range i' + \dim U^0 = \dim V,
   \end{align*}
   where the second line follows from
    Exercise \ref{exc:nullspaceOfInclusionDual}
    and Lemma \ref{lem:dimV=dimV'}.
   For any $\varphi\in U'$,
    Exercise \ref{exc:linearMapExtension} states that
%    and Lemma \ref{lem:annihilatorSubspace} 
    $\varphi\in U'$ can be extended to $\psi\in V'$
    such that $i'(\psi)= \varphi$.
   Hence $i'$ is surjective
    and we have $U'=\Range i'$.
   The proof is then completed by 
    Lemma \ref{lem:dimV=dimV'}.
\end{proof}

\begin{lem}
  \label{lem:nullSpaceOfDualMap}
  Any linear map $T\in {\cal L}(V, W)$
  satisfies
  \begin{equation}
    \label{eq:nullSpaceOfDualMap}
    \Null T' = (\Range T)^0.
  \end{equation}
\end{lem}
\begin{proof}
  Definitions \ref{def:nullSpace}, \ref{def:range},
   \ref{def:dualMap}, and \ref{def:annihilator}
   yield
  \begin{align*}
    \varphi\in \Null T' 
    &\Leftrightarrow 0= T'(\varphi) = \varphi\circ T
    \\
    &\Leftrightarrow \forall v\in V, \varphi(Tv) = 0
    \\
    &\Leftrightarrow \varphi(\Range T) = 0
    \\
    &\Leftrightarrow \varphi\in (\Range T)^0.
      \qedhere
  \end{align*}
\end{proof}

\begin{lem}
  \label{lem:nullSpaceOfDualMap3parts}
  For finite-dimensional vector spaces $V$ and $W$,
   any linear map $T\in {\cal L}(V, W)$
   satisfies
   \begin{equation}
     \label{eq:nullSpaceOfDualMap3parts}
     \dim \Null T' = \dim \Null T + \dim W -\dim V.
   \end{equation}
\end{lem}
\begin{proof}
  Lemma \ref{lem:nullSpaceOfDualMap} and 
  Theorem \ref{thm:dimensionsOfNullAndRange}
  yield
  \begin{align*}
    \dim \Null T' &= \dim (\Range T)^0
    = \dim W - \dim (\Range T)
    \\
    &= \dim W - \dim V + \dim (\Null T)
    \\
    &=\dim \Null T + \dim W -\dim V.
      \qedhere
  \end{align*}
\end{proof}

\begin{coro}
  \label{coro:Tsurjective2dualTinjective}
  For finite-dimensional vector spaces $V$ and $W$,
   any linear map $T\in {\cal L}(V, W)$
   is surjective if and only if
   $T'$ is injective.
\end{coro}
\begin{proof}
  $T$ is surjective $\Leftrightarrow$ $W=\Range T$
  $\Leftrightarrow$ $(\Range T)^0 = \{0\}$
  $\Leftrightarrow$ $\Null T' = \{0\}$
  $\Leftrightarrow$ $T'$ is injective.
  The second step follows from 
   Lemma \ref{lem:dimAnnihilator} applied to $W$: 
   \begin{displaymath}
     \dim W =\dim (\Range T) + \dim (\Range T)^0.
     \qedhere
   \end{displaymath}
\end{proof}

\begin{lem}
  \label{lem:dimRangeOfDualMap}
  For finite-dimensional vector spaces $V$ and $W$,
   any linear map $T\in {\cal L}(V, W)$ satisfies
  \begin{equation}
    \label{eq:dimRangeOfDualMap}
    \dim \Range T' = \dim \Range T.
  \end{equation}
\end{lem}
\begin{proof}
  Theorem \ref{thm:dimensionsOfNullAndRange}, 
  Lemma \ref{lem:nullSpaceOfDualMap},
  and Lemma \ref{lem:dimAnnihilator}
  yield
  \begin{align*}
    \dim \Range T' &= \dim W - \dim \Null T' 
                     \\
                    &= \dim W - \dim (\Range T)^0
    \\
                   &= \dim (\Range T).
      \qedhere
  \end{align*}
\end{proof}

\begin{lem}
  \label{lem:rangeOfDualMap}
  For finite-dimensional vector spaces $V$ and $W$,
   any linear map $T\in {\cal L}(V, W)$ satisfies
  \begin{equation}
    \label{eq:rangeOfDualMap}
    \Range T' = (\Null T)^0.
  \end{equation}
\end{lem}
\begin{proof}
  Theorem \ref{thm:dimensionsOfNullAndRange}, 
  Lemma \ref{lem:nullSpaceOfDualMap},
  and Lemma \ref{lem:dimAnnihilator}
  yield
  \begin{align*}
    \varphi \in \Range T' 
    &\Rightarrow
      \exists \psi\in W' \text{ s.t }
      T'(\psi) = \varphi
    \\
    &\Rightarrow 
      \forall v\in \Null T, \varphi(v) = \psi(Tv) = 0
    \\
    &\Rightarrow 
      \varphi \in (\Null T)^0.
  \end{align*}
  The proof is completed by
  \begin{align*}
    \dim \Range T' &= \dim (\Range T)
    \\
                   &= \dim V - \dim \Null T 
    \\
                   &= \dim (\Null T)^0.
      \qedhere
  \end{align*}
\end{proof}

\begin{coro}
  \label{coro:Tinjective2dualTsurjective}
  For finite-dimensional vector spaces $V$ and $W$,
   any linear map $T\in {\cal L}(V, W)$
   is injective if and only if
   $T'$ is surjective.
\end{coro}
\begin{proof}
  $T$ is injective $\Leftrightarrow$ $\Null T=\{0\}$
  $\Leftrightarrow$ $(\Null T)^0 = V'$
  $\Leftrightarrow$ $\Range T' = V'$
  $\Leftrightarrow$ $T'$ is surjective.
  The second step follows from 
   Lemmas \ref{lem:dimAnnihilator} and \ref{lem:dimV=dimV'},
   and the third step follows from
   Lemma \ref{lem:rangeOfDualMap}.
\end{proof}


\subsubsection{Matrix ranks}
\label{sec:matrix-ranks}

\begin{rem}
  Each matrix $A\in \mathbb{F}^{m\times n}$
   induces four fundamental subspaces.
\end{rem}

\begin{defn}
  \label{def:columnAndRowSpaces}
  For a matrix $A\in \mathbb{F}^{m\times n}:
  \mathbb{F}^n\rightarrow \mathbb{F}^m$,
  its \emph{column space} (or range or image)
  consists of all linear combinations of its columns,
  its \emph{row space} (or coimage) is the column space of $A^T$,
  its \emph{null space} (or kernel) is the null space of $A$
  as a linear operator, 
  and the \emph{left null space} (or cokernel)
  is the null space of $A^T$.
\end{defn}

\begin{defn}
  \label{def:columnAndRowRanks}
  The \emph{column rank} and \emph{row rank}
   of a matrix $A\in \mathbb{F}^{m\times n}$
   is the dimension of its column space
   and row space, respectively.
\end{defn}

\begin{lem}
  \label{lem:dimRangeTisColumnRankOfTmat}
  Let $A_T$ denote the matrix of a linear operator $T\in {\cal L}(V,W)$.
  Then the column rank of $A_T$ is the dimension of $\Range T$.
\end{lem}
\begin{proof}
  For $\mathbf{u}=\sum_i c_i\mathbf{v}_i$,
   Corollary \ref{coro:matrixOfLinearMap} yields
   \begin{align*}
     T\mathbf{u} = \sum_i c_iT\mathbf{v}_i = T [\mathbf{v}_1, \ldots, \mathbf{v}_n] \mathbf{c}
     = [\mathbf{w}_1, \ldots, \mathbf{w}_m] A_T \mathbf{c}.
   \end{align*}
  Hence we have
  \begin{displaymath}
    \{T\mathbf{u}: \mathbf{c}\in \mathbb{F}^n\}
    = \{[\mathbf{w}_1, \ldots, \mathbf{w}_m] A_T \mathbf{c}:
    \mathbf{c}\in \mathbb{F}^n\}.
  \end{displaymath}
  The LHS is $\Range T$ while
   $\{A_T \mathbf{c}: \mathbf{c}\in \mathbb{F}^n\}$
   is the column space of $A_T$.
  Since $(\mathbf{w}_1, \ldots, \mathbf{w}_m)$
   is a basis,
   by Definition \ref{def:columnAndRowRanks}
   the column rank of the matrix 
   $[\mathbf{w}_1, \ldots, \mathbf{w}_m]$ is $m$.
  Taking $\dim$ to both sides of the above equation
   yields the conclusion.
  Note that the RHS is a subspace of $\mathbb{F}^m$ (why?)
   and the dimension of it
   does not depend on the special choice of its basis,
   hence we can choose $(\mathbf{w}_1, \ldots, \mathbf{w}_m)$
   to be the standard basis
   and then $[\mathbf{w}_1, \ldots, \mathbf{w}_m]$ is
   simply the identity matrix.
\end{proof}

\begin{thm}
  \label{thm:rowRankIsColRank}
  For any $A\in \mathbb{F}^{m\times n}$,
   its row rank equals its column rank.
\end{thm}
\begin{proof}
  Define a linear map
   $T: \mathbb{F}^{n}\rightarrow\mathbb{F}^{m}$ as
   $T \mathbf{x} = A \mathbf{x}$.
   % \begin{displaymath}
   %   T \mathbf{x} = A \mathbf{x},
   % \end{displaymath}
  Clearly, $A$ is the matrix of $T$
   for the standard bases of $\mathbb{F}^n$
   and $\mathbb{F}^m$.
  Then we have,
  \begin{align*}
    \text{column rank of } A 
    &= \dim \Range T
    \\
    &= \dim \Range T'
    \\
    &= \text{column rank of the matrix of } T'
    \\
    &=\text{column rank of } A^T
    \\
    &=\text{row rank of } A,
  \end{align*}
  where the first step follows
  from Lemma \ref{lem:dimRangeTisColumnRankOfTmat},
  the second from Lemma \ref{lem:dimRangeOfDualMap},
  the third from Lemma \ref{lem:dimRangeTisColumnRankOfTmat},
  the fourth from Theorem \ref{thm:matrixOfDualMap},
  and the last from the definition of 
  matrix transpose and matrix products.
\end{proof}

\begin{defn}
  \label{def:matrixRank}
  The \emph{rank} of a matrix is its column rank.
\end{defn}

\begin{rem}
  Since we will not have the notion of inner products
  until Section \ref{sec:inner-product-spaces}, 
  we choose to state Theorem \ref{thm:fundamentalLinearAlgebra}
  for the special case of $\mathbb{F}=\mathbb{R}$.
\end{rem}

\begin{thm}[Fundamental theorem of linear algebra]
  \label{thm:fundamentalLinearAlgebra}
  For a matrix $A\in \mathbb{R}^{m\times n}:
  \mathbb{R}^n\rightarrow \mathbb{R}^m$,
  its column space and row space both have dimension
  $r\le \min(m,n)$;
  its null space and left null space
   have dimensions $n-r$ and $m-r$,
   respectively.
  In addition, we have
  \begin{subequations}
    \label{eq:fundamentalLinearAlgebra}
    \begin{align}
      \mathbb{R}^m &= \Range A \oplus \Null A^T,
      \\
      \mathbb{R}^n &= \Range A^T \oplus \Null A,
    \end{align}
  \end{subequations}
  where $\Range A\perp \Null A^T$ and
  $\Range A^T \perp \Null A$.
\end{thm}
\begin{proof}
  The first sentence is a rephrase of
   Theorem \ref{thm:rowRankIsColRank}
   and follows from Theorem \ref{thm:dimensionsOfNullAndRange}. 
  For the second sentence,
   we only prove (\ref{eq:fundamentalLinearAlgebra}b).
  $\mathbf{x}\in \Null A$ implies
  $\mathbf{x}\in \mathbb{R}^n$
  and $A\mathbf{x}=\mathbf{0}$.
  The latter expands to
  \begin{displaymath}
    \begin{bmatrix}
      \mathbf{a}^T_1
      \\
      \mathbf{a}^T_2
      \\
      \cdots
      \\
      \mathbf{a}^T_m
    \end{bmatrix}
    \mathbf{x}
    =
    \begin{bmatrix}
      0
      \\
      0
      \\
      \cdots
      \\
      0
    \end{bmatrix},
  \end{displaymath}
  which implies that $\forall j=1,2,\ldots, m$,
  $\mathbf{a}_j \perp \mathbf{x}$.
  Hence
  $\mathbf{x}$ is orthogonal to each basis vector
  of $\Range A^T$.
  The rest of the proof follows from
   Lemma \ref{lem:dimRangeOfDualMap},
   Theorem \ref{thm:matrixOfDualMap}, 
   Theorem \ref{thm:dimensionsOfNullAndRange}. 
\end{proof}

\begin{rem}
  Each $\mathbf{x}\in \mathbb{R}^n$
  can be split into a row space component $\mathbf{x}_r$
  and a null space component $\mathbf{x}_n$.
  Then $A \mathbf{x} = A \mathbf{x}_r \in \Range A$.
  \emph{Every vector goes to the column space!}
  Furthermore,
  \emph{every vector in the column space
    comes from one and only one vector in the row space}.
\end{rem}


\section{Eigenvalues, eigenvectors, and invariant subspaces}
\label{sec:eigenv-eigenv-invar}

\subsection{Invariant subspaces}

\begin{defn}
  \label{def:invariantSubspace}
  Under a linear operator $T\in {\cal L}({\cal V})$,
  a subspace ${\cal U}$ of ${\cal V}$ is \emph{invariant}
  if $\mathbf{u}\in {\cal U}$ implies $T \mathbf{u}\in {\cal U}$.
\end{defn}

\begin{exm}
  Under $T\in {\cal L}({\cal V})$,
  each of the following subspaces of ${\cal V}$
  is invariant:
  $\{\mathbf{0}\}$, ${\cal V}$,
  $\Null\, T$, and $\Range\, T$.
\end{exm}

\begin{defn}
  \label{def:eigenPairs}
  A number $\lambda\in \mathbb{F}$
  is called an \emph{eigenvalue of an operator}
  $T\in {\cal L}({\cal V})$ if
  there exists $\mathbf{v}\in {\cal V}$ such that
  $T \mathbf{v} = \lambda \mathbf{v}$
  and $\mathbf{v}\ne \mathbf{0}$.
  Then the vector $\mathbf{v}$ is called an \emph{eigenvector}
  of $T$ corresponding to $\lambda$.
\end{defn}

\begin{lem}
  \label{lem:eigenValMakesTmlINotInjective}
  Suppose $V$ is finite-dimensional.
  $\lambda\in V$ is an eigenvalue of $T\in {\cal L}(V)$
  if and only if $T-\lambda I$ is not injective.
\end{lem}
\begin{proof}
  This follows directly from Definition \ref{def:eigenPairs}.
\end{proof}

\begin{exm}
  For each eigenvector $\mathbf{v}$ of $T\in {\cal L}({\cal V})$,
  the subspace $\Span(\mathbf{v})$ is a one-dimensional
  invariant subspace of ${\cal V}$.
\end{exm}

\begin{lem}
  \label{lem:distinctEigenValImpliesIndependentEigenVec}
  Suppose $\lambda_1, \ldots, \lambda_m$ are distinct eigenvalues
  of $T\in {\cal L}({\cal V})$
  with corresponding eigenvectors $\mathbf{v}_1, \ldots, \mathbf{v}_m$.
  Then $\mathbf{v}_1, \ldots, \mathbf{v}_m$ is linearly independent.
\end{lem}

\begin{lem}
  Suppose ${\cal V}$ is finite-dimensional.
  Then each operator on ${\cal V}$ has at most $\dim {\cal V}$ distinct eigenvalues.
\end{lem}

\begin{defn}
  \label{def:restrictedLinearOp}
  Suppose $T\in{\cal L}(V)$ and $U$ is an invariant subspace of $V$
  under $T$.
  The \emph{restriction operator} $T|_U\in {\cal L}(U)$
  is defined by
  \begin{equation}
    \label{eq:restrictedLinearOp}
    \forall \mathbf{u}\in U,\quad T|_U (\mathbf{u}) = T\mathbf{u}.
  \end{equation}
  % The \emph{quotient operator} $T/U\in {\cal L}(V/U)$
  % is defined by
  % \begin{equation}
  %   \label{eq:quotientLinearOp}
  %   \forall \mathbf{v}\in V,\quad (T/U)(\mathbf{v}+U) = T\mathbf{v} + U.
  % \end{equation}
\end{defn}

\begin{rem}
  The behavior of an operator $T\in {\cal L}({\cal V})$
  is understood in a simple way
  if there exists a decomposition
  \begin{displaymath}
    {\cal V} = {\cal U}_1 \oplus {\cal U}_2 \oplus \cdots \oplus {\cal U}_m;
  \end{displaymath}
  then $T$ can be thought of as a Cartesion product 
  of restricted operators
  $T|_{{\cal U}_i}$, whose action is contained in ${\cal U}_i$.
\end{rem}

\subsection{Existence of eigenvalues}

\begin{ntn}
  \label{ntn:polyOfAnOp}
  Suppose $T\in {\cal L}({\cal V})$ and $p\in \mathbb{P}(\mathbb{F})$
  is a polynomial given by
  \begin{displaymath}
    p(z) = a_0 + a_1 z + \cdots + a_m z^m
  \end{displaymath}
  for $z\in \mathbb{F}$.
  Then $p(T)$ is the operator given by
  \begin{displaymath}
    p(T) = a_0I + a_1 T + \cdots + a_m T^m,
  \end{displaymath}
  where $I=T^0$ is the identity operator.
\end{ntn}

\begin{exm}
  Suppose $D\in {\cal L}(\mathbb{P}(\mathbb{R}))$
  is the differentiation operator defined
  by $Dq =q'$
  and $p$ is the polynomial defined by
  $p(x) = 7-3x + 5x^2$.
  Then we have
  \begin{displaymath}
    p(D) = 7-3D+5D^2, \qquad
    (p(D))q = 7q - 3q' + 5 q''.
  \end{displaymath}
\end{exm}

\begin{defn}
  \label{def:productPoly}
  The \emph{product polynomial} of two polynomials
  $p,q\in \mathbb{P}(\mathbb{F})$
  is the polynomial defined by
  \begin{equation}
    \label{eq:productPoly}
    \forall z\in \mathbb{F},\qquad
    (pq)(z) := p(z) q(z).
  \end{equation}
\end{defn}

\begin{lem}
  \label{lem:polyOpsCommutes}
  Any $T\in {\cal L}({\cal V})$ and $p,q\in \mathbb{P}(\mathbb{F})$
  satisfy
  \begin{equation}
    \label{eq:polyOpsCommutes}
      (pq)(T) = p(T)q(T) = q(T)p(T).
  \end{equation}
\end{lem}

\begin{thm}[Existence of eigenvalues]
  \label{thm:existenceOfEigenvalues}
  Every operator $T\in {\cal L}(V)$ on a finite-dimensional,
  nonzero, complex vector space $V$
  has an eigenvalue.
\end{thm}
\begin{proof}
  Write $n:=\dim V$.
  For a nonzero $\mathbf{v}\in V$,
  the $n+1$ vectors
  $(\mathbf{v}, T\mathbf{v}, T\mathbf{v}, \cdots, T^n\mathbf{v})$
  must be linear dependent, i.e., 
  \begin{displaymath}
    \mathbf{0}
    = a_0 \mathbf{v} + a_1T\mathbf{v} + \cdots + a_n T^n\mathbf{v}
    = (a_0 + a_1T + \cdots + a_n T^n)\mathbf{v}
  \end{displaymath}
  implies that there exists $j\in [1,n]$
  such that $a_j\ne 0$.
  By the fundamental theorem of algebra,
  the polynomial $\sum_{i=0}^n a_i T^i$
  has $m$ roots, say, $\lambda_1, \ldots, \lambda_m$,
  and thus for some $c\in\mathbb{C}\setminus\{0\}$,
  we have
  \begin{displaymath}
    \mathbf{0}=c (T-\lambda_1 I)(T-\lambda_2 I)\cdots (T-\lambda_m I)\mathbf{v}.
  \end{displaymath}
  Hence $T-\lambda_jI$ is not injective for some $\lambda_j$
  and the proof is completed by Lemma \ref{lem:eigenValMakesTmlINotInjective}.
\end{proof}

\subsection{Upper-triangular matrices}

\begin{defn}
  The \emph{matrix of a linear operator} $T\in {\cal L}({\cal V})$
  is the matrix of the linear map $T\in {\cal L}({\cal V}, {\cal V})$,
  c.f. Definition \ref{def:matrixOfLinearMap}.
\end{defn}

\begin{thm}
  \label{thm:upperTriangAndInvariantSubspaces}
  Suppose $T\in {\cal L}({\cal V})$
  and $\mathbf{v}_1, \ldots, \mathbf{v}_n$
  is a basis of ${\cal V}$.
  Then the following are equivalent:
  \begin{enumerate}[(a)]\itemsep0em
  \item the matrix of $T$ with respect to
    $\mathbf{v}_1, \ldots, \mathbf{v}_n$
    is upper triangular;
  \item $T\mathbf{v}_j\in \Span(\mathbf{v}_1, \ldots, \mathbf{v}_j)$
    for each $j=1, \ldots, n$;
  \item $\Span(\mathbf{v}_1, \ldots, \mathbf{v}_j)$
    is invariant under $T$
    for each $j=1, \ldots, n$.
  \end{enumerate}
\end{thm}

\begin{thm}
  \label{thm:linearOpHasTriangularMatrix}
  Every linear operator $T\in {\cal L}({\cal V})$
  on a finite-dimensional complex vector space ${\cal V}$ 
  has an upper-triangular matrix
  with respect to some basis of ${\cal V}$.
\end{thm}

\begin{thm}
  Suppose $T\in {\cal L}({\cal V})$ has an upper-triangular matrix
  with respect to some basis of ${\cal V}$.
  Then $T$ is invertible if and only if
  all the entries on the diagonal of
  that upper-triangular matrix are nonzero.
\end{thm}

\begin{thm}
  Suppose $T\in {\cal L}({\cal V})$ has an upper-triangular matrix
  with respect to some basis of ${\cal V}$.
  Then the eigenvalues of $T$ are precisely the entries
  on the diagonal of that upper-triangular matrix.
\end{thm}

\subsection{Eigenspaces and diagonal matrices}
\label{sec:eigensp-diag-matr}

\begin{defn}
  A \emph{diagonal entry} of a matrix
  is an entry of the matrix of which the row index
  equals the column index.
  The \emph{diagonal} of a matrix
  consists of all diagonal entries of the matrix.
  A \emph{diagonal matrix} is a matrix
  that is zero everywhere except possibly
  along the diagonal.
\end{defn}

\begin{defn}
  \label{def:eigenSpace}
  The \emph{eigenspace} of $T\in {\cal L}({\cal V})$
  corresponding to $\lambda\in \mathbb{F}$ is
  \begin{equation}
    \label{eq:eigenSpace}
    E(\lambda, T) := \Null(T-\lambda I).
  \end{equation}
\end{defn}

\begin{rem}
  In Definition \ref{def:eigenSpace}, 
  $\lambda$ is an eigenvalue of $T$
  if and only if $E(\lambda,T)\ne \{\mathbf{0}\}$.
\end{rem}

\begin{rem}
  The eigenspace $E(\lambda,T)$
  is a subspace of ${\cal V}$
  because the null space of any linear operator on ${\cal V}$ is.
\end{rem}

\begin{lem}
  Suppose $\lambda_1, \ldots, \lambda_m$ are distinct eigenvalues
  of $T\in {\cal L}({\cal V})$
  on a finite-dimensional space ${\cal V}$.
  Then
  \begin{displaymath}
    E(\lambda_1, T) + \cdots + E(\lambda_m, T) 
  \end{displaymath}
  is a direct sum and
  \begin{equation}
    \dim E(\lambda_1, T) + \cdots + \dim E(\lambda_m, T) \le \dim {\cal V}.
  \end{equation}
\end{lem}

\begin{defn}
  An operator $T\in {\cal L}({\cal V})$ is \emph{diagonalizable}
  if it has a diagonal matrix with respect to some basis of ${\cal V}$.
\end{defn}

\begin{thm}[Conditions of diagonalizability]
  \label{thm:diagonalizabilityConditions}
  Suppose $\lambda_1, \ldots, \lambda_m$ are distinct eigenvalues
  of $T\in {\cal L}({\cal V})$
  on a finite-dimensional space ${\cal V}$.
  Then the following are equivalent:
  \begin{enumerate}[(a)]\itemsep0em
  \item $T$ is diagonalizable;
  \item ${\cal V}$ has a basis consisting of eigenvectors of $T$;
  \item there exist one-dimensional subspaces
    $U_1, \ldots, U_n$ of ${\cal V}$,
    each invariant under $T$, such that
%    \begin{displaymath}
      ${\cal V} = U_1 \oplus \cdots \oplus U_n$;
%    \end{displaymath}
  \item ${\cal V} = E(\lambda_1,T) \oplus \cdots \oplus E(\lambda_m,T)$; 
  \item $\dim {\cal V} = \dim E(\lambda_1,T) + \cdots + \dim E(\lambda_m,T)$.
  \end{enumerate}
\end{thm}

\begin{coro}
  An operator $T\in {\cal L}({\cal V})$ is diagonalizable
  if $T$ has $\dim {\cal V}$ distinct eigenvalues.
\end{coro}


% \clearpage
\section{Operators on complex vector spaces}
\label{sec:oper-compl-vect}

\subsection{Generalized eigenvectors}
\label{sec:gener-eigenv}

\begin{lem}
  \label{lem:nullSpacePowerInclusion}
  For a linear operator $T\in {\cal L}(V)$, we have
  \begin{equation}
    \label{eq:nullSpacePowerInclusion}
    \{\mathbf{0}\} = \Null\, T^0
    \subseteq \Null\,T^1 \subseteq \cdots
    \subseteq \Null\,T^k \subseteq \Null\, T^{k+1}  \subseteq \cdots .
  \end{equation}
\end{lem}
\begin{proof}
  Suppose $\mathbf{v}\in\Null\, T^k$ for $k\in\mathbb{N}$.
  Thus $T^k\mathbf{v}=0$.
  Then $T^{k+1}\mathbf{v}= T T^k\mathbf{v}=T\mathbf{0}=\mathbf{0}$
  and therefore $\mathbf{v}\in\Null\, T^{k+1}$.
\end{proof}

\begin{lem}
  \label{lem:nullSpaceEqual}
  Suppose a linear operator $T\in {\cal L}(V)$ satisfies
  $\Null\,T^m = \Null\, T^{m+1}$. Then we have
  \begin{equation}
    \label{eq:nullSpaceEqual}
    \Null\,T^m = \Null\, T^{m+1} = \Null\, T^{m+2}
    = \cdots .
  \end{equation}
\end{lem}
\begin{proof}
  By Lemma \ref{lem:nullSpacePowerInclusion},
  it suffices to show
  \begin{displaymath}
    \forall k\in \mathbb{N}^+,\quad
    \Null\, T^{m+k+1}\subseteq \Null\, T^{m+k}, 
  \end{displaymath}
  which indeed holds because
  \begin{displaymath}
    \begin{array}{rl}
      \mathbf{v}\in \Null\,T^{m+k+1} \Rightarrow
      & T^{m+k+1}\mathbf{v}=\mathbf{0}\ \Rightarrow\ 
        T^{m+1}(T^k\mathbf{v})=\mathbf{0}\\
      \Rightarrow &
                    T^k \mathbf{v}\in \Null\,T^{m+1}=\Null\,T^{m}
      \\
      \Rightarrow &
                    T^{m}T^k\mathbf{v}=\mathbf{0}\ \Rightarrow\ 
                    T^{m+k}\mathbf{v}=\mathbf{0}\\
      \Rightarrow &
      \mathbf{v}\in \Null\,T^{m+k}. \qedhere
    \end{array}
  \end{displaymath}
\end{proof}

\begin{lem}
  \label{lem:nullSpaceStopGrowing}
  A linear operator $T\in {\cal L}(V)$ satisfies
  \begin{equation}
    \label{eq:nullSpaceStopGrowing}
    \Null\,T^n = \Null\, T^{n+1} = \Null\, T^{n+2}
    = \cdots ,
  \end{equation}
  where $n=\dim V$.
\end{lem}
\begin{proof}
  By Lemma \ref{lem:nullSpaceEqual}, it suffices to show
  \begin{displaymath}
    \Null\,T^n = \Null\, T^{n+1}.
  \end{displaymath}
%  \mbox{$\Null\,T^n = \Null\, T^{n+1}$}. 
  Suppose this is not true.
  Then Lemmas \ref{lem:nullSpacePowerInclusion}
  and \ref{lem:nullSpaceEqual} yield
  \begin{displaymath}
    \{\mathbf{0}\}=\Null\,T^0 \subset \Null\, T^{1}
    \subset \cdots \subset \Null\, T^{n} \subset \Null\, T^{n+1},
  \end{displaymath}
  where the symbol``$\subset$'' means strict inclusion,
  c.f. Definition \ref{def:subsets}.
  At each strict inclusion in the above chain,
  the dimension of the space increases by at least 1,
  and thus $\dim \Null\, T^{n+1}> n$.
  But the dimension of any subspace of $V$
  cannot exceed that of $V$.
\end{proof}

\begin{thm}
  \label{thm:directSumOfNullNandRangeN}
  A linear operator $T\in {\cal L}(V)$ satisfies
  \begin{equation}
    \label{eq:directSumOfNullNandRangeN}
    V = \Null\,T^n \oplus \Range\, T^{n}
  \end{equation}
  where $n=\dim V$.
\end{thm}
\begin{proof}
  We first show
%  \begin{displaymath}
  $\Null\,T^n \cap \Range\, T^{n}=\{\mathbf{0}\}$.
  % \end{displaymath}
  Indeed, if $\mathbf{v}\in \Null\,T^n \cap \Range\, T^{n}$,
  then $T^n\mathbf{v}=\mathbf{0}$
  and there exists $\mathbf{u}$ such that $\mathbf{v}=T^n \mathbf{u}$.
  Hence $T^{2n}\mathbf{u}=\mathbf{0}$.
  Lemma \ref{lem:nullSpaceStopGrowing} further implies
  $T^{n}\mathbf{u}=\mathbf{0}$ and thus $\mathbf{v}=\mathbf{0}$.

  By Theorem \ref{thm:directSumIFFintersection0},
  $\Null\,T^n + \Range\, T^{n}$ is a direct sum.
  Then (\ref{eq:directSumOfNullNandRangeN}) follows from
  \begin{displaymath}
    \begin{array}{rl}
    \dim(\Null\,T^n \oplus \Range\, T^{n})
      &= \dim \Null\,T^n + \dim \Range\, T^{n} \\
      &=\dim V,
    \end{array}
  \end{displaymath}
  where the second step follows
  from the fundamental theorem of linear maps
  (Theorem \ref{thm:dimensionsOfNullAndRange}).
\end{proof}

\begin{exm}
  For the operator $T\in {\cal L}(\mathbb{C}^3)$ given by
  \begin{equation}
    \label{eq:generalizedEigenvectorExmT}
    T(z_1, z_2, z_3) = (4z_2, 0, 5z_3),
  \end{equation}
  $\Null\, T+ \Range\, T$ is not a direct sum of $\mathbb{C}^3$
  because
  \begin{displaymath}
    \begin{array}{l}
      \Null\, T = \{(z_1, 0, 0): z_1\in \mathbb{C}\},
      \\
      \Range\, T = \{(z_1, 0, z_3): z_1,z_3\in \mathbb{C}\}.
    \end{array}
  \end{displaymath}
  In contrast, $T^3(z_1, z_2, z_3) = (0, 0, 125z_3)$
  and thus
  \begin{displaymath}
    \begin{array}{l}
      \Null\, T^3 = \{(z_1, z_2, 0): z_1, z_2\in \mathbb{C}\},
      \\
      \Range\, T^3 = \{(0, 0, z_3): z_3\in \mathbb{C}\},
      \\
      \Null\, T^3 \oplus \Range\, T^3 = \mathbb{C}^3.
    \end{array}
  \end{displaymath}
\end{exm}

\begin{rem}
  The motivation of eigenvectors and generalized eigenvectors
  is to describe an operator by decomposing its domain
  into invariant subspaces.
  For $T\in {\cal L}(V)$, we would like to describe $T$
  by finding a nice direct-sum decomposition
  \begin{displaymath}
     V = U_1 \oplus U_2 \oplus \cdots \oplus U_m,
  \end{displaymath}
  where each $U_i$ is a subspace of $V$ invariant under $T$.
  The simplest possible nonzero subspaces are 1-dimensional
  and such a decomposition happens if and only if
  $V$ has a basis that consists of eigenvectors of $T$.
  By Theorem \ref{thm:diagonalizabilityConditions},
  this happens if and only if $V$ has an eigenspace decomposition
  \begin{displaymath}
    V = E(\lambda_1, T) \oplus E(\lambda_2, T)
    \oplus \cdots \oplus E(\lambda_m, T),
  \end{displaymath}
  where $\lambda_k$'s are distinct eigenvalues of $T$.
  If $V$ is an inner product space,
  the above eigenspace decomposition holds
  for every normal operator if $\mathbb{F}=\mathbb{C}$
  and for every self-adjoint operator if $\mathbb{F}=\mathbb{R}$;
  see Theorems \ref{thm:complexSpectralThm}
  and \ref{thm:realSpectralThm}.

  Unfortunately, for a more general operator $T$
  it might be impossible to decompose
  $V$ into the eigenspaces of $T$.
  For example, the operator $T\in {\cal L}(\mathbb{C}^2)$
  given by $T(y,z)=(z,0)$ has
  $E(0,T)=\{(y,0)\in\mathbb{C}^2: y\in \mathbb{C}\}$
  and $0$ is the only eigenvalue of $T$.

  Generalized eigenvectors and generalized eigenspaces
  are introduced so that any operator $T$
  can be decomposed as the direct sum
  of its generalized eigenspaces.
\end{rem}

\begin{defn}
  \label{def:generalizedEigenvector}
  A \emph{generalized eigenvector of a linear operator}
  $T\in{\cal L}(V)$ corresponding to the eigenvalue $\lambda$ of $T$
  is a nonzero vector $\mathbf{v}\in V$ satisfying
  \begin{equation}
    \label{eq:generalizedEigenvector}
    \exists j\in \mathbb{N}^+     \text{ s.t. } 
    (T-\lambda I)^j \mathbf{v} = \mathbf{0}.
  \end{equation}
\end{defn}

\begin{defn}
  \label{def:generalizedEigenspace}
  The \emph{generalized eigenspace of a linear operator}
  $T\in{\cal L}(V)$ corresponding to the eigenvalue $\lambda$ of $T$,
  denoted $G(\lambda,T)$, 
  is the set of all generalized eigenvectors of $T$ corresponding
  to $\lambda$ along with the zero vector.
\end{defn}

\begin{rem}
  Any generalized eigenspace of $T\in{\cal L}(V)$
  is a subspace of $V$.
\end{rem}

\begin{rem}
  In contrast to generalized eigenvectors,
  a ``generalized'' eigenvalue would be the same as an eigenvalue,
  hence we do not define such a concept.
\end{rem}

\begin{lem}
  \label{lem:generalizedEigenspace}
  A generalized eigenspace $G(\lambda, T)$ satisfies
  \begin{equation}
    \label{eq:generalizedEigenspace}
    \forall T\in{\cal L}(V), \forall \lambda\in \mathbb{F},\ 
    G(\lambda, T) = \Null(T-\lambda I)^{\dim V}.
  \end{equation}
\end{lem}
\begin{proof}
  Suppose $\mathbf{v}\in \Null(T-\lambda I)^{\dim V}$.
  Then Definitions \ref{def:generalizedEigenvector}
  and \ref{def:generalizedEigenspace} imply
  $\mathbf{v}\in G(\lambda,T)$.
  Conversely, $\mathbf{v}\in G(\lambda,T)$ implies
  that $(T-\lambda I)^j \mathbf{v} = \mathbf{0}$
  for some $j\in \mathbb{N}^+$.
  Then we have $\mathbf{v}\in \Null(T-\lambda I)^{\dim V}$
  from Lemmas \ref{lem:nullSpacePowerInclusion}
  and \ref{lem:nullSpaceStopGrowing}.
\end{proof}

\begin{defn}
  \label{def:eigenvalueMultiplicity}
  The \emph{multiplicity or algebraic multiplicity
    of an eigenvalue} $\lambda$ of an operator $T$
  is the dimension of the corresponding generalized eigenspace,
  \begin{equation}
    \label{eq:algebraicMultiplicity}
    m_a(\lambda) := \dim G(\lambda, T)
    = \dim \Null(T-\lambda I)^{\dim V}
  \end{equation}
  while the \emph{geometric multiplicity
    of an eigenvalue} $\lambda$ of $T$
  is the dimension of the corresponding eigenspace,
  \begin{equation}
    \label{eq:geometricMultiplicity}
    m_g(\lambda) := \dim E(\lambda, T)
    = \dim \Null(T-\lambda I).
  \end{equation}
  The \emph{index of an eigenvalue} $\lambda$ of $T$
  is the smallest integer $k$ for which
  \begin{equation}
    \Null\,(T-\lambda I)^k = \Null\,(T-\lambda I)^{k+1}.
  \end{equation}
\end{defn}

\begin{coro}
  \label{coro:geoNotExceedAlg}
  Geometric multiplicity and algebraic multiplicity satisfy
  \begin{equation}
    \label{eq:geoNotExceedAlg}
    1 \le m_g(\lambda) \le m_a(\lambda).
  \end{equation}
\end{coro}
\begin{proof}
  The case $j=1$ in Definition \ref{def:generalizedEigenvector}
  implies that
  every eigenvector of $T$ is a generalized eigenvector of $T$.
  Hence 
  \begin{displaymath}
    \forall T\in{\cal L}(V),\ \forall \lambda\in \mathbb{F},\ 
    E(\lambda,T)\subseteq G(\lambda, T).
  \end{displaymath}
  Then Definition \ref{def:eigenvalueMultiplicity} completes the proof.
\end{proof}

\begin{defn}
  An eigenvalue $\lambda$ of $A$ is \emph{defective} iff
  \begin{equation}
    \label{eq:defectEigenvalue}
    m_g(\lambda) < m_a(\lambda).
  \end{equation}
  $A$ is \emph{defective}
  iff $A$ has one or more defective eigenvalues.
\end{defn}

\begin{exm}
  \label{exm:multiplicities}
  Eigenvalues of the operator $T\in {\cal L}(\mathbb{C}^3)$ in
  (\ref{eq:generalizedEigenvectorExmT}) are 0 and 5,
  with the corresponding eigenspaces as
  \begin{displaymath}
    \begin{array}{l}
      E(0,T) = \{(z_1,0,0): z_1\in \mathbb{C}\}, \\
      E(5,T) = \{(0,0,z_3): z_3\in \mathbb{C}\}.
    \end{array}
  \end{displaymath}
  The generalized eigenspaces are deduced as follows,
  \begin{displaymath}
    \begin{array}{l}
      T^3(z_1, z_2, z_3) = (0, 0, 125 z_3), 
      \\
      G(0, T) = \{(z_1, z_2, 0): z_1, z_2\in \mathbb{C}\},
      \\
      (T-5I)^3(z_1, z_2, z_3) =
      (-125 z_1+300 z_2, -125 z_2, 0),
      \\
      G(5,T) = E(5,T) = \{(0,0,z_3): z_3\in \mathbb{C}\}.
    \end{array}
  \end{displaymath}
  Since $m_g(5)=1=m_a(5)$ and $m_g(0)=1<m_a(0)=2$, 
  the eigenvalue 5 is not defective
  but the eigenvalue 0 is. 
  Hence the operator $T$ is defective.
\end{exm}

\begin{lem}
  \label{lem:generalizedEigenvecsLinearIndep}
  Generalized eigenvectors of distinct eigenvalues
  of an operator $T\in {\cal L}(V)$
  are linearly independent.
\end{lem}
\begin{proof}
  Let $(\lambda_i,\mathbf{v}_i)$ be
  a generalized eigenpair of $T$.
  Define
  \begin{displaymath}
    \mathbf{w} := (T-\lambda_1I)^k\mathbf{v}_1,
  \end{displaymath}
  where $k\in \mathbb{N}$ is the largest integer
  such that $\mathbf{w}\ne \mathbf{0}$.
  By Definition \ref{def:generalizedEigenvector},
  $(T-\lambda_1I)\mathbf{w}=\mathbf{0}$
  and thus $\mathbf{w}$ is an eigenvector of $T$.
  Consequently, we have
  \begin{displaymath}
    (*):\quad \forall j\in \mathbb{N}^+,\ \forall \lambda\in \mathbb{C},\quad
    (T-\lambda I)^j \mathbf{w} = (\lambda_1-\lambda)^j \mathbf{w}.
  \end{displaymath}
  
  Write $n:=\dim V$ and define a polynomial of $T$ by
  \begin{displaymath}
    p(T) = (T-\lambda_1 I)^k(T-\lambda_2I)^n\cdots(T-\lambda_mI)^n; 
  \end{displaymath}
  By Lemma \ref{lem:polyOpsCommutes}, the factors of $p(T)$ commute.
  
  Suppose $\mathbf{0} = \sum_{i=1}^m a_i \mathbf{v}_i$
  where each $a_i\in \mathbb{C}$.
  Then the application of $p(T)$ to this equation yields $a_1=0$
  because of the distinctness of the eigenvalues and
  \begin{displaymath}
    \begin{array}{rl}
    \mathbf{0} &= a_1(T-\lambda_1 I)^k(T-\lambda_2I)^n\cdots(T-\lambda_mI)^n\mathbf{v}_1\\
      &= a_1(T-\lambda_2I)^n\cdots(T-\lambda_mI)^n\mathbf{w}\\
      &= a_1(\lambda_1-\lambda_2)^n\cdots(\lambda_1-\lambda_m)^n \mathbf{w},
    \end{array}
  \end{displaymath}
  where the first equality follows from Lemma \ref{lem:generalizedEigenspace}, 
  the second from the definition of $\mathbf{w}$, 
  and the third from $(*)$.
  
  Similarly,
  the above arguments applied to
  the other generalized eigenpairs
  yield $a_2=\cdots=a_m=0$,
  which implies the linear independence of generalized eigenvectors.
\end{proof}

\subsection{Nilpotent operators}
\label{sec:nilpotent-operators}

\begin{defn}
  \label{def:nilpotentOp}
  An operator $N\in{\cal L}(V)$
  is \emph{nilpotent} iff $N^k=\mathbf{0}$
  for some $k\in \mathbb{N}^+$.
\end{defn}

\begin{exm}
  The differentiation operator
  on the vector space of polynomials
  of degree at most $m$
  is nilpotent.
  The operator $N\in{\cal L}(\mathbb{F}^3)$
  given by $N(x,y,z)=(y,z,0)$
  is nilpotent because $N^3=\mathbf{0}$.
\end{exm}

\begin{lem}
  \label{def:nilpotentOpHighestPower}
  A nilpotent operator $N\in {\cal L}(V)$
  satisfies
  \begin{equation}
    \label{eq:nilpotentOpHighestPower}
    N^{\dim V}=\mathbf{0}.
  \end{equation}
\end{lem}
\begin{proof}
  Definitions \ref{def:nilpotentOp} and
  \ref{def:generalizedEigenspace} imply
  $G(0, N)=V$.
  The rest follows from Lemma \ref{lem:generalizedEigenspace}.
\end{proof}

\begin{lem}
  \label{lem:matrixOfNilpotentOperators}
  Any nilpotent operator $N\in {\cal L}(V)$
  has a strictly upper triangular matrix $M$, 
  % there exists a basis of $V$ with respect to which
  % the  of $N$ is 
  i.e., $\forall i\ge j$, $M_{i,j}=0$.
\end{lem}
\begin{proof}
  Write $n:=\dim V$.
  Choose a basis of $\Null\, N$,
  extend it to a basis of $\Null\, N^2$,
  $\ldots$, $\Null\, N^n$,
  and we have a matrix $U$ whose columns
  are the vectors of this basis:
  \begin{displaymath}
    U = [\mathbf{u}_1, \mathbf{u}_2, \ldots, \mathbf{u}_n].
  \end{displaymath}
  Corollary \ref{coro:matrixOfLinearMap} states that
  \begin{displaymath}
    N [\mathbf{u}_1, \mathbf{u}_2, \ldots, \mathbf{u}_n]
    = [\mathbf{u}_1, \mathbf{u}_2, \ldots, \mathbf{u}_n] M.
  \end{displaymath}
  The fact of $\mathbf{u}_1\in \Null\,N$ dictates
  that all entries in the first column of $M$ must be zero.
  Let $\mathbf{u}_j$ be the first basis vector in $\Null\,N^2$
  but not in $\Null\,N$.
  Then $N \mathbf{u}_j\in \Null\,N$, i.e., 
  $N \mathbf{u}_j$ must be a linear combination
  of $\mathbf{u}_1, \mathbf{u}_2, \ldots, \mathbf{u}_{j-1}$.
  Hence the first $j-1$ entries in the $j$th column of $M$
  could be nonzero while all other entries must be zero.
  Proceeding in this fashion completes the proof.
\end{proof}

\begin{lem}
  \label{lem:nilpotentOpPlusIdenHaveSquareRoots}
  If an operator $N\in{\cal L}(V)$ is nilpotent, 
  then $I+N$ has a square root,
  i.e., there exists an operator $M\in{\cal L}(V)$
  such that $M^2=I+N$.
\end{lem}
\begin{proof}
  By Definition \ref{def:nilpotentOp},
  $N^m=\mathbf{0}$ for some $m\in \mathbb{N}^+$. 
  For operators of the form
  \begin{displaymath}
    M = I + \sum_{i=1}^{m-1} a_i N^i,
  \end{displaymath}
  $M^2=I+N$ yields
%  \begin{displaymath}
  $\left(I + \sum_{i=1}^{m-1} a_i N^i\right)^2 = I +N$,
%  \end{displaymath}
  which is equivalent to
  \begin{displaymath}
    2a_1 = 1,\ 2a_2+a_1^2=0, \ 2a_3+2a_1a_2=0,\ \ldots,
  \end{displaymath}
  where $a_1=\frac{1}{2}$ and each $a_j$
  can be uniquely determined from $a_1, a_2, \ldots, a_{j-1}$.
\end{proof}


\subsection{Operator decomposition}
\label{sec:oper-decomp}

\begin{lem}
  \label{lem:NullRangeInvariantUnderPolyT}
  Suppose $p(T)$ is a polynomial of an operator $T\in {\cal L}(V)$.
  Then both $\Null\,p(T)$ and $\Range\,p(T)$
  are invariant under $T$.
\end{lem}
\begin{proof}
  $\mathbf{v}\in \Null\,p(T)$ implies $p(T)\mathbf{v}=\mathbf{0}$
  and 
  \begin{displaymath}
    p(T) (T\mathbf{v}) = T (p(T)\mathbf{v}) = T \mathbf{0}= \mathbf{0}.
  \end{displaymath}
  Hence $T\mathbf{v}\in \Null\, p(T)$,
  and, by Definition \ref{def:invariantSubspace},
  $\Null\,p(T)$ is invariant under $T$.
  Similarly,
  \begin{displaymath}
    \mathbf{v}\in \Range\,p(T)\ \Rightarrow\
    \exists \mathbf{u}\in V \text{ s.t. }
    p(T)\mathbf{u}=\mathbf{v}
  \end{displaymath}
  and thus
  \begin{displaymath}
    T\mathbf{v} = T p(T)\mathbf{u} =
    p(T)(T\mathbf{u}).
  \end{displaymath}
  Since $T\mathbf{u}\in V$,
  we have $T\mathbf{v}\in\Range\,p(T)$
  and, by Definition \ref{def:invariantSubspace},
  $\Range\,p(T)$ is invariant under $T$.
\end{proof}

\begin{rem}
  The following theorem is a major conclusion
  that leads to many useful results.
\end{rem}

\begin{thm}[Decomposing operators on complex vector spaces]
  \label{thm:operatorDecompositionComplexVectorSpace}
  Suppose $V$ is a complex vector space,
  $T\in {\cal L}(V)$, 
  and $\lambda_1, \lambda_2, \ldots, \lambda_m$
  are all the distinct eigenvalues of $T$.
  Then
  \begin{enumerate}[(DOC-1)]\itemsep0em
  \item $V=G(\lambda_1,T)\oplus \cdots \oplus G(\lambda_m, T)$;
  \item each $G(\lambda_j, T)$ is invariant under $T$;
  \item each $(T-\lambda_jI)|_{G(\lambda_j, T)}$ is nilpotent.
  \end{enumerate}
\end{thm}
\begin{proof}
  Write $n:=\dim V$. 
  Lemma \ref{lem:generalizedEigenspace} states that
  for each $j=1, 2, \ldots, m$
  we have $G(\lambda_j, T) = \Null(T-\lambda_j I)^{n}$.
  Then (DOC-3) follows from Definition \ref{def:nilpotentOp}
  and (DOC-2) follows from Lemma \ref{lem:NullRangeInvariantUnderPolyT}
  by choosing $p(z)=(z-\lambda_j)^n$.
  
  (DOC-1) can be proven by an induction on $n$.
  The induction basis of $n=1$ clearly holds.
  As the induction hypothesis, (DOC-1) holds
  on all vector spaces of which the dimension
  is smaller than $n$.% and bigger than 1.

  By Theorem \ref{thm:existenceOfEigenvalues},
  $T$ has an eigenvalue $\lambda_1$.
  The application of Lemma \ref{thm:directSumOfNullNandRangeN}
  to $T-\lambda_1 I$ yields
  \begin{displaymath}
    (*):\quad V = G(\lambda_1, T) \oplus U,
  \end{displaymath}
  where $U=\Range(T-\lambda_1I)^n$.
  If $U$ is empty, \mbox{(DOC-1)} holds trivially; 
  otherwise Lemma \ref{lem:NullRangeInvariantUnderPolyT} implies that
  $U$ is invariant under $T$,
  furnishing the restriction operator \mbox{$T|_U\in{\cal L}(U)$}
  in (\ref{eq:restrictedLinearOp}).
  By $(*)$ and Theorem \ref{thm:existenceOfEigenvalues},
  $T|_U$ has distinct eigenvalues $\lambda_2,\ldots,\lambda_m$,
  each of which is different from $\lambda_1$.
  Since $\dim G(\lambda_1,T)\ge 1$,
  we have $\dim U<n$
  and thus we can apply the induction hypothesis to $U$
  to obtain
  \begin{displaymath}
    (**):\quad
    U = G(\lambda_2,T|_U)\oplus \cdots \oplus G(\lambda_m, T|_U).
  \end{displaymath}
  Combining $(**)$ with $(*)$
  completes the proof if we can show
  \begin{displaymath}
    \forall j=2,\ldots,m,\quad
    G(\lambda_j,T|_U) = G(\lambda_j, T).
  \end{displaymath}
  Indeed, 
  apply $\cap G(\lambda_j, T)$ to both sides of $(*)$
  and we have
  \begin{displaymath}
    \begin{array}{rl}
      G(\lambda_j, T) &= (G(\lambda_1, T)\cap G(\lambda_j, T))\oplus
                        (U \cap G(\lambda_j, T))
                        \\ &= G(\lambda_j,T|_U), 
    \end{array}
  \end{displaymath}
  where the second step follows from Lemma
  \ref{lem:generalizedEigenvecsLinearIndep} and the identity
  $G(\lambda_j,T|_U) = G(\lambda_j, T) \cap U$.
  %%%% Below is another way to show $G(\lambda_j,T|_U) = G(\lambda_j, T)$.
  % The inclusion $G(\lambda_j,T|_U) \subset G(\lambda_j, T)$
  % clearly holds because $U$ is an invariant subspace of $V$ under $T$.
  % To show $G(\lambda_j,T|_U) \supset G(\lambda_j, T)$,
  % consider $\mathbf{v}\in G(\lambda_j, T)$.
  % By $(*)$, we have a unique decomposition
  % $\mathbf{v}= \mathbf{v}_1 + \mathbf{u}$
  % with $\mathbf{v}_1\in G(\lambda_1,T)$
  % and $\mathbf{u}\in U$.
  % Then $(**)$ yields the unique decomposition
  % \begin{displaymath}
  %   \mathbf{v} = \mathbf{v}_1 + \mathbf{v}_2 + \cdots + \mathbf{v}_m,
  % \end{displaymath}
  % where each $\mathbf{v}_j\in G(\lambda_j, T|_U)$.
  % Furthermore, Lemma \ref{lem:generalizedEigenvecsLinearIndep}
  % states that these $\mathbf{v}_j$'s are linearly independent.
  % Therefore, $\mathbf{v}\in G(\lambda_j, T)$ implies
  % $\mathbf{v}=\mathbf{v}_j$. Thus we have $\mathbf{v}\in U$
  % and $\mathbf{v}\in G(\lambda_j,T|_U)$.
\end{proof}

\begin{exm}
  For the operator $T\in {\cal L}(\mathbb{C}^3)$ in
  (\ref{eq:generalizedEigenvectorExmT}),
  $T$ does not have enough eigenvectors to span
  $\mathbb{C}^3$.
  Example \ref{exm:multiplicities} shows that
  \begin{displaymath}
    \begin{array}{l}
      E(0,T)\oplus E(5,T) \subset \mathbb{C}^3,\\
      G(0,T)\oplus G(5,T) = \mathbb{C}^3.
    \end{array}
  \end{displaymath}
\end{exm}

\begin{coro}
  \label{coro:basisAsGeneralizedEigenvectors}
  Suppose $V$ is a complex vector space and
  $T\in {\cal L}(V)$.
  Then there is a basis of $V$
  that consists of generalized eigenvectors of $T$.
\end{coro}
\begin{proof}
  This follows from (DOC-1)
  in Theorem \ref{thm:operatorDecompositionComplexVectorSpace}.
\end{proof}

\begin{defn}
  A \emph{block diagonal matrix} $A$ is a square matrix of the form
  \begin{equation}
    \label{eq:blockDiagonal}
    \begin{pmatrix}
      A_1& & \mathbf{0}
      \\
      & \ddots &
      \\
      \mathbf{0} & & A_m
    \end{pmatrix},
  \end{equation}
  where $A_1, \ldots, A_m$ are square matrices along the diagonal
  and all other entries of $A$ is 0.
\end{defn}

\begin{rem}
  By Theorem \ref{thm:linearOpHasTriangularMatrix},
   a linear operator
   has an upper triangular matrix,
   which is the sum of a diagonal matrix
   and a strictly upper triangular matrix.
  As a consequence of Theorem
  \ref{thm:operatorDecompositionComplexVectorSpace}, 
  the upper triangular form
  can be further simplified
  into a block diagonal form.
\end{rem}

\begin{thm}
  \label{thm:blockMatrixOpDecomposition}
  Suppose $V$ is a complex vector space,
  \mbox{$T\in {\cal L}(V)$}, 
  and $\lambda_1, \lambda_2, \ldots, \lambda_m$
  are all the distinct eigenvalues of $T$
  with multiplicities $d_1, d_2, \ldots, d_m$.
  Then there is a basis of $V$
  with respect to which $T$ has a block diagonal form
  (\ref{eq:blockDiagonal}) where each $A_j$
  is a $d_j$-by-$d_j$ upper triangular matrix of the form
  \begin{equation}
    \label{eq:blockMatrixForm}
    \begin{bmatrix}
      \lambda_j & & *
      \\
      & \ddots &
      \\
      0 & & \lambda_j
    \end{bmatrix}.
  \end{equation}
\end{thm}
\begin{proof}
  By Theorem
  \ref{thm:operatorDecompositionComplexVectorSpace},
  each \mbox{$(T-\lambda_jI)|_{G(\lambda_j, T)}$} is nilpotent.
  By Lemma \ref{lem:matrixOfNilpotentOperators},
  we can choose a basis of $G(\lambda_j, T)$
  such that the matrix of $(T-\lambda_jI)|_{G(\lambda_j, T)}$
  is strictly upper triangular.
  Then the form of (\ref{eq:blockMatrixForm})
  follows from
  \begin{displaymath}
    T|_{G(\lambda_j, T)}
    = (T-\lambda_jI)|_{G(\lambda_j, T)}
    + \lambda_jI|_{G(\lambda_j, T)}.
  \end{displaymath}
  The rest follows from (DOC-1) in Theorem
  \ref{thm:operatorDecompositionComplexVectorSpace}.
\end{proof}

\begin{exm}
  For $T\in{\cal L}(\mathbb{F}^3)$ given by
  \begin{equation}
    \label{eq:blockMatrixFormExam}
    T(x,y,z) =(6x+3y+4z, 6y+2z, 7z),
  \end{equation}
  the matrix of $T$ with respect to the standard basis is
  \begin{displaymath}
    T_S = \begin{pmatrix}
      6 & 3 & 4
      \\
      0 & 6 & 2
      \\
      0 & 0 & 7
    \end{pmatrix}.
  \end{displaymath}
  It is readily verified that
  the eigenvalues of $T$ are 6 and 7, with
  the corresponding generalized eigenspaces as
  \begin{displaymath}
    \begin{array}{l}
      G(6,T) = \Span\{(1,0,0),(0,1,0)\};
      \\
      G(7,T) = \Span\{(10,2,1)\}.
    \end{array}
  \end{displaymath}
  The matrix of $T$ with respect to the basis
  \begin{displaymath}
    \{\mathbf{v}_1=(1,0,0), 
    \mathbf{v}_2=(0,1,0), 
    \mathbf{v}_3=(10,2,1)\}
  \end{displaymath}
  is of the block diagonal form:
  \begin{displaymath}
    T_B = 
    \left(
      \begin{array}{cc}
        \left[
        \begin{array}{cc}
          6 & 3
          \\
          0 & 6
        \end{array}
              \right]
            &
              \begin{array}{c}
                0 \\ 0
              \end{array}
        \\
        \begin{array}{cc}
          0 & 0
        \end{array}
            &
              [7]
      \end{array}
    \right).
  \end{displaymath}
  More precisely,
  by Corollary \ref{coro:matrixOfLinearMap},
  we have
  \begin{displaymath}
    T_S[\mathbf{v}_1, \mathbf{v}_2, \mathbf{v}_3] = [\mathbf{v}_1, \mathbf{v}_2, \mathbf{v}_3]T_B.
  \end{displaymath}
\end{exm}

\begin{rem}
  From the viewpoint of operator decomposition, 
  Theorem \ref{thm:operatorDecompositionComplexVectorSpace}
  states that every operator on a complex vector space
  are composed of pieces,
  each of which,
  by Theorem \ref{thm:blockMatrixOpDecomposition}, 
  is a nilpotent operator plus
  a scalar multiple of the identity.
  If we assemble the diagonal pieces
  and the nilpotent pieces back into
  a diagonal matrix and a nilpotent matrix, respectively,
  we get Corollary \ref{coro:opDecomposeDiagNilp}.  
\end{rem}

\begin{coro}
  \label{coro:opDecomposeDiagNilp}
  Any operator $T\in {\cal L}(V)$
  on a complex vector space $V$
  can be decomposed as $T=\Lambda+N$
  where $\Lambda$ is diagonalizable,
  $N$ is nilpotent, and $\Lambda N=N\Lambda$.  
\end{coro}
\begin{proof}
  By Theorem \ref{thm:blockMatrixOpDecomposition},
  $T$ has a block diagonal form
  where the diagonal matrices are
  $T_1, T_2, \cdots, T_p$
  and each $T_j$ can be decomposed as
  $T_j=N_j + \Lambda_j$
  with $N_j=T_j-\lambda_jI_j$
  and $\Lambda_j=\lambda_jI_j$.
  Clearly $N_j$ is nilpotent and $\Lambda_j$ is diagonalizable.
  Also, any matrix commutes with the identity matrix,
  hence we have $N_j \lambda_jI_j = \lambda_jI_j N_j$.
  The rest follows from the block diagonal form of $T$.
\end{proof}

\begin{exm}
  \label{exm:operatorDecompositionExp}
  If $p$ is a polynomial of degree $k$,
  then $p(a+x)$ can be expressed as a Taylor series, 
  \begin{displaymath}
    p(a+x) = \sum_{i=0}^k \frac{p^{(i)}(a)}{i!}x^i.
  \end{displaymath}
  The formula is an algebraic identity
  and can be generalized to an operator
  $T=\Lambda+N$ with the help of $\Lambda N=N\Lambda$,
  \begin{displaymath}
    p(\Lambda+N) = \sum_{i=0}^k \frac{p^{(i)}(\Lambda)}{i!}N^i
    = \sum_{i=0}^{m} \frac{p^{(i)}(\Lambda)}{i!}N^i,
  \end{displaymath}
  where the nilpotent operator $N$ satisfies $N^{m+1}=\mathbf{0}$.

  The same approach works if $p$ is not a polynomial, 
  but an infinite power series with its radius of convergence
  as $\infty$.
  In particular, if $p(x)=e^x$,
  we have
  \begin{equation}
    \label{eq:matrixExponentialFromDecomp}
    e^T = \sum_{i=0}^{+\infty} \frac{e^{\Lambda}}{i!}N^i
    = e^{\Lambda}\sum_{i=0}^m \frac{1}{i!}N^i,
  \end{equation}
  which can be adopted as a definition of matrix exponentials.
\end{exm}

\begin{thm}
  \label{thm:invertibleOpComplexVecSpaceHasRoot}
  Any invertible operator $T\in {\cal L}(V)$
  on a complex vector space $V$
  has a square root.
\end{thm}
\begin{proof}
  Let $\lambda_1, \lambda_2, \ldots, \lambda_m$
  be all distinct eigenvalues of $T$.
  By (DOC-3) in Theorem
  \ref{thm:operatorDecompositionComplexVectorSpace},
  for each $j=1,2,\ldots,m$ there exists a niplotent operator
  $N_j\in {\cal L}(G(\lambda_j,T))$ 
  such that
  $T|_{G(\lambda_j, T)}=\lambda_jI+N_j$.
  Since $T$ is invertible, $\lambda_j\ne 0$
  and
  \begin{displaymath}
    \forall j=1,2, \ldots, m,\quad
    T|_{G(\lambda_j, T)}=\lambda_j\left(I+\frac{N_j}{\lambda_j}\right).
  \end{displaymath}
  By Lemma \ref{lem:nilpotentOpPlusIdenHaveSquareRoots}
  and the condition of $\mathbb{F}=\mathbb{C}$,
  there exists an operator $R_j\in {\cal L}(G(\lambda_j,T))$
  such that $R_j^2=T|_{G(\lambda_j, T)}$.

  By (DOC-1) in Theorem
  \ref{thm:operatorDecompositionComplexVectorSpace},
  any vector $\mathbf{v}\in V$ can be uniquely expressed
  in the form $\mathbf{v}=\sum_{i=1}^m \mathbf{u}_i$
  where $\mathbf{u}_i\in G(\lambda_i,T)$.
  Then it is straightforward to verify that
  the operator $R\in{\cal L}(V)$ given below
  satisfies $R^2=T$:
  \begin{displaymath}
    R (\mathbf{v}) = \sum_{j=1}^m R_j \mathbf{u}_j. \qedhere
  \end{displaymath}
\end{proof}

\begin{rem}
  In fact, the techniques adopted in
  proving Lemma \ref{lem:nilpotentOpPlusIdenHaveSquareRoots}
  and Theorem \ref{thm:invertibleOpComplexVecSpaceHasRoot}
  are general enough for showing that
  an invertible operator $T$ on a complex vector space
  has a $k$th root for every $k\in \mathbb{N}^+$.
\end{rem}


% \subsection{Characteristic and minimal polynomials}
% \label{sec:char-minim-polyn}

% \begin{thm}[Factorization of a polynomial over $\mathbb{C}$]
%   If $p\in \mathcal{P}(\mathbb{C})$ is a nonconstant polynomial,
%   then $p$ has a unique factorization (except for the order of the factors)
%   of the form
%   \begin{equation}
%     \label{eq:polyFactorization}
%     p(z) = c(z-\lambda_1)\cdots(z-\lambda_m),
%   \end{equation}
%   where $c,\lambda_1,\ldots,\lambda_m\in \mathbb{C}$.
% \end{thm}


\subsection{Jordan basis}
\label{sec:revi-jord-canon}

\begin{rem}
  We have been trying to simplify the matrix $M_T$ of a linear
  operator $T$
  as much as possible, i.e., 
  to zero out as many elements of $M_T$ as possible.
  The first milestone was the Schur form
  in Theorem \ref{thm:linearOpHasTriangularMatrix}.
  The second one was the block diagonal form
  in Theorem \ref{thm:blockMatrixOpDecomposition}.
  Finallly, the Jordan canoncial form in this subsection
  is the simplest in the most general context.
\end{rem}
% \begin{lem}
%   A nondefective matrix $A$ is diagonalizable,
%   i.e.,
%   \begin{equation}
%     \label{eq:nondefectDiagnolizable}
%     \exists \text{ nonsingular } R \text{ s.t. }
%     R^{-1}AR = \Lambda \text{ is diagonal.}
%   \end{equation}
% \end{lem}

\begin{defn}
  A \emph{Jordan block} of order $k$ has the form
  \begin{equation}
    \label{eq:JordanBlock}
    J(\lambda, k) = \lambda I_k+S_k,
  \end{equation}
  where
  \begin{equation*}
    (S_k)_{i,j}= \begin{cases}
      1, \quad i=j-1, \\
      0, \quad \text{otherwise.}
    \end{cases}
  \end{equation*}
\end{defn}

\begin{exm}
  The Jordan blocks of orders 1, 2, and 3 are
  \begin{equation*}
    J(\lambda, 1) = \lambda, \quad
    J(\lambda,2) = \begin{bmatrix}
      \lambda & 1 \\
      0 & \lambda
    \end{bmatrix}, \quad
    J(\lambda, 3) = \begin{bmatrix}
      \lambda & 1 & 0 \\
      0 & \lambda & 1 \\
      0 & 0 & \lambda
    \end{bmatrix}.
  \end{equation*}
\end{exm}

\begin{exm}
  \label{exm:nilpotentOp3}
  The nilpotent operator % $N\in {\cal L}(\mathbb{F}^3)$,
%  given by
  \begin{displaymath}
    N(x,y,z) = (0, x, y)
  \end{displaymath}
  can be exploited to construct a basis of $\mathbb{F}^3$:
  $(N^2 \mathbf{v}, N\mathbf{v}, \mathbf{v})$
  where $\mathbf{v}=(1,0,0)$.
  With respect to this basis,
  the matrix of $N$ is the Jordan block $J(0,3)$.
\end{exm}

\begin{exm}
  \label{exm:nilpotentOp6}
  The nilpotent operator %$N\in {\cal L}(\mathbb{F}^6)$,
  \begin{displaymath}
    N(z_1,z_2,z_3,z_4,z_5,z_6) = (0, z_1, z_2, 0, z_4, 0)
  \end{displaymath}
  can be exploited to construct a basis of $\mathbb{F}^6$:
  \begin{displaymath}
    (N^2 \mathbf{v}_1, N\mathbf{v}_1, \mathbf{v}_1, N\mathbf{v}_2,
    \mathbf{v_2}, \mathbf{v}_3)
  \end{displaymath}
  where
  \begin{displaymath}
    \begin{array}{l}
    \mathbf{v}_1=(1,0,0,0,0,0),\\
      \mathbf{v}_2=(0,0,0,1,0,0),\\
      \mathbf{v}_3=(0,0,0,0,0,1).
    \end{array}
  \end{displaymath}
  With respect to this basis,
  the matrix of $N$ is the block diagonal matrix
  \begin{displaymath}
    \begin{pmatrix}
      J(0,3) & & \\
      & J(0,2) & \\
      & & J(0,1) 
    \end{pmatrix}.
  \end{displaymath}
\end{exm}

\begin{rem}
  The next lemma shows that every nilpotent operator
  behaves similarly to those
  in Examples \ref{exm:nilpotentOp3} and \ref{exm:nilpotentOp6}.
\end{rem}

\begin{lem}
  \label{lem:nilpotentOpLeadsToBasis}
  For a nilpotent operator $N\in{\cal L}(V)$,
  there exist
  $\mathbf{v}_1, \mathbf{v}_2, \ldots, \mathbf{v}_n\in V$
  and $m_1, m_2, \ldots, m_n\in \mathbb{N}$
  such that
  \begin{enumerate}[(a)]\itemsep0em
  \item $N^{m_1}\mathbf{v}_1, \cdots, N\mathbf{v}_1, \mathbf{v}_1,
    \cdots, N^{m_n}\mathbf{v}_n, \cdots, N\mathbf{v}_n, \mathbf{v}_n$
    form a basis of $V$;
  \item $N^{m_1+1}\mathbf{v}_1 = \cdots = N^{m_n+1}\mathbf{v}_n=\mathbf{0}$.
  \end{enumerate}
\end{lem}
\begin{proof}
  % See \cite[p. 271]{axler15:_linear_algeb_done_right}.
  If $\Range\,N$ is empty, Theorem \ref{thm:dimensionsOfNullAndRange}
  implies that $\Null\, N=V$.
  Then $N=\mathbf{0}$ and both (a) and (b) hold trivially.
  Hereafter we prove this lemma by an induction on $\dim V$.
  The induction basis for $\dim V=1$ clearly holds.
  Hereafter we assume that $\dim V>1$ and (a) and (b)
  hold on all vector spaces of smaller dimensions.

  Because $N$ is nilpotent, $N$ is not injective.
  By Theorem \ref{thm:finiteDimVecSpaceInjecSurjecInvert},
  $N$ is not surjective either.
  The range of $N$, 
  a subspace of $V$ (c.f. Theorem \ref{thm:RangeIsAsubspace}),
  has a smaller dimension than $V$.
  Thus we can apply our induction hypothesis to $\Range\,N$
  to obtain that, 
  for $N|_{\Range\,N}\in{\cal L}(\Range\,N)$,
  there exist vectors 
  $\mathbf{v}_1, \mathbf{v}_2, \ldots, \mathbf{v}_n\in \Range\,N$
  and $m_1, m_2, \ldots, m_n\in \mathbb{N}$ such that
  \begin{displaymath}
    (*):\quad N^{m_1}\mathbf{v}_1, \ldots, N\mathbf{v}_1, \mathbf{v}_1,
    \cdots, N^{m_n}\mathbf{v}_n, \ldots, N\mathbf{v}_n, \mathbf{v}_n
  \end{displaymath}
  is a basis of $\Range\,N$
  and (b) holds for this basis. Then
  \begin{displaymath}
    \forall j=1,\ldots,n,\quad \exists \mathbf{u}_j\in V
    \text{ s.t. } \mathbf{v}_j=N \mathbf{u}_j.
  \end{displaymath}

  Next, we claim that
  the following is an independent list of vectors in $V$:
  \begin{displaymath}
    (\square):\quad N^{m_1+1}\mathbf{u}_1, \ldots, N\mathbf{u}_1, \mathbf{u}_1,
    \cdots, N^{m_n+1}\mathbf{u}_n, \ldots, N\mathbf{u}_n, \mathbf{u}_n.
  \end{displaymath}
  Indeed, suppose a linear combination of $(\square)$ equals
  $\mathbf{0}$.
  Then apply $N$ to it and we get a linear combination of $(*)$
  equal to $\mathbf{0}$ and thus all the coefficients
  in the original linear combination must be 0 except for
  those of the vectors
  \begin{displaymath}
    N^{m_1+1}\mathbf{u}_1=N^{m_1}\mathbf{v}_1,\
    \ldots,\  N^{m_n+1}\mathbf{u}_n=N^{m_n}\mathbf{v}_n.
  \end{displaymath}
  Again, the linear independence of the list $(*)$ dictates
  that all coefficients of the above vectors must be 0.

  By Lemma \ref{lem:linearIndependentListExtendableToBasis},
  we can extend $(\square)$ to a basis of $V$, 
  \begin{displaymath}
    (\triangle):\ \ N^{m_1+1}\mathbf{u}_1, \ldots, \mathbf{u}_1,
    \cdots, N^{m_n+1}\mathbf{u}_n, \ldots, \mathbf{u}_n,
    \mathbf{w}_1, \ldots, \mathbf{w}_p.
  \end{displaymath}
  Since $N \mathbf{w}_j\in \Range\,N$,
  each $N\mathbf{w}_j$ is in the span of $(*)$,
  thus there exists $\mathbf{x}_j$ in the span of $(\square)$
  such that \mbox{$N\mathbf{w}_j=N \mathbf{x}_j$}.
%  
  Define $\mathbf{u}_{n+j}=\mathbf{w}_j-\mathbf{x}_j$
  and we have $N \mathbf{u}_{n+j} = \mathbf{0}$.
  Therefore the following list of vectors
  satisfies (a) and (b):
  \begin{displaymath}
    N^{m_1+1}\mathbf{u}_1, \ldots, \mathbf{u}_1,
    \cdots, N^{m_n+1}\mathbf{u}_n, \ldots, \mathbf{u}_n,
    \mathbf{u}_{n+1}, \ldots, \mathbf{u}_{n+p}.
  \end{displaymath}
  This completes the induction and the proof.
\end{proof}

\begin{defn}
  \label{def:JordanBasis}
  A \emph{Jordan basis for a linear operator} $T\in {\cal L}(V)$
  is a basis of $V$ with respect to which
  the matrix of $T$ is a block diagonal matrix of the form
  \begin{equation}
    \label{eq:JordanBasis}
    J = 
    \begin{pmatrix}
      J(\lambda_1,k_1) & & & \\
      & J(\lambda_2,k_2) & \\
      & & \ddots & \\
      & & & J(\lambda_p,k_p) 
    \end{pmatrix},
  \end{equation}
  where %$k_j\in \mathbb{N}^+$
  the $\lambda_j$'s might not be distinct.
\end{defn}

\begin{thm}[Jordan]
  \label{thm:JordanBasis}
  Any operator $T\in{\cal L}(V)$ on a complex vector space
  has a Jordan basis.
\end{thm}
\begin{proof}
  If $T$ is a nilpotent operator $N$,
  we consider the basis given in
  Lemma \ref{lem:nilpotentOpLeadsToBasis}(a).
  For each $j$, $N$ annihilates
  $N^{m_j}\mathbf{v}_j$ in the list
  $N^{m_j}\mathbf{v}_j, \cdots, N\mathbf{v}_j, \mathbf{v}_j$
  and sends another vector to its previous.
  Hence
  $N$ has a block diagonal matrix where each matrix
  on the diagonal is the Jordan block $J(0,k_j)$.
  %with respect to the basis in Lemma \ref{lem:nilpotentOpLeadsToBasis}(a).
  Thus the statement holds for nilpotent operators.
  
  Otherwise let $\lambda_1, \lambda_2, \ldots, \lambda_m$
  be all distinct eigenvalues of $T$.
  By Theorem \ref{thm:operatorDecompositionComplexVectorSpace},
  we have the decomposition
  \begin{displaymath}
    V=G(\lambda_1,T)\oplus \cdots \oplus G(\lambda_m, T),
  \end{displaymath}
  with each $N_j:=(T-\lambda_jI)|_{G(\lambda_j, T)}$ being nilpotent.
  By the previous paragraph, each $N_j$ has a Jordan basis.
  The combination of all these Jordan bases is a Jordan basis for $T$.
\end{proof}

\begin{thm}[Jordan canonical form]
  \label{thm:JordanDecomp}
  Every matrix $A\in\mathbb{C}^{n\times n}$ has a similarity transformation
  \begin{equation}
    \label{eq:JordanCanoForm}
    A = RJR^{-1},
  \end{equation}
  where $R$ is invertible and
  $J$ is a block diagonal matrix of the form (\ref{eq:JordanBasis}), 
  % \begin{equation}
  %   \label{eq:JordanCanoFormJ}
  %   J = \begin{bmatrix}
  %     J(\lambda_1, k_1) & & & \\
  %     & J(\lambda_2, k_2) & & \\
  %     & & \ddots & \\
  %     & & & J(\lambda_s, k_s)
  %   \end{bmatrix}.
  % \end{equation}
  and each $J(\lambda_i, k_i)$ is a Jordan block of order $k_i$, 
  $\sum_{i=1}^sk_i=n$, and each $k_i$ is no greater
  than the index of $\lambda_i$.
  Let $m_a$ and $m_g$ respectively denote
  the algebraic multiplicity and
  the geometric multiplicity of an eigenvalue $\lambda$ of $A$.
  Then $\lambda$ appears in $m_g$ blocks and
  the sum of the orders of these blocks is $m_a$.
\end{thm}
\begin{proof}
  This is simply a restatement of Theorem \ref{thm:JordanBasis}
  using Definition \ref{def:eigenvalueMultiplicity}.
\end{proof}


\section{Inner product spaces}
\label{sec:inner-product-spaces}

\subsection{Inner products}
\label{sec:inner-products-norms}

\begin{defn}
  \label{def:innerProduct}
  Denote by $\mathbb{F}$ the underlying field of
  a vector space ${\cal V}$.
  The \emph{inner product} $\innerProd{\mathbf{u},\mathbf{v}}$ on ${\cal V}$
  is a function ${\cal V}\times{\cal V}\rightarrow \mathbb{F}$
  that satisfies
  \begin{enumerate}[({IP}-1)]
    \itemsep0em
  \item real positivity: 
    $\forall \mathbf{v}\in{\cal V}$,
    $\innerProd{\mathbf{v},\mathbf{v}}\ge 0$;
  \item definiteness:
    $\innerProd{\mathbf{v},\mathbf{v}}= 0$
    iff $\mathbf{v}=\mathbf{0}$;
  \item additivity in the first slot:\\
    $\forall \mathbf{u},\mathbf{v},\mathbf{w}\in{\cal V}$,
    $\innerProd{\mathbf{u}+\mathbf{v},\mathbf{w}}
    = \innerProd{\mathbf{u},\mathbf{w}}
    + \innerProd{\mathbf{v},\mathbf{w}}$;
  \item homogeneity in the first slot:\\
    $\forall a\in \mathbb{F}$,
    $\forall \mathbf{v},\mathbf{w}\in{\cal V}$,
    $\innerProd{a\mathbf{v},\mathbf{w}}
    = a\innerProd{\mathbf{v},\mathbf{w}}$;
  \item conjugate symmetry:
    $\forall \mathbf{v},\mathbf{w}\in{\cal V}$,
    $\innerProd{\mathbf{v},\mathbf{w}}
    = \overline{\innerProd{\mathbf{w},\mathbf{v}}}$.
  \end{enumerate}
  An \emph{inner product space} is a vector space ${\cal V}$
  equipped with an inner product on ${\cal V}$.
\end{defn}

\begin{coro}
  \label{coro:additivity2ndSlot}
  An inner product has additivity
  in the second slot, i.e.
  $\innerProd{\mathbf{u}, \mathbf{v}+\mathbf{w}}
  = \innerProd{\mathbf{u}, \mathbf{v}}
  + \innerProd{\mathbf{u}, \mathbf{w}}$.
\end{coro}

\begin{coro}
  \label{coro:conjugatehomogeneity}
  An inner product has conjugate homogeneity
  in the second slot,
  i.e.
  \begin{equation}
    \label{eq:conjugatehomogeneity}
    \forall a\in \mathbb{F},\ 
    \forall \mathbf{v},\mathbf{w}\in{\cal V},\qquad 
    \innerProd{\mathbf{v},a\mathbf{w}}
    = \bar{a}\innerProd{\mathbf{v},\mathbf{w}}.
  \end{equation}
\end{coro}

\begin{exc}
  Prove Corollaries \ref{coro:additivity2ndSlot}
  and \ref{coro:conjugatehomogeneity}
  from Definition \ref{def:innerProduct}.
\end{exc}
\begin{solution}
  Complex conjugate of a complex number $x=a+ib$
  is $\bar{x}=a-ib$.
  For any two complex numbers $x,y$, $\overline{x+y}=\bar{x}+\bar{y}$,
  $\overline{xy}=\bar{x}\bar{y}$,
  and $x\bar{x}=|x|^2$.
\end{solution}

\begin{defn}
  \label{def:EuclideanInnerProduct}
  The \emph{Euclidean inner product} on $\mathbb{F}^n$ is 
  \begin{equation}
    \label{eq:EuclideanInnerProduct}
    \innerProd{\mathbf{v},\mathbf{w}} = \sum_{i=1}^n v_i\overline{w_i}.
  \end{equation}
\end{defn}

\begin{rem}
  The dot product is a special case of
  the Euclidean inner product for $\mathbb{F}=\mathbb{R}$
  and is probably the most familiar form of an inner product.
\end{rem}

\subsection{Norms induced from inner products}
\label{sec:norms-induced-from}

\begin{defn}
  \label{def:inducedNorm}
  Let $\mathbb{F}$ be the underlying field of
  an inner product space ${\cal V}$.
  The \emph{norm induced by an inner product} on ${\cal V}$
  is a function ${\cal V}\rightarrow\mathbb{F}$:
  \begin{equation}
    \label{eq:inducedNorm}
    \|\mathbf{v}\|=\sqrt{\innerProd{\mathbf{v},\mathbf{v}}}.
  \end{equation}
\end{defn}

\begin{defn}
  \label{def:EuclideanLpNorm}
  For $p\in [1,\infty)$, the \emph{Euclidean $\ell_p$ norm}
  of a vector $\mathbf{v}\in\mathbb{F}^n$ is
  \begin{equation}
    \label{eq:EuclideanLpNorm}
    \|\mathbf{v}\|_p= \left(\sum_{i=1}^n |v_i|^p\right)^{\frac{1}{p}}
  \end{equation}
  and the \emph{Euclidean $\ell_{\infty}$ norm} is 
  \begin{equation}
    \label{eq:EuclideanLinfNorm}
    \|\mathbf{v}\|_{\infty}= \max_{i} |v_i|.
  \end{equation}
\end{defn}

\begin{thm}[Equivalence of norms]
  \label{thm:normEquivalence}
  Any two norms $\|\cdot\|_N$ and $\|\cdot\|_M$
  on a finite dimensional vector space
  ${\cal V}=\mathbb{C}^n$ satisfy
  \begin{equation}
    \label{eq:normEquivalence}
    \exists c_1,c_2\in \mathbb{R}^+,
    \textrm{ s.t. } \forall\mathbf{x}\in {\cal V}, \ 
    c_1\|\mathbf{x}\|_M \le \|\mathbf{x}\|_N \le c_2\|\mathbf{x}\|_M.
  \end{equation}
\end{thm}

\begin{defn}
  \label{def:angle}
  The angle between two vectors $\mathbf{v},\mathbf{w}$
  in an inner product space with $\mathbb{F}=\mathbb{R}$
  is the number $\theta\in [0,\pi]$, 
  \begin{equation}
    \label{eq:angleOf2Vectors}
    \theta = \arccos \frac{\innerProd{\mathbf{v}, \mathbf{w}}}
    {\|\mathbf{v}\|\|\mathbf{w}\|}.
    % {\sqrt{\innerProd{\mathbf{v}, \mathbf{v}}}
    % \sqrt{\innerProd{\mathbf{w}, \mathbf{w}}}}.
  \end{equation}
\end{defn}

\begin{rem}
  The generalization of the distance
  between two points in $\mathbb{R}^n$ leads to the metric.
  Meanwhile, two points in $\mathbb{R}^n$
  can also be considered as two vectors; 
  % which naturally yield the notion of angles.
  the generalization of the angle between two vectors
  leads to the inner product.
\end{rem}

% \begin{rem}
%   The intuitive geometry meaning of the angle between two vectors
%   is made rigorous by the dot product.
%   From elementary geometry,
%   we know that the angle $\theta$ between two vectors
%   $\mathbf{v},\mathbf{w} \in \mathbb{R}^2$
%   is
%   The angle between two vectors that drives
%   the introduction of inner products
%   is analogous to the Euclidean distance between two points
%   that drives the development of metric.
% \end{rem}

\begin{rem}
  By (\ref{eq:angleOf2Vectors}),
  the geometric meaning of the inner product of $\mathbf{u}$
  and $\mathbf{v}$ is   
  \begin{displaymath}
    \innerProd{\mathbf{u},\mathbf{v}} = \|\mathbf{u}\| \|\mathbf{v}\|
    \cos\theta, 
  \end{displaymath}
  i.e. the product of the length of $\mathbf{u}$
  with that of the vector obtained by projecting $\mathbf{v}$
  onto the direction of $\mathbf{u}$.
  This geometric meaning underlies the properties
  (IP-1,2,3).
\end{rem}

\begin{Figure}
  \centering
  % png/lawOfCosines
  \includegraphics[width=0.4\linewidth]{pst/CosinesLaw}
  \captionof{figure}{The law of cosines.
  }
  \label{fig:lawOfCosines}
\end{Figure}

\begin{thm}[The law of cosines]
  \label{thm:lawOfCosines}
  A triangle in Figure \ref{fig:lawOfCosines} satisfies
  \begin{equation}
    \label{eq:lawOfCosines}
    c^2 = a^2 + b^2 -2ab\cos\gamma.
  \end{equation}
\end{thm}
\begin{proof}
  The dot product of $AB$ to $AB=CB-CA$ yields
  \begin{displaymath}
    c^2 = \innerProd{AB, CB} - \innerProd{AB,CA}.
  \end{displaymath}
  The dot products of $CB$ and $CA$ to $AB=CB-CA$ yield
  \begin{align*}
    \innerProd{CB, AB} &= a^2 - \innerProd{CB,CA};
    \\
    -\innerProd{CA, AB} &= -\innerProd{CA, CB} + b^2.
  \end{align*}
  The proof is completed by adding up all three equations
  and applying (\ref{eq:angleOf2Vectors}).
\end{proof}

\begin{thm}[The law of cosines: abstract version]
  \label{thm:lawOfCosinesAbstractVersion}
  Any induced norm on a real vector space
  satisfies
  \begin{equation}
    \label{eq:lawOfCosinesAbstractVersion}
    \|\mathbf{u} - \mathbf{v}\|^2
    = \|\mathbf{u}\|^2 + \|\mathbf{v}\|^2 - 2\innerProd{\mathbf{u}, \mathbf{v}}.
  \end{equation}
\end{thm}
\begin{proof}
  Definitions \ref{def:inducedNorm} and \ref{def:innerProduct}
  and $\mathbb{F}=\mathbb{R}$ yield
  \begin{align*}
    \|\mathbf{u} - \mathbf{v}\|^2
    &= \innerProd{\mathbf{u} - \mathbf{v}, \mathbf{u} - \mathbf{v}}
    \\
    &= \innerProd{\mathbf{u}, \mathbf{u}} + \innerProd{\mathbf{v}, \mathbf{v}}
      - \innerProd{\mathbf{u}, \mathbf{v}} - \innerProd{\mathbf{v}, \mathbf{u}}
    \\
    &= \|\mathbf{u}\|^2 + \|\mathbf{v}\|^2 - 2\innerProd{\mathbf{u},
      \mathbf{v}}. \qedhere
  \end{align*}
\end{proof}

\subsection{Norms and induced inner-products}
\label{sec:normed-vector-spaces}

\begin{defn}
  \label{def:norm}
  A function $\|\cdot\|: {\cal V}\rightarrow\mathbb{F}$
  is a \emph{norm} for a vector space ${\cal V}$ iff it satisfies
  \begin{enumerate}[(NRM-1)]
    \itemsep0em
  \item real positivity: 
    $\forall \mathbf{v}\in{\cal V}$, 
    $\|\mathbf{v}\| \ge 0$;
  \item point separation:
    $\|\mathbf{v}\|=0$ $\Rightarrow$ $\mathbf{v}=\mathbf{0}$.
  \item absolute homogeneity: \\
    $\forall a\in \mathbb{F}$,
    $\forall \mathbf{v}\in{\cal V}$,\ 
    $\|a\mathbf{v}\|=|a|\|\mathbf{v}\|$;
  \item triangle inequality:\\
    $\forall \mathbf{u},\mathbf{v}\in{\cal V}$,\ 
    $\|\mathbf{u}+\mathbf{v}\|\le\|\mathbf{u}\|+\|\mathbf{v}\|$.
  \end{enumerate}
  The function $\|\cdot\|: {\cal V}\rightarrow\mathbb{F}$
  is called a \emph{semi-norm} iff
  it satisfies (NRM-1,3,4).
\end{defn}

\begin{defn}
  \label{def:normedSpace}
  A \emph{normed vector space} (or simply a \emph{normed space})
  is a vector space ${\cal V}$
  equipped with a norm on ${\cal V}$.
\end{defn}

\begin{exm}
  \label{exm:RdEuclideanSpace}
  $\left(\mathbb{R}^n, \|\cdot\|_p\right)$
  is a normed space, 
  where $\|\cdot\|_p$ is the Euclidean norm
  in Definition \ref{def:EuclideanLpNorm}
  with $p\in[1,\infty)$: 
  \begin{displaymath}
    \|\mathbf{x}\|_p= \left(\sum_{j=1}^n |x_j|^p\right)^{\frac{1}{p}}.
  \end{displaymath}
\end{exm}

\begin{rem}
  Example \ref{exm:notEuclidNormIfLess1}
  shows that $\|\cdot\|_p$ is not a norm for $p\in(0,1)$.
  In addition, the reason for the name of $\infty$-norm
  will be explained in Remark \ref{rem:reasonForInftyNorm}.
\end{rem}

\begin{exc}
  Prove the backward triangle inequality of a norm $\|\cdot\|$, i.e., 
  \begin{equation}
    \label{eq:backwardTriangle}
    \forall u,v \in V,\quad
    \left|\|u\|-\|v\|\right| \le \|u-v\|.
  \end{equation}
\end{exc}
\begin{solution}
  Without loss of generality, we may assume that $\|u\|\geq\|v\|$, and then we just need to verify that
  \begin{equation*}
    \|u\|\leq\|v\|+\|u-v\|,
  \end{equation*}
  which obviously holds because of the triangular inequality.
\end{solution}

\begin{exc}
  Explain how (NRM-1,2,3,4) relate to the geometric meaning
  of the norm of vectors in $\mathbb{R}^3$.
\end{exc}

\begin{lem}
  \label{lem:inducedNormIsNorm}
  The norm induced by an inner product
  is a norm as in Definition \ref{def:norm}.
\end{lem}
\begin{proof}
  The induced norm as in (\ref{eq:inducedNorm})
  satisfies (NRM-1,2) trivially. For (NRM-3),
  \begin{equation*}
    \|a\mathbf{v}\|^2 = \innerProd{a\mathbf{v}, a\mathbf{v}}
    = a\innerProd{\mathbf{v}, a\mathbf{v}}
    = a\bar{a}\innerProd{\mathbf{v},\mathbf{v}}
    = |a|^2 \|\mathbf{v}\|^2.
  \end{equation*}
  To prove (NRM-4), we have
  \begin{align*}
    \|\mathbf{u}+\mathbf{v}\|^2 &=
                                  \innerProd{\mathbf{u}+\mathbf{v}, \mathbf{u}+\mathbf{v}}
    \\ &=
         \innerProd{\mathbf{u}, \mathbf{u}}+ \innerProd{\mathbf{v},\mathbf{v}}
         + \innerProd{\mathbf{u}, \mathbf{v}}+ \overline{\innerProd{\mathbf{u},\mathbf{v}}}
    \\ &\le
         \innerProd{\mathbf{u}, \mathbf{u}}+ \innerProd{\mathbf{v},\mathbf{v}}
         + 2|\innerProd{\mathbf{u}, \mathbf{v}}|
    \\ &\le
         \|\mathbf{u}\|^2+\|\mathbf{v}\|^2+2\|\mathbf{u}\|\|\mathbf{v}\|
    \\ &=
         \left(\|\mathbf{u}\|+\|\mathbf{v}\|\right)^2,
  \end{align*}
  where the second step follows from (IP-5)
  and the fourth step from
  Cauchy-Schwarz inequality.
         %          T\ref{thm:SchwarzInequality} below.
\end{proof}
% \begin{thm}[Parallelogram equality]
%   $\forall \mathbf{u}, \mathbf{v}\in {\cal V}$, 
%   $\|\mathbf{u}+\mathbf{v}\|^2 + \|\mathbf{u}-\mathbf{v}\|^2
%   =2(\|\mathbf{u}\|^2+\|\mathbf{v}\|^2)$.
% \end{thm}

\begin{Figure}
  \centering
  % png/parallelogram
  \includegraphics[width=0.5\linewidth]{pst/ParallelogramLaw}
  \captionof{figure}{The law of parallelograms. 
  }
  \label{fig:lawOfParallelograms}
\end{Figure}

\begin{thm}[The parallelogram law]
  \label{thm:parallelogramLaw}
  The sum of squares of the lengths of the four sides
  of a parallelogram equals
  the sum of squares of the two diagonals.
  More precisely,
  the parallelogram in Figure \ref{fig:lawOfParallelograms}
  satisfies 
  \begin{equation}
    \label{eq:parallelogramLaw}
    (AB)^2 + (BC)^2 + (CD)^2 + (DA)^2 = (AC)^2 + (BD)^2.
  \end{equation}
\end{thm}
\begin{proof}
  Apply the law of cosines to the two diagonals,
  add the two equations, and we obtain (\ref{eq:parallelogramLaw}).
\end{proof}

\begin{rem}
  The purpose of our stating
  Theorems \ref{thm:lawOfCosines} and \ref{thm:parallelogramLaw}
  is to relate the abstract theory to geometric intuitions;
  they will not be used in any proofs.
\end{rem}

\begin{thm}[The parallelogram law: abstract version]
  \label{thm:inducedNormSatisfiesParellelogramLaw}
  Any induced norm (\ref{eq:inducedNorm}) satisfies
  \begin{equation}
    \label{eq:inducedNormSatisfiesParellelogramLaw}
    2\|\mathbf{u}\|^2 + 2\|\mathbf{v}\|^2
    = \|\mathbf{u}+\mathbf{v}\|^2 + \|\mathbf{u}-\mathbf{v}\|^2.
  \end{equation}
\end{thm}
\begin{proof}
  Replace $\mathbf{v}$ in (\ref{eq:lawOfCosinesAbstractVersion})
  with $-\mathbf{v}$ and we have
  \begin{align*}
    \|\mathbf{u} + \mathbf{v}\|^2
    = \|\mathbf{u}\|^2 + \|\mathbf{v}\|^2
    + 2\innerProd{\mathbf{u}, \mathbf{v}}.
  \end{align*}
  (\ref{eq:inducedNormSatisfiesParellelogramLaw})
  follows from adding the above equation
  to (\ref{eq:lawOfCosinesAbstractVersion}).
\end{proof}

\begin{exc}
  \label{exc:EuclideanPnormParallelogramLaw}
  In the case of Euclidean $\ell_p$ norms,
  show that the parallelogram law
  (\ref{eq:inducedNormSatisfiesParellelogramLaw})
  holds if and only if $p=2$.  
\end{exc}

\begin{rem}
  The following theorem states
  that the parallelogram characterizes the inner product spaces
  among the normed vector spaces.
\end{rem}

\begin{thm}
  \label{thm:parallelogramCharacterization}
  The induced norm (\ref{eq:inducedNorm}) holds
  for some inner product $\innerProd{\cdot, \cdot}$  
  if and only if the parallelogram law
  (\ref{eq:inducedNormSatisfiesParellelogramLaw})
  holds for every pair of $\mathbf{u}, \mathbf{v}\in {\cal V}$.
\end{thm}

\begin{exc}
  Prove Theorem \ref{thm:parallelogramCharacterization}.
\end{exc}
%          \begin{solution}
%            The necessity follows from Theorem \ref{thm:inducedNormSatisfiesParellelogramLaw}.
%            As for the sufficiency,
%            we employ the law of cosines to define an inner product
%            as
%            \begin{displaymath}
%              \innerProd{\mathbf{u}, \mathbf{v}}:=\frac{1}{4}\left(
%                \|\mathbf{u}+\mathbf{v}\|^2
%                - \|\mathbf{u}-\mathbf{v}\|^2
%              \right)
%            \end{displaymath}
%            if the scalar field is $\mathbb{R}$
%            and by
%            \begin{align*}
%              \innerProd{\mathbf{u}, \mathbf{v}} := &\frac{1}{4}\left(
%                \|\mathbf{u}+\mathbf{v}\|^2
%                + \|\mathbf{u}-\mathbf{v}\|^2 \right)
%              \\
%              &+
%              \frac{i}{4}\left(
%                \|i\mathbf{u}-\mathbf{v}\|^2
%                - \|i\mathbf{u}+\mathbf{v}\|^2
%              \right)
%            \end{align*}
%            if the scalar field is $\mathbb{C}$.
%            It is straightforward to verify
%            that the above inner product satisfies (IP-1,2,3,4,5)
%            and (\ref{eq:inducedNorm}).
%          \end{solution}

\begin{exm}
  By Theorem \ref{thm:parallelogramCharacterization}
  and Exercise \ref{exc:EuclideanPnormParallelogramLaw},
  none of the Euclidean $\ell_1$ and $\ell_{\infty}$ norms
  has a corresponding inner product.
\end{exm}


\subsection{Orthonormal bases}
\label{sec:orthonormal-bases}

\begin{defn}
  \label{def:orthogonalVectors}
  Two vectors $\mathbf{u}, \mathbf{v}$ are called \emph{orthogonal}
  if $\langle \mathbf{u}, \mathbf{v}\rangle=0$,
  i.e., their inner product is the additive identity of the underlying
  field.
\end{defn}

\begin{rem}
  From a geometric viewpoint,
  two vectors $\mathbf{u}, \mathbf{v}$ are orthogonal iff
  they are perpendicular to each other,
  i.e. the angle between them is $\frac{\pi}{2}$. 
  A more general and useful notion of orthogonality
  is a major motivation of inner product spaces.
  As a fruit of this abstraction,
  Definition \ref{def:orthogonalVectors} applies to other problems
  that may appear unrelated to two planar vectors being perpendicular.
\end{rem}

\begin{exm}
  An inner product on the vector space of 
  continuous real-valued functions on the interval $[-1,1]$ is 
  \begin{equation*}
    \langle f, g \rangle = \int_{-1}^{+1} f(x) g(x)\dif x.
  \end{equation*}
  $f$ and $g$ are said to be orthogonal
  if the integral is zero.
\end{exm}

\begin{thm}[Pythagorean]
  \label{thm:Pythagorean}
  If $\mathbf{u}, \mathbf{v}$ are orthogonal,
  then $\|\mathbf{u}+\mathbf{v}\|^2=\|\mathbf{u}\|^2+\|\mathbf{v}\|^2$.
\end{thm}
\begin{proof}
  This follows from (\ref{eq:lawOfCosinesAbstractVersion})
  % Theorem \ref{thm:lawOfCosinesAbstractVersion}
  and Definition \ref{def:orthogonalVectors}.
\end{proof}

\begin{thm}[Cauchy-Schwarz inequality]
  \label{thm:SchwarzInequality}
  \begin{equation}
    \label{eq:CauchySchwarz}
    |\langle \mathbf{u},\mathbf{v}\rangle|
    \le \|\mathbf{u}\| \|\mathbf{v}\|,
  \end{equation}
  where the equality holds iff one of $\mathbf{u}, \mathbf{v}$
  is a scalar multiple of the other.
\end{thm}
\begin{proof}
  For any complex number $\lambda$, (IP-1) implies
  \begin{align*}
    % \forall \lambda\in \mathbb{C},
    & \innerProd{\mathbf{u}+\lambda \mathbf{v}, \mathbf{u}+\lambda \mathbf{v}}\ge 0
    \\ \Rightarrow
    & 
      \innerProd{\mathbf{u}, \mathbf{u}} + \lambda\innerProd{ \mathbf{v}, \mathbf{u}}
      + \bar{\lambda}\innerProd{\mathbf{u}, \mathbf{v}}+\lambda\bar{\lambda}\innerProd{\mathbf{v},\mathbf{v}}\ge 0.
  \end{align*}
  If $\mathbf{v}=\mathbf{0}$, (\ref{eq:CauchySchwarz}) clearly holds.
  Otherwise (\ref{eq:CauchySchwarz}) follows from substituting
  \mbox{$\lambda=-\frac{\innerProd{\mathbf{u},\mathbf{v}}}{\innerProd{\mathbf{v},\mathbf{v}}}$}
  into the above equation.
\end{proof}

\begin{exc}
  To explain the choice of $\lambda$ in the proof
  of Theorem \ref{thm:SchwarzInequality}, 
  what is the geometric meaning of (\ref{eq:CauchySchwarz})
  in the plane? When will the equality hold?
\end{exc}
\begin{solution}
  The absolute value of inner product
  is never greater than the length product of the two vectors.
  The equality holds when the two vectors are collinear.
\end{solution}

\begin{exm}
  \label{exm:CS1}
  If $x_i,y_i\in \mathbb{R}$, then for any $n\in \mathbb{N}^+$
  \begin{equation*}
    \left|\sum_{i=1}^n x_iy_i\right|^2
    \le \sum_{j=1}^nx_j^2 \sum_{k=1}^ny_k^2.
  \end{equation*}
\end{exm}

\begin{exm}
  \label{exm:CS2}
  If $f,g: [a,b]\rightarrow \mathbb{R}$ are continuous, then
  \begin{equation*}
    \left|\int_{a}^{b} f(x) g(x) \dif x\right|^2
    \le      
    \left(\int_{a}^{b} f^2(x) \dif x\right)
    \left(\int_{a}^{b} g^2(x) \dif x\right)
  \end{equation*}
\end{exm}

% \begin{rem}
%   In Examples \ref{exm:CS1} and \ref{exm:CS2},
%   ``$|\cdot|$'' denotes the modulus of a complex number,
%   and thus cannot be replaced by ``$(\cdot)$.''
% \end{rem}

% \begin{defn}
%   \label{def:openBall}
%   The \emph{open ball} centered at $\mathbf{x}\in \mathbb{R}^n$
%   with radius $r\in \mathbb{R}^+$ is
%   \begin{equation}
%     \label{eq:openBall}
%     {\cal B}_r(\mathbf{x}) = \{
%     \mathbf{x}\in \mathbb{R}^n : \|\mathbf{x}\|_2<r
%     \}.
%   \end{equation}
%   Replacing ``$<$'' with ``$\le$'' 
%   yields a \emph{closed ball} $\overline{\cal B}_r(\mathbf{x})$.
% \end{defn}


\begin{defn}
  \label{def:orthonormalList}
  A list of vectors
  $(\mathbf{e}_1, \mathbf{e}_2, \ldots, \mathbf{e}_m)$
  is called \emph{orthonormal}
  if the vectors in it are pairwise orthogonal
  and each vector has norm 1, i.e.
  \begin{equation}
    \label{eq:orthonormalList}
    \renewcommand{\arraystretch}{1.2}
    \left\{
      \begin{array}{l}
        \forall i=1,2,\ldots,m, \qquad
        \|\mathbf{e}_i\| = 1;
        \\
        \forall i\ne j, \qquad \innerProd{\mathbf{e}_i, \mathbf{e}_j}=0.
      \end{array}
    \right.
  \end{equation}
\end{defn}

\begin{defn}
  \label{def:orthonormalBasis}
  An \emph{orthonormal basis}
  of an inner-product space ${\cal V}$
  is an orthonormal list of vectors in ${\cal V}$
  that is also a basis of ${\cal V}$.
\end{defn}

\begin{rem}
  The motivation of a basis
  is to express any vector as a linear combination
  of the basis vectors.
  There is a certain amount of wiggle room
  for a basis.
  We can further trade the wiggle room
  for more convenience
  in orthonormal bases,
  as illustrated in Theorem \ref{thm:significanceOfOthonormalBases}.
\end{rem}

\begin{thm}
  \label{thm:significanceOfOthonormalBases}
  If $(\mathbf{e}_1, \mathbf{e}_2, \ldots, \mathbf{e}_n)$
  is an orthonormal basis of ${\cal V}$,
  then 
  \begin{subequations}
    \label{eq:expressVectorsInOrthonormalBasis}
    \begin{align}
      \forall \mathbf{v}\in {\cal V},\ \ 
      \mathbf{v}
      &= \sum_{i=1}^n \innerProd{\mathbf{v}, \mathbf{e}_i} \mathbf{e}_i,
      \\
      \|\mathbf{v}\|^2
      &= \sum_{i=1}^n \left|\innerProd{\mathbf{v}, \mathbf{e}_i}\right|^2.
    \end{align}
  \end{subequations}
\end{thm}

\begin{lem}
  \label{lem:existenceOfOrthnormalBasis}
  Every finite-dimensional inner-product space has an orthonormal basis.
\end{lem}

\begin{thm}[Schur]
  \label{thm:Schur}
  Every linear operator $T\in {\cal L}({\cal V})$
  on a finite-dimensional complex vector space ${\cal V}$ 
  has an upper-triangular matrix
  with respect to some orthonormal basis of ${\cal V}$.
\end{thm}
\begin{proof}
  This follows from Theorem \ref{thm:linearOpHasTriangularMatrix}, 
  Lemma \ref{lem:existenceOfOrthnormalBasis}
  and the Gram-Schmidt process.
  %; see Section \ref{sec:orthonormalSystems}. 
\end{proof}

\begin{defn}
  \label{def:linearFunctional}
  A \emph{linear functional} on ${\cal V}$ is a linear map from ${\cal V}$ to
  $\mathbb{F}$,
  or, it is an element of ${\cal L}({\cal V},\mathbb{F})$.
\end{defn}

\begin{rem}
  The following theorem reduces
  the $O(\infty^2)$ complexity of a linear functional
  $V\mapsto \mathbb{R}$
  to $O(1)$.
\end{rem}

\begin{thm}[Riesz representation theorem]
  \label{thm:RieszRep}
  If ${\cal V}$ is a finite-dimensional vector space, 
  then
  \begin{equation}
    \label{eq:RieszRep}
    \forall \varphi\in {\cal V}',\ 
    \exists! \mathbf{u}\in {\cal V} \text{ s.t. }
    \forall \mathbf{v}\in {\cal V},\ \  \varphi(\mathbf{v}) = \innerProd{\mathbf{v},\mathbf{u}}.
  \end{equation}
%  Furthermore, $\mathbf{u}$ is unique.
\end{thm}
\begin{proof}
  Let $(\mathbf{e}_1,\mathbf{e}_2, \ldots, \mathbf{e}_n)$
  be an orthonormal basis of ${\cal V}$.
  \begin{align*}
    \varphi(\mathbf{v})
    &= \varphi\left(
      \sum_{i=1}^n\innerProd{\mathbf{v}, \mathbf{e}_i} \mathbf{e}_i\right)
      = \sum_{i=1}^n\innerProd{\mathbf{v}, \mathbf{e}_i} \varphi(\mathbf{e}_i)
    \\
    &=\sum_{i=1}^n\innerProd{\mathbf{v}, \overline{\varphi(\mathbf{e}_i)}
      \mathbf{e}_i} 
      =\innerProd{\mathbf{v},  \sum_{i=1}^n\overline{\varphi(\mathbf{e}_i)}
      \mathbf{e}_i},
  \end{align*}
  where the last two steps follow from
  Corollaries \ref{coro:additivity2ndSlot}
  and \ref{coro:conjugatehomogeneity}.

  As for the uniqueness,
  suppose that $\exists \mathbf{u}_1, \mathbf{u}_2\in {\cal V}$
  s.t. $\varphi(\mathbf{v})=\innerProd{\mathbf{v},\mathbf{u}_1}
  =\innerProd{\mathbf{v},\mathbf{u}_2}$.
  Then for each $\mathbf{v}\in {\cal V}$,
  \begin{displaymath}
    0=\innerProd{\mathbf{v},\mathbf{u}_1}-\innerProd{\mathbf{v},\mathbf{u}_2}
    = \innerProd{\mathbf{v},\mathbf{u}_1-\mathbf{u}_2}.
  \end{displaymath}
  Taking $\mathbf{v}=\mathbf{u}_1-\mathbf{u}_2$
  shows that $\mathbf{u}_1-\mathbf{u}_2=\mathbf{0}$.
\end{proof}

\begin{rem}
  In Theorem \ref{thm:RieszRep},
  $\mathbf{u}$ does not depend on $\mathbf{v}$, 
  % since the logic statement in (\ref{eq:RieszRep})
  % is existential-universal.
  but $\mathbf{u}$ does depend on $\varphi$
  as it is supposed to represent $\varphi$!
\end{rem}


\section{Operators on inner-product spaces}
\label{sec:oper-inner-prod}

\subsection{Adjoint and self-adjoint operators}

\begin{defn}
  \label{def:adjoint}
  The \emph{adjoint of a linear map}
  \mbox{$T\in {\cal L}({\cal V},{\cal W})$}
  between inner-product spaces is a function
  $T^*:{\cal W}\rightarrow{\cal V}$
  that satisfies
  \begin{equation}
    \label{eq:adjoint}
    \forall \mathbf{v}\in {\cal V},\ \forall \mathbf{w}\in {\cal W},
    \qquad \innerProd{T\mathbf{v},\mathbf{w}} = \innerProd{\mathbf{v},T^*\mathbf{w}}.
  \end{equation}
\end{defn}

\begin{rem}
  The two inner products in (\ref{eq:adjoint}) may be different.
\end{rem}

\begin{rem}
  The concept of adjoint follows naturally
  from the Riesz representation theorem (Theorem \ref{thm:RieszRep}).
  Given a fixed $\mathbf{w}\in {\cal W}$,
  we can construct from $T$ and $\mathbf{w}$ a linear functional
  $\phi: {\cal V}\rightarrow \mathbf{R}$
  given by $\phi(\mathbf{v}) = \innerProd{T\mathbf{v},\mathbf{w}}$.
  By Theorem \ref{thm:RieszRep},
  there exists a unique vector $\mathbf{u}\in{\cal V}$
  such that $\phi(\mathbf{v})=\innerProd{\mathbf{v},\mathbf{u}}$.
  The uniqueness of $\mathbf{u}$
  and the above process furnish a linear map
  $T^*:{\cal W}\rightarrow{\cal V}$
  that assigns for each $\mathbf{w}\in {\cal W}$
  a unique vector $\mathbf{u}\in {\cal V}$
  according to (\ref{eq:adjoint}).
\end{rem}

\begin{exm}
  Define a linear operator $T:\mathbb{R}^3\rightarrow \mathbb{R}^2$, 
  \begin{displaymath}
    T(x_1,x_2,x_3)=(x_2+3x_3,2x_1).
  \end{displaymath}
  Then $T^*(y_1,y_2)=(2y_2,y_1,3y_1)$ because
  \begin{align*}
    \innerProd{(x_1,x_2,x_3), T^*(y_1,y_2)}
    &= \innerProd{T(x_1,x_2,x_3), (y_1,y_2)}
    \\
    &= \innerProd{(x_2+3x_3,2x_1), (y_1,y_2)}
    \\
    &= x_2y_1+3x_3y_1+2x_1y_2
    \\
    &= \innerProd{(x_1,x_2,x_3), (2y_2,y_1,3y_1)}.
  \end{align*}
\end{exm}

% \begin{exm}
%   Fix $\mathbf{u}\in {\cal V}$ and $\mathbf{x}\in {\cal W}$.
%   Define a linear operator $T\in {\cal L}({\cal V},{\cal W})$ by
%   $T\mathbf{v}=\innerProd{\mathbf{v},\mathbf{u}}\mathbf{x}$.
%   Then %$T^*$ is found by
%   \begin{align*}
%     \innerProd{\mathbf{v},T^*\mathbf{w}} &= \innerProd{T\mathbf{v},\mathbf{w}} = \innerProd{\innerProd{\mathbf{v},\mathbf{u}}\mathbf{x},\mathbf{w}}
%     \\
%     &= \innerProd{\mathbf{v},\mathbf{u}}\innerProd{\mathbf{x},\mathbf{w}} = \innerProd{\mathbf{v},\innerProd{\mathbf{w},\mathbf{x}}\mathbf{u}}.
%   \end{align*}
%   Hence $T^*\mathbf{w}=\innerProd{\mathbf{w},\mathbf{x}}\mathbf{u}$.
% \end{exm}

\begin{lem}
  If $T\in {\cal L}({\cal V},{\cal W})$,
  then $T^*\in {\cal L}({\cal W},{\cal V})$.
\end{lem}
\begin{proof}
  Use Definition \ref{def:linearMap}.
\end{proof}

\begin{rem}
  \label{rem:adjointAndDual}
  For a linear map $T\in {\cal L}({\cal V},{\cal W})$,
  its adjoint $T^*\in {\cal L}({\cal W},{\cal V})$
  is closely related to
  its dual $T'\in {\cal L}({\cal W}',{\cal V}')$.
  By Lemma \ref{lem:dimV=dimV'}
   and the Riez representation theorem, 
   we can define a natural isomorphism
   $I_{{\cal V}}: {\cal V}\rightarrow {\cal V}'$
%   and $I_{{\cal W}}: {\cal W}\rightarrow {\cal W}'$
   as
   \begin{displaymath}
     (*):\quad
     I_{{\cal V}}(\mathbf{v})(\mathbf{u})
     := \innerProd{\mathbf{u},\mathbf{v}}.
   \end{displaymath}
   Then we have the commutative diagram 
   \begin{displaymath}
     \begin{tikzcd}[column sep=3em]
       {\cal W} \ar{r}{T^*} 
       \ar{d}{I_{\cal W}}
       & {\cal V}
       \ar{d}{I_{\cal V}}
       \\
       {\cal W}' \ar{r}{T'}
       & {\cal V}' 
     \end{tikzcd}
   \end{displaymath}
   which tells us that the adjoint
   can be regarded as a reformulation of the dual, 
   and vice versa, because
   \begin{align*}
     (T'\circ I_{\cal W})(\mathbf{w})(\mathbf{v})
     &= I_{\cal W}(\mathbf{w})(T\mathbf{v})
     = \innerProd{T\mathbf{v}, \mathbf{w}}
     \\
     &= \innerProd{\mathbf{v}, T^*\mathbf{w}}
     = (I_{\cal V}\circ T^*)(\mathbf{w})(\mathbf{v}).
   \end{align*}
   In category theory, 
   the isomorphism $I_{\cal V}$ given in $(*)$
   is called a \emph{natural transformation}
   that transforms one morphism into another
   while respecting the internal structure
   of the involved categories,
   which, in the above example,
   are the category of inner-product spaces
   and that of vector spaces.
\end{rem}

\begin{thm}
  \label{thm:adjointProperties}
  The adjoint of a linear map has the following properties.
  \begin{enumerate}[({ADJ}-1)]\itemsep0em
  \item additivity:\\
    $\forall S,T\in {\cal L}({\cal V},{\cal W})$,\qquad
    $(S+T)^*=S^*+T^*$;
  \item conjugate homogeneity:\\
    $\forall T\in {\cal L}({\cal V},{\cal W})$,\ 
    $\forall a\in \mathbb{F}$,\qquad
    $(aT)^*=\bar{a}T^*$;
  \item adjoint of adjoint:\\
    $\forall T\in {\cal L}({\cal V},{\cal W})$,\ 
    $(T^*)^*=T$;
  \item identity: $I^*=I$;
  \item products:
    let ${\cal U}$ be an inner-product space,\\
    $\forall T\in {\cal L}({\cal V},{\cal W})$,\ 
    $\forall S\in {\cal L}({\cal W},{\cal U})$,\ \ 
    $(ST)^*=T^*S^*$.
  \end{enumerate}
\end{thm}
\begin{proof}
  Use Definitions \ref{def:adjoint}
  and \ref{def:innerProduct}.
\end{proof}

\begin{rem}
  \label{rem:adjointGeneralizesComplexConjugate}
  The adjoint can also be considered
  as a generalization of complex conjugate.
  For the vector space ${\cal V}=\mathbb{C}$
  with the underlying field $\mathbb{F}=\mathbb{C}$,
  we have
  \begin{displaymath}
%    \label{eq:complexConjugateAsAdjoint}
    \forall z, x, y\in \mathbb{C},\qquad
    \innerProd{zx,y}=\innerProd{x,\ccg{z}y}.
  \end{displaymath}
\end{rem}

\begin{lem}
  $T\in {\cal L}(V,W)$ and $T^*$ satisfy
  \begin{enumerate}[(a)]\itemsep0em
  \item $\Null\,T^* = (\Range\,T)^{\perp}$;
  \item $\Range\,T^* = (\Null\,T)^{\perp}$;
  \item $\Null\,T = (\Range\,T^*)^{\perp}$;
  \item $\Range\,T = (\Null\,T^*)^{\perp}$.
  \end{enumerate}
\end{lem}

\begin{defn}
  \label{def:conjugateTranspose}
  The \emph{conjugate transpose}, or
  \emph{Hermitian transpose}, or
  \emph{Hermitian conjugate}, or \emph{adjoint matrix}, 
  of a matrix $A\in \mathbb{C}^{m\times n}$
  is the matrix $A^*\in \mathbb{C}^{n\times m}$ defined by
  \begin{equation}
    \label{eq:conjugateTranspose}
    (A^*)_{ij}=\overline{a_{ji}},
  \end{equation}
  where $\overline{a_{ji}}$ denotes
  the complex conjugate of the entry $a_{ji}$.%\in \mathbb{C}$.
\end{defn}

\begin{exc}
  \label{exc:adjointAndConjugateTranspose}
  Show that 
  the conjugate transpose is an adjoint operator
  in ${\cal L}({\cal V},{\cal W})$
  with ${\cal V}=\mathbb{C}^n$ and ${\cal W}=\mathbb{C}^m$.
\end{exc}
\begin{solution}
  By Definition \ref{def:conjugateTranspose}, we have
  \begin{align*}
    \forall x\in {\cal V}, \forall y\in {\cal W},\ 
    \innerProd{Ax,y} &= \sum_{i,j} A_{ij}x_j\ccg{y_i}
                 = \sum_{i,j} x_j\ccg{A^*_{ji}}\ccg{y_i}
    \\
               &= \sum_{i,j} x_j\ccg{A^*_{ji}y_i}
                 = \innerProd{x, A^* y}.
  \end{align*}
  For $A\in \mathbb{R}^{m\times n}$,
  $A^*$ is simply $A^T$, the transpose of $A$.
\end{solution}

\begin{defn}
  \label{def:unitaryMatrices}
  A matrix $U\in \mathbb{C}^{n\times n}$ is \emph{unitary}
  iff $U^*U=I$.
  A matrix $U\in \mathbb{R}^{n\times n}$ is \emph{orthogonal}
  iff $U^TU=I$.
\end{defn}

\begin{thm}
  \label{thm:unitaryMatrixHasOrthonormalColumns}
  A matrix $U\in \mathbb{C}^{n\times n}$ is unitary
  if and only if its columns form an orthonormal basis
  for $\mathbb{C}^n$.
\end{thm}
\begin{proof}
  This follows from
  considering the $(i,j)$th element of $U^*U$
  and applying $U^*U=I$
  in Definition \ref{def:unitaryMatrices}.
\end{proof}

\begin{coro}
  \label{coro:unitaryMatricesPreserveNorms}
  A unitary matrix $U$ preserves norms and inner products.
  More precisely, we have 
  \begin{displaymath}
    \forall \mathbf{v}, \mathbf{w}\in \mathbb{C}^n,\qquad
    \innerProd{U\mathbf{v}, U\mathbf{w}} = \innerProd{\mathbf{v}, \mathbf{w}}.
  \end{displaymath}
\end{coro}
\begin{proof}
  This follows from Definitions \ref{def:adjoint}
  and \ref{def:unitaryMatrices}.
\end{proof}

\begin{rem}
  By Corollary \ref{coro:unitaryMatricesPreserveNorms},
  a unitary matrix is a linear isometry;
  see Definition \ref{def:linearIsometry}.
  as in can be considered
  as ``rotations'' and ``reflections.''
\end{rem}

\begin{thm}
  \label{thm:unitaryMatrices2by2}
  Every unitary matrix $U\in \mathbb{C}^{2\times 2}$
  with $\det U=1$ is of the form
  \begin{equation}
    \label{eq:unitaryMatrices2by2}
    U =
    \begin{bmatrix}
      a & b \\ -\ccg{b} & \ccg{a}
    \end{bmatrix},
  \end{equation}
  where $|a|^2+|b|^2=1$.
\end{thm}
\begin{proof}
  Let 
  \begin{displaymath}
    U=
    \begin{bmatrix}
      a & b \\ c & d
    \end{bmatrix}.    
  \end{displaymath}
  Then Theorem \ref{thm:unitaryMatrixHasOrthonormalColumns}
  and the condition $\det U=1$ yield
  \begin{align*}
    a \ccg{b} + c\ccg{d} &= 0,
    \\
    a d - cb &= 1.
  \end{align*}
  In other words, the linear system
  \begin{equation*}
    \begin{bmatrix}
      \ccg{b} & \ccg{d} \\ d & -b
    \end{bmatrix}
    \begin{bmatrix}
      x \\ y
    \end{bmatrix}
    =
    \begin{bmatrix}
      0 \\ 1
    \end{bmatrix}
  \end{equation*}
  has solution $x=a, y=c$.
  Furthermore, Theorem \ref{thm:unitaryMatrixHasOrthonormalColumns}
  and the form of $U$ yield
  $|b|^2+|d|^2=1$.
  Hence the solution $x=a, y=c$ is unique
  and we have $a=\ccg{d}$ and $c=-\ccg{b}$,
  which completes the proof.
\end{proof}

\begin{thm}
  \label{thm:matrixRelationOfAdjoints}
  Let $T\in {\cal L}({\cal V},{\cal W})$.
  Suppose $e_1,\ldots,e_n$ is an orthonormal basis of ${\cal V}$
  and $f_1,\ldots,f_m$ is an orthonormal basis of ${\cal W}$.
  Then 
  \begin{displaymath}
    M(T^*,(f_1,\ldots,f_m), (e_1,\ldots,e_n))
  \end{displaymath}
  is the conjugate transpose of 
  \begin{displaymath}
    M(T, (e_1,\ldots,e_n) ,(f_1,\ldots,f_m)).
  \end{displaymath}
\end{thm}
\begin{proof}
  By Corollary \ref{coro:matrixOfLinearMap},
  we have
  \begin{displaymath}
    T[e_1,\ldots,e_n] = [f_1,\ldots,f_m] M_T.
  \end{displaymath}
  The orthonormality of the two bases
  and Definition \ref{def:EuclideanInnerProduct} further imply
  \begin{displaymath}
    M_T =
    \begin{bmatrix}
      \innerProd{Te_1,f_1} & \innerProd{Te_2,f_1} & \cdots & \innerProd{Te_n,f_1}
      \\
      \innerProd{Te_1,f_2} & \innerProd{Te_2,f_2} & \cdots & \innerProd{Te_n,f_2}
      \\
      \vdots & \vdots & \ddots & \vdots
      \\
      \innerProd{Te_1,f_m} & \innerProd{Te_2,f_m} & \cdots & \innerProd{Te_n,f_m}
    \end{bmatrix}.
  \end{displaymath}
  The proof is completed by
  repeating the above derivation for $T^*$
  and
  then applying Definitions \ref{def:innerProduct} and \ref{def:adjoint}.
\end{proof}

\begin{lem}
  \label{lem:zeroOpCondition}
  Suppose ${\cal V}$ is a complex inner product space
  and $T\in {\cal L}({\cal V})$.
  If
  \begin{equation}
    \label{eq:zeroOpCondition}
    \forall \mathbf{v}\in {\cal V},\qquad \innerProd{T\mathbf{v},\mathbf{v}}=0,
  \end{equation}
  then $T=\mathbf{0}$.  
\end{lem}
\begin{proof}
  By Definition \ref{def:innerProduct}
  and (\ref{eq:zeroOpCondition}),
  we have, $\forall \mathbf{u},\mathbf{w}\in {\cal V}$,
  \begin{align*}
    \innerProd{T\mathbf{u},\mathbf{w}}
    =& \frac{\innerProd{T(\mathbf{u}+\mathbf{w}),\mathbf{u}+\mathbf{w}}-\innerProd{T(\mathbf{u}-\mathbf{w}),\mathbf{u}-\mathbf{w}}}{4}
    \\
     &+\ii 
       \frac{\innerProd{T(\mathbf{u}+\ii \mathbf{w}),\mathbf{u}+\ii \mathbf{w}}-\innerProd{T(\mathbf{u}-\ii \mathbf{w}),\mathbf{u}-\ii \mathbf{w}}}{4}
    \\
    =& 0.
  \end{align*}
  Setting $\mathbf{w}=T\mathbf{u}$ completes the proof.
\end{proof}

\begin{defn}
  \label{def:selfAdjoint}
  An operator $T\in {\cal L}({\cal V})$
  is \emph{self-adjoint} iff $T=T^*$,
  i.e. 
  \begin{equation}
    \label{eq:selfAdjoint}
    \forall \mathbf{v},\mathbf{w} \in {\cal V},
    \qquad \innerProd{T\mathbf{v},\mathbf{w}} = \innerProd{\mathbf{v},T\mathbf{w}}.
  \end{equation}
\end{defn}

\begin{lem}
  \label{lem:selfAdjointEigenvalueReal}
  Every eigenvalue of a self-adjoint operator $T$ is real.
\end{lem}
\begin{proof}
  Let $(\lambda,\mathbf{u})$ be an eigenpair of $T$.
  % of a self-adjoint operator $T$.
  We have
  \begin{displaymath}
    \lambda \|\mathbf{u}\|^2
    = \innerProd{\lambda \mathbf{u}, \mathbf{u}}
    = \innerProd{T \mathbf{u}, \mathbf{u}}
    = \innerProd{\mathbf{u}, T\mathbf{u}}
    = \innerProd{\mathbf{u}, \lambda\mathbf{u}}
    = \bar{\lambda} \|\mathbf{u}\|^2,
  \end{displaymath}
  where the third step follows from Definition \ref{def:selfAdjoint}.
  Then $\mathbf{u}\ne \mathbf{0}$ implies $\lambda=\bar{\lambda}$.
\end{proof}

\begin{thm}
  \label{thm:selfAdjointSufNecConds}
  Suppose ${\cal V}$ is a complex inner product space
  and $T\in {\cal L}({\cal V})$.
  Then $T$ is self-adjoint if and only if
  \begin{equation}
    \label{eq:selfAdjointSufNecConds}
    \forall \mathbf{v}\in {\cal V},\ \ \innerProd{T\mathbf{v}, \mathbf{v}}\in \mathbb{R}.
  \end{equation}
\end{thm}
\begin{proof}
  By Definitions \ref{def:innerProduct}, \ref{def:adjoint}, 
  and \ref{def:selfAdjoint},
  we have
  \begin{align*}
    \innerProd{T\mathbf{v},\mathbf{v}} - \ccg{\innerProd{T\mathbf{v},\mathbf{v}}}
    &= \innerProd{T\mathbf{v},\mathbf{v}} - \innerProd{\mathbf{v},T\mathbf{v}}
    \\
    &= \innerProd{T\mathbf{v},\mathbf{v}} - \innerProd{T^*\mathbf{v},\mathbf{v}}
      = \innerProd{(T-T^*)\mathbf{v},\mathbf{v}}.
  \end{align*}
  Then Lemma \ref{lem:zeroOpCondition} completes the proof.
\end{proof}

\begin{rem}
  Theorem \ref{thm:selfAdjointSufNecConds} characterizes
  self-adjoint operators
  in that they act like real numbers.
  Indeed, for some orthonormal basis
  a self-ajoint operator does nothing
  but scaling components of vectors; 
  see Theorem \ref{thm:realSpectralThm}.
\end{rem}

\begin{lem}
  \label{lem:zeroOpConditionSelfAdjoint}
  Suppose ${\cal V}$ is a real inner product space
  and $T\in {\cal L}({\cal V})$.
  If $T$ is self-adjoint
  and satisfies
  \begin{equation}
    \label{eq:zeroOpConditionSelfAdjoint}
    \forall \mathbf{v}\in {\cal V},\qquad \innerProd{T\mathbf{v},\mathbf{v}}=0,
  \end{equation}
  then $T=\mathbf{0}$.
\end{lem}
\begin{proof}
  By the self-adjointness and the underlying field being real,
  we have
  \begin{displaymath}
    \innerProd{T\mathbf{w},\mathbf{u}} = \innerProd{\mathbf{w},T\mathbf{u}} = \innerProd{T\mathbf{u},\mathbf{w}},
  \end{displaymath}
  which, together with Definition \ref{def:innerProduct},
  implies
  \begin{displaymath}
    \innerProd{T\mathbf{u},\mathbf{w}} = \frac{\innerProd{T(\mathbf{u}+\mathbf{w}),\mathbf{u}+\mathbf{w}} - \innerProd{T(\mathbf{u}-\mathbf{w}),\mathbf{u}-\mathbf{w}}}{4}.
  \end{displaymath}
  Setting $\mathbf{w}=T\mathbf{u}$ completes the proof.
\end{proof}

\begin{rem}
  It is instructive to compare
  Lemma \ref{lem:zeroOpCondition}
  and Lemma \ref{lem:zeroOpConditionSelfAdjoint}.
  The condition (\ref{eq:zeroOpCondition})
  in Lemma \ref{lem:zeroOpCondition}
  is stronger than that %(\ref{eq:zeroOpConditionSelfAdjoint})
  in Lemma \ref{lem:zeroOpConditionSelfAdjoint}
  because both the real part
  and the imaginary part of the inner product
  have to be zero.
  For a real inner product space,
  the condition (\ref{eq:zeroOpConditionSelfAdjoint}) 
  itself cannot guarantee $T=\mathbf{0}$, 
  c.f. the example of $T$ being a rotation of $\frac{\pi}{2}$.
  With self-adjointness added, $T=\mathbf{0}$ holds again.

  Although $\mathbb{C}\simeq \mathbb{R}^2$,
  the Euclidean inner product in Definition
  \ref{def:EuclideanInnerProduct}
  on these two vector spaces are different in that
  the inner product of
  two perpendicular vectors in $\mathbb{R}^2$
  is zero because they are perpendicular
  while that in $\mathbb{C}$
  is zero only when at least one of these two vectors
  is the zero vector.
\end{rem}


\subsection{Normal operators}
\label{sec:normal-operators}

\begin{defn}
  \label{def:normalOps}
  An operator $T\in {\cal L}({\cal V})$
  is \emph{normal} iff $TT^*=T^*T$.
\end{defn}

\begin{coro}
  \label{coro:TTsIsNormal}
  Every self-adjoint operator is normal.
\end{coro}

\begin{rem}
  Denote by ``$\subset$'' the ``is-a'' relation.
  Then 
  \begin{center}
    self-adjoint operators $\subset$ normal operators
    $\subset$ linear operators.
  \end{center}
  The strict inclusion of the first $\subset$
  is justified by the fact
  that skew self-adjoint operators and orthogonal operators
  are also normal, but not self-adjoint.
\end{rem}

\begin{lem}
  \label{lem:normalOpCondition}
  % An operator
  $T\in {\cal L}({\cal V})$
  is normal if and only if
  \begin{equation}
    \label{eq:normalOpCondition}
    \forall \mathbf{v}\in {\cal V},\qquad
    \|T\mathbf{v}\| = \|T^*\mathbf{v}\|.
  \end{equation}
\end{lem}
\begin{proof}
  By Lemma \ref{lem:zeroOpConditionSelfAdjoint}
  and Definition \ref{def:adjoint},
  we have
  \begin{align*}
    T^*T=TT^* %=\mathbf{0}
    &\Leftrightarrow
      \forall \mathbf{v}\in {\cal V},\ 
      \innerProd{(T^*T-TT^*)\mathbf{v}, \mathbf{v}}=0
    \\
    &\Leftrightarrow
      \forall \mathbf{v}\in {\cal V},\ 
      \innerProd{T^*T\mathbf{v}, \mathbf{v}}=\innerProd{TT^*\mathbf{v}, \mathbf{v}}
    \\
    &\Leftrightarrow
      \forall \mathbf{v}\in {\cal V},\ 
      \|T\mathbf{v}\|^2 = \|T^*\mathbf{v}\|^2.
  \end{align*}
  The positivity of a norm completes the proof.
\end{proof}

\begin{lem}
  \label{lem:normalOpEigenVecCondition}
  %An operator
  $T\in {\cal L}({\cal V})$
  is normal if and only if
  each eigenvector of $T$
  is also an eigenvector of $T^*$.
\end{lem}
\begin{proof}
  If $T$ is normal, so is $T-\lambda I$.
  By Lemma \ref{lem:normalOpCondition},
  an eigenpair $(\lambda, \mathbf{u})$ of $T$ satisfies
  \begin{displaymath}
    0=\|(T-\lambda I)\mathbf{u}\|
    =\|(T-\lambda I)^*\mathbf{u}\|
    =\|(T^* - \bar{\lambda} I)\mathbf{u}\|,
  \end{displaymath}
  and thus $\mathbf{u}$ is also an eigenvector of $T^*$.

  Conversely, suppose each eigenvector $\mathbf{u}$ of $T$
  is also an eigenvector of $T^*$.
  Then the above equation implies that the corresponding eigenvalue of
  $T^*$ is the conjugate of that of $T$.
  It suffices to prove that these eigenvectors form a basis of $V$,
  because then we have
  \begin{displaymath}
    T U = U \Lambda,\quad T^* U = U \Lambda^*,
  \end{displaymath}
  where $U$ is the matrix of these eigenvectors.
  Thus
  \begin{displaymath}
    TT^*U = TU\Lambda^* = U\Lambda\Lambda^* = U\Lambda^*\Lambda =
    T^*U\Lambda = T^*TU
  \end{displaymath}
  and we have $TT^*=T^* T$ because $U$ is nonsingular.
  By Theorem \ref{thm:diagonalizabilityConditions},
  it suffices to show that $T$ is diagonalizable.
  By Theorem \ref{thm:JordanBasis},
  we only need to show that
  for any eigenpair $(\lambda, \mathbf{u})$ of $T$, 
  \begin{displaymath}
    (A-\lambda I)^2\mathbf{u}=\mathbf{0}\
    \Rightarrow\ (A-\lambda I)\mathbf{u}=\mathbf{0},
  \end{displaymath}
  because this condition will annihilate
  all Jordan blocks of size greater than 1.
  Define $\mathbf{v}= (A-\lambda I)\mathbf{u}$
  and we have
  \begin{displaymath}
    \begin{array}{l}
      \innerProd{A\mathbf{u},\mathbf{v}}= \innerProd{\mathbf{u}, A^*\mathbf{v}}
      = \innerProd{\mathbf{u}, \bar{\lambda}\mathbf{v}}
      = \innerProd{\lambda\mathbf{u}, \mathbf{v}},
      \\
      \innerProd{A\mathbf{u}, \mathbf{v}}
      = \innerProd{\mathbf{v}+\lambda\mathbf{u}, \mathbf{v}}
      = \|\mathbf{v}\|^2 + \innerProd{\lambda\mathbf{u}, \mathbf{v}}, 
    \end{array}
  \end{displaymath}
  which imply %$\|\mathbf{v}\|^2=0$, i.e.,
  $\mathbf{v}=0$, i.e., $(A-\lambda I)\mathbf{u}=\mathbf{0}$.
\end{proof}

\begin{rem}
  In the proof of Lemma \ref{lem:normalOpEigenVecCondition},
  we see that the condition of normal operators
  does not allow the existence of Jordan blocks.
\end{rem}

\begin{thm}
  \label{thm:normalFormInRealIPspaces}
  For a linear operator $T\in {\cal L}({\cal V})$
  on a two-dimensional real inner product space ${\cal V}$, 
  the following are equivalent:
  \begin{enumerate}[(a)]\itemsep0em
  \item $T$ is normal but not self-adjoint.
  \item The matrix of $T$ with respect to every orthonormal basis of
    ${\cal V}$ has the form
    \begin{equation}
      \label{eq:normalFormInRealIPspaces}
      M(T) =
      \begin{bmatrix}
        a & -b
        \\
        b & a
      \end{bmatrix}
    \end{equation}
    where $b\ne 0$.
    % \item The matrix of $T$ with respect to some orthonormal basis of
    %   ${\cal V}$ has the form
    %   \begin{equation}
    %     \label{eq:normalNotSelfAdjointMat}
    %     M(T) =
    %     \begin{bmatrix}
    %       a & -b
    %       \\
    %       b & a
    %     \end{bmatrix}
    %   \end{equation}
    %   where $b> 0$.
  \end{enumerate}
\end{thm}
\begin{proof}
  (b) $\Rightarrow$ (a) trivially holds,
  so we only prove \mbox{(a) $\Rightarrow$ (b)}.
  Let $(e_1,e_2)$ be an orthonormal basis of ${\cal V}$
  and set 
  \begin{displaymath}
    M(T,(e_1,e_2)) =
    \begin{bmatrix}
      a & c
      \\
      b & d
    \end{bmatrix}.
  \end{displaymath}
  By Definition \ref{def:normalOps}, we have
  \begin{align*}
    &
    \begin{bmatrix}
      a & c
      \\
      b & d
    \end{bmatrix}
    \begin{bmatrix}
      a & b
      \\
      c & d
    \end{bmatrix}
          =
    \begin{bmatrix}
      a^2+c^2 & ab+cd
      \\
      ab+cd & b^2+d^2
    \end{bmatrix}
    \\
    =&
    \begin{bmatrix}
      a & b
      \\
      c & d
    \end{bmatrix}
    \begin{bmatrix}
      a & c
      \\
      b & d
    \end{bmatrix}
    =
    \begin{bmatrix}
      a^2+b^2 & ac+bd
      \\
      ac+bd & c^2+d^2
    \end{bmatrix}.
  \end{align*}
  $b^2=c^2$
  and the condition of $T$ being not self-adjoint
  further yields
  $c=-b \ne 0$,
  which, together with $ab+cd=ac+bd$,
  yields $a=d$.
  % -----------------------------------------
  % The following is an incomplete proof
  % -----------------------------------------
  % By Definition \ref{def:matrixOfLinearMap},
  % Theorem \ref{thm:Pythagorean},
  % and Definition \ref{def:orthonormalList},
  % we have $\|Te_1\|^2=a^2+b^2$.
  % In addition, Theorem \ref{thm:matrixRelationOfAdjoints}
  % yields $\|T^*e_1\|^2=a^2+c^2$.
  % Then Lemma \ref{lem:normalOpCondition}
  % implies $b^2=c^2$
  % and the condition of $T$ being not self-adjoint
  % further yields
  % $c=-b \ne 0$.
\end{proof}


\subsection{The spectral theorems}
\label{sec:spectral-theorem}

\begin{thm}[Complex spectral]
  \label{thm:complexSpectralThm}
  For a linear operator $T\in {\cal L}({\cal V})$
  with $\mathbb{F}=\mathbb{C}$, 
  the following are equivalent:
  \begin{enumerate}[(a)]\itemsep0em
  \item $T$ is normal;
  \item ${\cal V}$ has an orthonormal basis consisting of eigenvectors of
    $T$;
  \item $T$ has a diagonal matrix with respect
    to some orthonormal basis of ${\cal V}$.
  \end{enumerate}
\end{thm}

\begin{coro}
  \label{coro:normalMatWithRealEigenValIsSelfAdjoint}
  A normal operator $T$ whose eigenvalues are real is self-adjoint.
\end{coro}
\begin{proof}
  By Theorem \ref{thm:complexSpectralThm},
  we write $T=V\Lambda V^{-1}$
  and thus $T^*=V \Lambda^* V^{-1}$.
  Then all entries in $\Lambda$ being real
  implies $T=T^*$.
\end{proof}

\begin{rem}
  With respect to normal operators, 
  Corollary \ref{coro:normalMatWithRealEigenValIsSelfAdjoint}
  is the converse of
  Lemma \ref{lem:selfAdjointEigenvalueReal}.
\end{rem}

\begin{thm}[Real spectral]
  \label{thm:realSpectralThm}
  For a linear operator $T\in {\cal L}({\cal V})$
  with $\mathbb{F}=\mathbb{R}$, 
  the following are equivalent:
  \begin{enumerate}[(a)]\itemsep0em
  \item $T$ is self-adjoint;
  \item ${\cal V}$ has an orthonormal basis consisting of eigenvectors of
    $T$;
  \item $T$ has a diagonal matrix with respect
    to some orthonormal basis of ${\cal V}$.
  \end{enumerate}
\end{thm}

\begin{rem}
  Note that the condition (b) is the same
  for both Theorem \ref{thm:complexSpectralThm}
  and Theorem \ref{thm:realSpectralThm}.
  These theorems are important
   for numerical solutions of systems of ordinary differential
   equations
   of the form $\frac{\dif \mathbf{x}}{\dif t} = A \mathbf{x}$
   where $A$ is a square matrix.
\end{rem}


\subsection{Isometries}

\begin{defn}
  \label{def:linearIsometry}
  An operator $S\in {\cal L}({\cal V})$
  is called a (linear) \emph{isometry} iff
  \begin{equation}
    \label{eq:linearIsometry}
    \forall \mathbf{v}\in {\cal V}, \qquad \|S \mathbf{v}\| = \|\mathbf{v}\|.
  \end{equation}
\end{defn}

\begin{thm}
  \label{thm:isometryFormsInRealIPspaces}
  An operator $S\in {\cal L}({\cal V})$
  on a real inner product space
  is an isometry
  if and only if
  there exists an orthonormal basis of ${\cal V}$
  with respect to which ${\cal S}$ has a block diagonal matrix
  such that each block on the diagonal is a 1-by-1 matrix
  containing 1 or $-1$,
  or, is a 2-by-2 matrix of the form
  \begin{equation}
    \label{eq:isometryFormsInRealIPspaces}
    \begin{bmatrix}
      \cos\theta & -\sin\theta
      \\
      \sin\theta & \cos \theta
    \end{bmatrix}
  \end{equation}
  where $\theta\in (0,\pi)$.
\end{thm}

\begin{coro}
  \label{coro:isometryTypes}
  For an operator ${\cal S}\in {\cal L}({\cal V})$
  on a two-dimensional real inner product space, 
  the following are equivalent:
  \begin{enumerate}[(a)]\itemsep0em
  \item ${\cal S}$ is an isometry;
  \item ${\cal S}$ is either an identity or a reflection or a rotation.
  \end{enumerate}
\end{coro}

\begin{rem}
  By Theorem \ref{thm:normalFormInRealIPspaces}
   and Corollary \ref{coro:isometryTypes},
   an isometry is a special type of normal operators, 
   but it may or may not be self-adjoint.
\end{rem}


\subsection{Singular value decomposition}
\label{sec:sing-value-decomp}

\begin{defn}
  \label{def:positiveDefinite}
  An operator $T\in{\cal L}(V)$ is \emph{positive semi-definite}
  iff
  \begin{equation}
    \label{eq:positiveSemiDefinite}
    \forall \mathbf{v}\in V,\quad
    \innerProd{T\mathbf{v}, \mathbf{v}} \ge 0
  \end{equation}
  and is \emph{positive definite} iff 
  \begin{equation}
    \label{eq:positiveDefinite}
    \forall \mathbf{v}\in V\setminus\{\mathbf{0}\},\quad
    \innerProd{T\mathbf{v}, \mathbf{v}} > 0.
  \end{equation}
  An operator is \emph{positive}
  if it is self-adjoint
  and positive semi-definite.
\end{defn}

\begin{coro}
  \label{coro:fStarCatf2SelfAdj}
  For any linear operator $f\in {\cal L}({\cal V})$,
  both $f^*\circ f$ and $f\circ f^*$ are positive.
\end{coro}
\begin{proof}
  By Definition \ref{def:selfAdjoint}, % and \ref{def:adjoint},
  $f^*\circ f$ is self-adjoint since
  \begin{displaymath}
    \innerProd{(f^*\circ f)\mathbf{u}, \mathbf{v}}
    = \innerProd{ f\mathbf{u}, f \mathbf{v}}
    = \innerProd{ \mathbf{u}, (f^*\circ f) \mathbf{v}}.
  \end{displaymath}
  Suppose $(\lambda, \mathbf{u})$ is an eigenpair of $(f^*\circ f)$.
  Then we have
  \begin{align*}
    &\lambda\innerProd{\mathbf{u}, \mathbf{u}}
      = \innerProd{(f^*\circ f)\mathbf{u}, \mathbf{u}}
      = \innerProd{ f\mathbf{u}, f \mathbf{u}}
    \\
    \Rightarrow
    & \lambda = \frac{\innerProd{ f\mathbf{u}, f
      \mathbf{u}}}{\innerProd{\mathbf{u}, \mathbf{u}}}\ge 0.
  \end{align*}
  Similar arguments apply to $f\circ f^*$. 
\end{proof}

\begin{rem}
  Corollary \ref{coro:fStarCatf2SelfAdj} tells us that both $f^*\circ f$ and $f\circ f^*$
  can be diagonalized and they have nonnegative eigenvalues.
  This is important 
  as it leads to the concept of singular values.
\end{rem}

\begin{defn}
  \label{def:singularValues}
  The \emph{singular values} of a linear map
  $f: \mathbb{C}^n\rightarrow \mathbb{C}^m$
%  $f\in {\cal L}({\cal V}, {\cal W})$
  are the non-negative square roots of the eigenvalues
  of $f^*\circ f$,  % (and $f\circ f^*$)
  usually sorted in non-increasing order as 
  \begin{displaymath}
    \sigma_1 \ge \sigma_2 \ge \cdots \ge \sigma_r
    > \sigma_{r+1} = \cdots = \sigma_n = 0, 
  \end{displaymath}
  where $r$ is the rank of $f$.
\end{defn}

\begin{thm}
\label{thm:SVD}
  For any matrix $A\in \mathbb{C}^{m\times n}$ with rank $r$, 
  there exist orthonormal bases
  $\mathbb{C}^n=\Span\{\mathbf{u}_1, \mathbf{u}_2, \ldots, \mathbf{u}_n\}$
  and $\mathbb{C}^m=\Span\{\mathbf{v}_1, \mathbf{v}_2, \ldots,
  \mathbf{v}_m\}$ such that
  \begin{subequations}
  \begin{align}
    \label{eq:SVDtransform}
    \forall j=1, 2, \ldots, r, \quad
    \left\{
    \begin{array}{l}
    A \mathbf{u}_j = \sigma_j \mathbf{v}_j;\\
    A^* \mathbf{v}_j = \sigma_j \mathbf{u}_j,
    \end{array}
    \right.
    \\
    \label{eq:SVDannihilate}
    \left\{
    \begin{array}{l}
    \forall j=r+1, r+2, \ldots, n, \quad
      A \mathbf{u}_j = \mathbf{0}; \\
    \forall j=r+1, r+2, \ldots, m, \quad
      A^* \mathbf{v}_j = \mathbf{0},
    \end{array}
    \right.
  \end{align}
  \end{subequations}
  where $\sigma_j$'s are the singular values of $A$
  in Definition \ref{def:singularValues}. 
\end{thm}
\begin{proof}
  The matrix $A^*A\in \mathbb{C}^{n\times n}$
  is self-adjoint and thus normal.
  By Theorem \ref{thm:complexSpectralThm}, 
  $\mathbb{C}^{n}$ has an orthonormal basis
  that are also eigenvectors of $A^*A$; 
  choose them to be
  $\{\mathbf{u}_1, \mathbf{u}_2, \ldots, \mathbf{u}_n\}$.
  By Definition \ref{def:singularValues} we have
  \begin{displaymath}
    A^* A \mathbf{u}_j = \sigma_j^2 \mathbf{u}_j.
  \end{displaymath}
  Then we choose %the first $r$ $\mathbf{v}_j$'s to be
  \begin{displaymath}
    \forall j=1,2,\ldots,r, \quad
    \mathbf{v}_j := \frac{1}{\sigma_j} A \mathbf{u}_j,
  \end{displaymath}
  which implies
  \begin{displaymath}
    \innerProd{\mathbf{v}_i, \mathbf{v}_j}
    = \frac{1}{\sigma_i\sigma_j}
    \innerProd{A\mathbf{u}_i, A \mathbf{u}_j}
    = \frac{1}{\sigma_i\sigma_j}
    \innerProd{\mathbf{u}_i, A^*A \mathbf{u}_j}
    = \frac{\sigma_j}{\sigma_i}
    \innerProd{\mathbf{u}_i, \mathbf{u}_j}. %= \delta_{i,j},
  \end{displaymath}
  Then the orthonormality of $\mathbf{u}_j$'s
  implies that the first $r$ $\mathbf{v}_j$'s
  are orthonormal and if $r<m$ we can extend them
  by the Gram-Schmidt process
  to arrive at an orthonormal basis of $\mathbb{C}^m$.
  Therefore (\ref{eq:SVDtransform}) holds.

  The first line of (\ref{eq:SVDannihilate}) follows from
  $\Null A = \Null (A^* A)$, which is implied by
  \begin{displaymath}
    \|A\mathbf{u}\|^2=\innerProd{A\mathbf{u}, A\mathbf{u}}
    = \innerProd{\mathbf{u}, A^*A\mathbf{u}}.
  \end{displaymath}
  The rank of $A^*$ is $r$. 
  If $r=m$,
  the second line of (\ref{eq:SVDannihilate}) holds vacuously.
  Otherwise $r<m$. The fundamental theorem of linear algebra
  (Theorem \ref{thm:fundamentalLinearAlgebra}) implies that
  $\null A^*$ has rank $m-r$, which completes the proof.  
\end{proof}

\begin{defn}
  \label{def:SVD}
  The \emph{singular value decomposition} (SVD)
  of a rectangular matrix $A\in \mathbb{C}^{m\times n}$
  is the factorization $A=V\Sigma U^*$
  where $\Sigma$ is a diagonal matrix
  with its diagonal entries as the singular values
  in Definition \ref{def:singularValues}
  and $V$ and $U$ are unitary matrices whose columns
  are respectively the vectors $\mathbf{v}_j$'s and $\mathbf{u}_j$'s
  specified in Theorem \ref{thm:SVD}, 
  which are also called the \emph{left singular vectors}
  and the \emph{right singular vectors} of $A$,
  respectively.
  We also refer to the sequence of triples
  $(\sigma_j,\mathbf{u}_j,\mathbf{v}_j)$
  as the \emph{singular system} of $A$.
\end{defn}

% \begin{thm}
%   \label{thm:existenceOfSVD}
%   Any matrix $A\in \mathbb{C}^{m\times n}$ has an SVD. 
% \end{thm}

\begin{rem}%[Geometric interpretation of SVD]
  \label{rem:geometryOfSVD}
  A matrix $A\in \mathbb{C}^{m\times n}$
  can be viewed as a transformation
  from $n$-vectors to $m$-vectors.
  Consider transforming a unit ball in $\mathbb{R}^n$
  to a hyper-ellipse in $\mathbb{R}^m$.
  The left singular vectors
  form an orthonormal basis of $\mathbb{R}^m$
  and the right singular vectors
  form an orthonormal basis of $\mathbb{R}^n$.
  The semi-axes of the hyper-ellipse in $\mathbb{R}^m$
  are the vectors $A\mathbf{u}_1=\sigma_1\mathbf{v}_1$
  and $A\mathbf{u}_2=\sigma_2\mathbf{v}_2$, 
  as shown in Figure \ref{fig:geometricMeaningOfSVD}. 
\end{rem}

\begin{Figure}
  \centering
  \includegraphics[width=0.8\hsize]{pst/svd}
  \captionof{figure}{Geometric interpretation of SVD.
  }
  \label{fig:geometricMeaningOfSVD}
\end{Figure}

\begin{rem}
  If $A$ has rank $r$, 
  the singular value decomposition $A = V\Sigma U^T$ yields
  \begin{align*}
    {\cal R}(A) &= \Span\{V_1, V_2, \ldots, V_r\},
    \\
    {\cal N}(A) &= \Span\{U_{r+1}, U_{r+2}, \ldots, U_n\},
    \\
    {\cal R}(A^T) &= \Span\{U_1, U_2, \ldots, U_r\},
    \\
    {\cal N}(A^T) &= \Span\{V_{r+1}, V_{r+2}, \ldots, V_m\}.
  \end{align*}
  This is closely related to Theorem \ref{thm:fundamentalLinearAlgebra}.
\end{rem}

\begin{rem}
  A self-adjoint operator $T$ on a real inner product space
   is special in that, 
   with respect to some orthonormal basis, 
   its effect is nothing but
   scaling components of all vectors:
   $T=P\Lambda P^*$. 
   c.f. (\ref{eq:selfAdjoint})
   and Corollary \ref{coro:unitaryMatricesPreserveNorms}:
   \begin{align*}
     &\innerProd{P\Lambda P^*\mathbf{v},\mathbf{w}}
%     = \innerProd{\Lambda P^*\mathbf{v},P^*\mathbf{w}}
     = \innerProd{P\Lambda P^* \mathbf{v}, P  P^* \mathbf{w}}
       = \innerProd{\Lambda P^* \mathbf{v},  P^* \mathbf{w}}
       \\
     =& \innerProd{\Lambda \mathbf{v}', \mathbf{w}'}
     = \innerProd{ \mathbf{v}', \Lambda\mathbf{w}'}
     = \innerProd{\mathbf{v}, P\Lambda P^*\mathbf{w}}. 
   \end{align*}
%
  % In the most general case of a linear operator $A$, 
  %  we have $A=P\Sigma Q^*$
  %  where $Q$ can not be expressed as a rotation of $P$. 
\end{rem}

\begin{defn}
  \label{def:similarMatrices}
  Two matrices $A, B\in \mathbb{R}^{n\times n}$
  are called \emph{similar} iff
  there exists an invertible matrix $P$
  such that $B=P^{-1}AP$.
  The map $A\mapsto P^{-1}AP$
  is called a \emph{similarity transformation}
  or \emph{conjugation of the matrix} $A$.
\end{defn}

\begin{rem}
  Two matrices are similar
  if and only if they can be considered
  as matrices of the same linear operator
  under two bases.
\end{rem}
% \begin{defn}
%   In $\mathbb{C}^n$,
%   a sequence of vectors
%   $\{\mathbf{x}_1, \mathbf{x}_2, \ldots \}$
%   is said to \emph{converge} to a vector
%   $\mathbf{x}$
%   iff $\lim_{m\rightarrow\infty} \|\mathbf{x}_m-\mathbf{x}\|=0$.
% \end{defn}


\section{Trace and determinant}

%\subsection{Trace}

\begin{defn}
  \label{def:trace}
  The \emph{trace of a matrix} $A$, denoted by $\text{Trace } A$,
  is the sum of the diagonal entries of $A$. 
\end{defn}

\begin{lem}
  \label{lem:trace}
  The trace of a matrix is the sum of its eigenvalues, 
  each of which is repeated according to its multiplicity.
\end{lem}

% \subsection{Determinants}
% \label{sec:determinants}

\begin{defn}
  A \emph{permutation of a set} $A$
  is a bijective function $\sigma: A \rightarrow A$.
\end{defn}

\begin{defn}
  \label{def:permutationSignature}
  Let $\sigma$ be a permutation of $A=\{1,2,\ldots, n\}$
  and 
  let $s$ denote the number of pairs of integers $(j,k)$ 
  with $1\le j < k \le n$ such that
  $j$ appears after $k$ in the list $(m_1, \ldots, m_n)$
  given by $m_i=\sigma(i)$.
  The \emph{sign of the permutation} $\sigma$
  is 1 if $s$ is even and $-1$ if $s$ is odd.
\end{defn}

\begin{defn}
  \label{def:signedVolume}
  The \emph{signed volume of a parallelotope}
  % i.e. the generalization of parallelopiped in $\mathbb{R}^n$,
  spanned by $n$ vectors
  $\mathbf{v}_1, \mathbf{v}_{2},\ldots, \mathbf{v}_n\in \mathbb{R}^n$
  is a function $\delta: \mathbb{R}^{n\times n} \rightarrow \mathbb{R}$
  that satisfies
  \begin{enumerate}[(SVP-1)]
  \item $\delta(I) = 1$;
  \item $\delta(\mathbf{v}_1, \mathbf{v}_2, \ldots, \mathbf{v}_n)=0$
    if $\mathbf{v}_i=\mathbf{v}_j$ for some $i\ne j$;
  \item $\delta$ is linear, i.e.,
    $\forall j=1,\ldots,n,\ \forall c\in \mathbb{R}$,
    \begin{equation}
      \label{eq:linearityOfSignedVolumn}
      \begin{array}{l}
        \ \ \ \delta(\mathbf{v}_1, \ldots,
          \mathbf{v}_{j-1}, \mathbf{v}+ c\mathbf{w}, \mathbf{v}_{j+1},
          \ldots, \mathbf{v}_n)
        \\
        = \delta(\mathbf{v}_1, \ldots,
            \mathbf{v}_{j-1}, \mathbf{v}, \mathbf{v}_{j+1},
            \ldots, \mathbf{v}_n)
        \\
        \ \ + 
      c\delta(\mathbf{v}_1, \ldots,
      \mathbf{v}_{j-1}, \mathbf{w}, \mathbf{v}_{j+1},
      \ldots, \mathbf{v}_n).
      \end{array}
    \end{equation}
  \end{enumerate}
\end{defn}

\begin{exc}
  Give a geometric proof that
  the signed volume of the parallelogram
  determined by the two vectors $\mathbf{v}_1=(a,b)^T$
  and  $\mathbf{v}_2=(c,d)^T$
  is 
   \begin{equation}
     \label{eq:determinantN2}
    \delta(\mathbf{v}_1, \mathbf{v}_2)
    % \begin{bmatrix}
    %   a & c
    %   \\
    %   b & d
    % \end{bmatrix}
    = ad-bc = \innerProd{\mathbf{v}^{\perp}_1, \mathbf{v}_2}.
  \end{equation}
\end{exc}
\begin{solution}
  By high school geometry, we have
  \begin{displaymath}
    \delta = |\mathbf{v}_1||\mathbf{v}_2|\sin \alpha
    = |\mathbf{v}^{\perp}_1||\mathbf{v}_2|\cos(\frac{\pi}{2}-\alpha)
    = \innerProd{\mathbf{v}^{\perp}_1, \mathbf{v}_2}
  \end{displaymath}
  where $|\cdot|$ denotes the Euclidean 2-norm, 
  $\alpha$ the angle between the two vectors, 
  and $\mathbf{v}^{\perp}_1=(-b,a)^T$ the vector obtained from $\mathbf{v}_1$
  by a counterclockwise rotation of $\frac{\pi}{2}$.
\end{solution}

\begin{rem}
  The conditions in Definition \ref{def:signedVolume} capture
  the properties of signed volumes.
  (SVP-1) says that the volume of a cube is 1.
  (SVP-2) says that the volume is zero
  if two spanning vectors coincide.
  (SVP-3) says that if we fix any $n-1$ spanning vectors
  and change the last vector,
  the volume change is linear with the change of the last vector
  because (\ref{eq:determinantN2}) yields
  \begin{align*}
    \delta(\mathbf{u}, \mathbf{v})
    + \delta(\mathbf{w}, \mathbf{v})
    &= \innerProd{\mathbf{u}^{\perp}, \mathbf{v}}
      + \innerProd{\mathbf{w}^{\perp}, \mathbf{v}}
      = \innerProd{\mathbf{u}^{\perp}+ {\mathbf{w}}^{\perp}, \mathbf{v}}
      \\
    & = \innerProd{(\mathbf{u}+\mathbf{w})^{\perp}, \mathbf{v}}
      = \delta(\mathbf{u}+\mathbf{w}, \mathbf{v}).
  \end{align*}
  Hereafter we shall see how these physical intuitions
%  (or reasonable assumptions)
  lead to a unique definition of signed volume,
  which is identical to Leibniz formula of determinants.  
\end{rem}

\begin{lem}
  Adding a multiple of one vector to another does not
  change the signed volume.
\end{lem}
\begin{proof}
  This follows directly from (SVP-2,3).
\end{proof}

\begin{lem}
  If the vectors $\mathbf{v}_1, \mathbf{v}_{2},\ldots,
  \mathbf{v}_{n}$ are linearly dependent,
  then $\delta(\mathbf{v}_1, \mathbf{v}_{2}, \ldots, \mathbf{v}_{n})=0$.
\end{lem}
\begin{proof}
  WLOG, we assume $\mathbf{v}_1=\sum_{i=2}^n c_i \mathbf{v}_i$.
  Then the result follows from (SVP-2,3).
\end{proof}

\begin{lem}
  \label{lem:alternatingSignedVolume}
  The signed volume $\delta$ is alternating, i.e.,
  \begin{equation}
    \label{eq:alternatingSignedVolume}
    \delta(\mathbf{v}_1, \ldots,
    \mathbf{v}_{i},\ldots, \mathbf{v}_{j},
    \ldots, \mathbf{v}_n)
    = - 
    \delta(\mathbf{v}_1, \ldots,
    \mathbf{v}_{j},\ldots, \mathbf{v}_{i},
    \ldots, \mathbf{v}_n).
  \end{equation}
\end{lem}

\begin{exc}
  Prove Lemma \ref{lem:alternatingSignedVolume} using (SVP-2,3).
\end{exc}
\begin{solution}
  Define the shorthand notation
  \begin{displaymath}
    \delta(\mathbf{v}_i, \mathbf{v}_j)
    := \delta(\mathbf{v}_1, \ldots,
    \mathbf{v}_{i},\ldots, \mathbf{v}_{j},
    \ldots, \mathbf{v}_n), 
  \end{displaymath}
  apply (SVP-2) and (SVP-3) repeatedly, and we have 
  \begin{align*}
    \delta(\mathbf{v}_i, \mathbf{v}_j)
    &= \delta(\mathbf{v}_i, \mathbf{v}_j)
      + \delta(\mathbf{v}_j, \mathbf{v}_j)
    \\
    &= \delta(\mathbf{v}_i+\mathbf{v}_j, \mathbf{v}_j)
      - \delta(\mathbf{v}_i+\mathbf{v}_j, \mathbf{v}_i+\mathbf{v}_j)
    \\
    &= \delta(\mathbf{v}_i+\mathbf{v}_j, -\mathbf{v}_i)
      + \delta(-\mathbf{v}_i, -\mathbf{v}_i)
    \\
    &= \delta(\mathbf{v}_j, -\mathbf{v}_i)
      = -\delta(\mathbf{v}_j, \mathbf{v}_i).
  \end{align*}
\end{solution}

\begin{lem}
  \label{lem:matrixPermIsSgnSigma}
  Let $M_{\sigma}$ denote the matrix of a permutation
  $\sigma: E \rightarrow E$ where $E$ is the set of standard basis
  vectors in (\ref{eq:standardBasis}). 
%  (\ref{eq:standardBasisVectors}).
  Then we have $\delta(M_{\sigma}) = \sgn(\sigma)$.
\end{lem}
\begin{proof}
  There is a one-to-one correspondence
  between the vectors in the matrix
  \begin{displaymath}
    M_{\sigma} = [e_{\sigma(1)}, e_{\sigma(2)}, \ldots, e_{\sigma(n)}]
  \end{displaymath}
  and the scalars in the one-line notation
  \begin{displaymath}
    (\sigma(1)\  \sigma(2)\ \ldots\ \sigma(n)).
  \end{displaymath}
  A sequence of transpositions taking $\sigma$
   to the identity map also
   takes $M_{\sigma}$ to the identity matrix.
  By Lemma \ref{lem:alternatingSignedVolume},
  each transposition yields a multiplication factor $-1$.
  Definition \ref{def:permutationSignature}
  and (SVP-1) give
  $\delta(M_{\sigma})=\sgn(\sigma)\delta(I)=\sgn(\sigma)$.
\end{proof}

\begin{defn}[Leibniz formula of determinants]
  \label{def:determinantOfLeibniz}
  The \emph{determinant of a square matrix}
  $A\in \mathbb{R}^{n\times n}$
  is 
  \begin{equation}
    \label{eq:determinantOfLeibniz}
    \det A = \sum_{\sigma\in S_n} \sgn(\sigma) \prod_{i=1}^n
    a_{\sigma(i),i},
  \end{equation}
  where the sum is over 
  the symmetric group $S_n$ of all permutations
%  $S_n: \{\sigma: \{1,2,\ldots, n\}\rightarrow \{1,2,\ldots, n\}\}$
  and $a_{\sigma(i),i}$ is the element of $A$
  at the $\sigma(i)$th row and the $i$th column.
\end{defn}

\begin{rem}
  The formula in (\ref{eq:determinantOfLeibniz})
  can be easily memorized by its symmetry
  and the aid of (\ref{eq:determinantN2}).
  Each RHS term of (\ref{eq:determinantN2}) is a product of $2$
  numbers,
  which come from different rows and different columns.
  For $n>2$, this implies that
  no two numbers in a product term
  come from the same row, nor the same column.
  This condition is equivalent to require
  that the row index function of the column index
  be a permutation.
  The sign of the permutation is natural
  for each product.
  Finally, all permutations in the symmetric group
  are equal as far as signed volumn is concerned.
\end{rem}

\begin{lem}
  \label{lem:determinantIsProductOfEigenvalues}
  The determinant of a matrix is the product of its eigenvalues, 
  each of which is repeated according to its multiplicity.
\end{lem}

\begin{thm}
  The signed volume function satisfying (SVP-1,2,3)
  in Definition \ref{def:signedVolume} is unique
  and is the same as the determinant
  in (\ref{eq:determinantOfLeibniz}).
\end{thm}
\begin{proof}
  Let the parallelotope be spanned by the column vectors
  $\mathbf{v}_1, \mathbf{v}_{2},\ldots, \mathbf{v}_n$.
  We have
  \begin{align*}
    & \delta
    \begin{bmatrix}
      v_{11} & v_{12} & \ldots & v_{1n}
      \\
      v_{21} & v_{22} & \ldots & v_{2n}
      \\
      \vdots & \vdots & \ddots & \vdots
      \\
      v_{n1} & v_{n2} & \ldots & v_{nn}
    \end{bmatrix}
    \\
    =& \sum_{i_1=1}^n v_{i_1 1}
    \delta
    \begin{bmatrix}
      | & v_{12} & \ldots & v_{1n}
      \\
      e_{i_1} & v_{22} & \ldots & v_{2n}
      \\
      | & \vdots & \ddots & \vdots
      \\
      | & v_{n2} & \ldots & v_{nn}
    \end{bmatrix}
    \\
    =& \sum_{i_1,i_2=1}^n v_{i_1 1}v_{i_2 2}
    \delta
    \begin{bmatrix}
      | & | & v_{13} & \ldots & v_{1n}
      \\
      e_{i_1} & e_{i_2} & v_{23} & \ldots & v_{2n}
      \\
      | & | & \vdots & \ddots & \vdots
      \\
      | & | & v_{n2} & \ldots & v_{nn}
    \end{bmatrix}
    \\
    =& \cdots
    \\
    =& \sum_{i_1,i_2, \ldots, i_n=1}^n v_{i_1 1}v_{i_2 2}
       \cdots v_{i_n n}
    \delta
    \begin{bmatrix}
      | & | & \ldots & |
      \\
      e_{i_1} & e_{i_2} & \ldots & e_{i_n}
      \\
      | & | & \ldots & |
    \end{bmatrix}
    \\
    =& \sum_{\sigma\in S_n} v_{\sigma(1),1}v_{\sigma(2),2}
       \cdots v_{\sigma(n),n}
    \delta
    \begin{bmatrix}
      | & | & \ldots & |
      \\
      e_{\sigma(1)} & e_{\sigma(2)} & \ldots & e_{\sigma(n)}
      \\
      | & | & \ldots & |
    \end{bmatrix}
    \\
    =& \sum_{\sigma\in S_n} v_{\sigma(1),1}v_{\sigma(2),2}
       \cdots v_{\sigma(n),n}
       \sgn(\sigma)
    \\
    =& \sum_{\sigma\in S_n} \sgn(\sigma)
       \prod_{i=1}^nv_{\sigma(i),i}, 
  \end{align*}
  where the first four steps follow from (SVP-3),
  the sixth step from Lemma \ref{lem:matrixPermIsSgnSigma},
  and the fifth step from \mbox{(SVP-2)}.
  The signed volume $\delta(\cdot)$ is zero for any $i_j=i_k$
   and hence the only nonzero terms are those
   of which $(i_1, i_2, \ldots, i_n)$
   is a permutation of $(1,2,\ldots,n)$.
\end{proof}

\begin{exc}
  Use the formula in (\ref{eq:determinantOfLeibniz})
   to show that $\det A = \det A^T$.
\end{exc}

\begin{defn}
  \label{def:cofactor}
  The $i, j$ \emph{cofactor} of $A\in \mathbb{R}^{n\times n}$
   is 
   \begin{equation}
     \label{eq:cofactor}
     C_{ij} = (-1)^{i+j} M_{ij},
   \end{equation}
   where $M_{ij}$ is the $i, j$ \emph{minor of a matrix} $A$,
   i.e. the determinant of the $(n-1)\times(n-1)$ matrix
   that results from deleting the $i$-th row and the $j$-th column of $A$.
\end{defn}

\begin{thm}[Laplace formula of determinants]
  \label{thm:LaplaceFormulaForDeterminants}
  Given fixed indices $i, j \in {1, 2, \ldots, n}$,
  the determinant of an $n$-by-$n$ matrix $A = [a_{ij}]$
   is given by
   \begin{equation}
     \label{eq:LaplaceFormulaForDeterminants}
     \det A = \sum_{j'=1}^n a_{ij'} C_{ij'} = \sum_{i'=1}^n a_{i'j} C_{i'j}.
   \end{equation}
\end{thm}

\begin{exc}
  Prove Theorem \ref{thm:LaplaceFormulaForDeterminants}
  by induction.
\end{exc}


%%% Local Variables:
%%% mode: latex
%%% TeX-master: "../notesNumAnal"
%%% End:

% LocalWords:  diagonalizable eigenvalue eigenvector eigenvectors
% LocalWords:  orthonormal isometry invertible semidefinite adjoint

% LocalWords:  nonnegative nilpotent diagonalized eigenspace
