% \begin{rem}
%   In this section we review major results in basic analysis
%   that are relevant to numerical analysis.
% \end{rem}

\section{Sequences}
\label{sec:sequences}

\begin{defn}
  A \emph{sequence} is a function on $\mathbb{N}$.
%   or $\mathbb{N}$.
   % or a countable set
   % $\{a_n\ |\ n\in \mathbb{N}^+\}$.
\end{defn}

\begin{rem}
  Whether the sequence starts from 0 or some $m\in \mathbb{N}^+$
   is a matter of convention and convenience
   according to the context.
\end{rem}

\begin{defn}
  The \emph{extended real number system}
  is the real line $\mathbb{R}$ with two additional elements
  $-\infty$ and $+\infty$: 
  \begin{equation}
    \label{eq:extendedReals}
    \mathbb{R}^* := \mathbb{R} \cup \{-\infty, + \infty\}.
  \end{equation}
  An extended real number $x\in \mathbb{R}^*$
  is \emph{finite} if $x\in \mathbb{R}$
  and it is \emph{infinite} otherwise.
\end{defn}

\begin{defn}
  The \emph{supremum of a sequence} $(a_n)_{n=m}^{\infty}$ is
  \begin{equation}
    \label{eq:supSequence}
    \sup(a_n)_{n=m}^{\infty}
    := \sup \{a_n: n\ge m\},
  \end{equation}
  and the \emph{infimum of a sequence} $(a_n)_{n=m}^{\infty}$ is 
  \begin{equation}
    \label{eq:infSequence}
    \inf(a_n)_{n=m}^{\infty}
    := \inf \{a_n: n\ge m\}.
  \end{equation}
\end{defn}


\subsection{Convergence and limits}
\label{sec:convergenceOfSequence}

\begin{defn}[Limit of a sequence]
  \label{def:limitOfSequence}
  A sequence $(a_n)$ in $\mathbb{R}$
  is said to \emph{converge} in $\mathbb{R}$ iff
  \begin{equation}
    \label{eq:limitOfSequence}
    \exists L\in \mathbb{R} \text{ s.t. }  
    \forall \epsilon>0,\ \exists N\in \mathbb{N}, \text{ s.t. }\ 
    \forall n>N, \ |a_n-L| < \epsilon, 
  \end{equation}
  where $L$ is called the \emph{limit} of $(a_n)$, 
  written $\lim_{n\rightarrow \infty} a_n = L$,
  or 
  \mbox{$a_n\rightarrow L \text{ as } n\rightarrow\infty$}.
  We also say that $(a_n)$ \emph{converges} to $L$.
\end{defn}

\begin{exm}[A story of $\pi$]
  A famous estimation of $\pi$ in ancient China
  is given by Zu, ChongZhi 1500 years ago,
  \begin{equation*}
    \pi\approx \frac{355}{113}\approx 3.14159292.
  \end{equation*}
  In modern mathematics,
   we approximate $\pi$ with a sequence
   for increasing accuracy, e.g. 
   \begin{equation}
     \label{eq:pi}
     \pi \approx 3.14159 26535 89793 \ldots
   \end{equation}
  As of March 2019, we human beings have more than 31 trillion digits
  of $\pi$.
  However, real world applications
  never use even a small fraction of the 31 trillion digits:
  \begin{itemize}\itemsep0em
  \item If you want to build a fence over your backyard swimming pool,
    several digits of $\pi$ is probably enough;
  \item in NASA, calculations involving $\pi$
    use 15 digits for Guidance Navigation and Control;
  \item if you want to compute the circumference of the entire universe
    to the accuracy of less than the diameter of a hydrogen atom,
    you need only 39 decimal places of $\pi$.
  \end{itemize}
  
 %  is good enough for estimating the length of the
 % perimeter of your backyard foundation
 % and many other common geometric calculations.

 On one hand, computational mathematics
 is judged by a metric that is different
 from that of pure mathematics;
 this may cause a huge gap
 between what needs to be done and what has been done.
%
On the other hand,
a computational mathematician cannot assume
that a fixed accuracy is good enough
for all applications.
In the approximation a number or a function, 
 she must develop theory and algorithms
 to provide the user the choice of an ever-increasing amount of accuracy, 
 so long as the user is willing
 to invest an increasing amount of computational resources.
This is one of the main motivations of infinite sequence and series.
\end{exm}
% We will derive the following example later this week.

% \begin{equation*}
%   \label{eq:pi1}
%   \pi = 4(1-\frac{1}{3}+\frac{1}{5}-\frac{1}{7}+ \cdots)
%       = 4\sum_{k=0}^{\infty}(-1)^k\frac{1}{2k+1}.
% \end{equation*}

\begin{lem}
  \label{lem:seqLimitIsUnique}
  A convergent sequence has a unique limit.
\end{lem}

\begin{defn}
  \label{def:CauchySequence}
  A sequence $\{a_n\}$ is \emph{Cauchy} if
  \begin{equation}
    \label{eq:CauchySequence}
    \forall \epsilon>0, \exists N\in \mathbb{N}
    \text{ s.t. } m,n>N\ \Rightarrow\ |a_n-a_m| < \epsilon.
  \end{equation}
\end{defn}

\begin{lem}
  \label{lem:convergentSeqIsCauchy}
  Every convergent sequence in $\mathbb{R}$
  is Cauchy.
\end{lem}
\begin{proof}
  Since $(x_n)$ converges to some $L\in \mathbb{R}$,
  for any given $\epsilon>0$,
  there exists $N\in \mathbb{N}$ such that
  for all $n>N$ we have $|x_n-L|<\frac{\epsilon}{2}$.
  It follows that
  \begin{align*}
    \forall n,m>N,\  |x_n-x_m|
    &\le |x_n-L+L-x_m|
    \\
    &\le |x_n-L| + |x_m-L| \le \epsilon.\qedhere
  \end{align*}
%  The proof is completed by Definition \ref{def:CauchySequence}. 
\end{proof}

\begin{lem}
  \label{lem:CauchySubseqConvergence}
  If a Cauchy sequence contains a convergent subsequence,
  then the entire sequence converges to the same limit.
\end{lem}
\begin{proof}
  Suppose $\{a_n\}$ is a Cauchy sequence
  %in a normed space $V$
  and $\{a_{n_j}\}$ is a subsequence converging to some $a\in \mathbb{R}$.
  It follows that
  \begin{displaymath}
    \begin{array}{l}
    \forall \epsilon>0, \exists n_0 \text{ s.t. }
    \forall m,n\ge n_0,\quad |a_m-a_n|\le \frac{\epsilon}{2};
    \\
    \forall \epsilon>0, \exists j_0 \text{ s.t. }
    \forall j\ge j_0,\quad |a_{n_j}-a|\le \frac{\epsilon}{2}.
    \end{array}
  \end{displaymath}
  Set $N=\max\{n_0, n_{j_0}\}$ and we have
  \begin{displaymath}
    \forall \epsilon>0, \forall n\ge N, 
    |a_{n}-a|\le |a_{n}-a_N|+ |a_{N}-a|
    \le \epsilon,
  \end{displaymath}
  which completes the proof.
\end{proof}

\begin{lem}
  \label{lem:CauchySeqIsBounded}
  Every Cauchy sequence is bounded.
\end{lem}

\begin{lem}
  \label{lem:monotoneSubsequence}
  Every real sequence has a monotone subsequence.
\end{lem}

\begin{rem}
  The idea to prove Lemma \ref{lem:monotoneSubsequence} is as follows.
  Imagine that there is an infinite chain of hotels
  along the real line,
  where the $n$th hotel has height $x_n$.
  There is a sea at $+\infty$.
  A hotel is said to have the \emph{seaview property}
  if it is higher than all hotels to the right of it.
  There are only two possibilities:
  infinitely many hotels have the seaview property
  or finitely many hotels have the seaview property.
  The former case corresponds to a monotonically decreasing sequence
  while
  the latter case corresponds to a monotonically increasing sequence.
\end{rem}

\begin{thm}
  \label{thm:boundedMonotoneSeqIsConvergent}
  A bounded monotone sequence is convergent.
\end{thm}

\begin{thm}[Bolzano-Weierstrass]
  \label{thm:Bolzano-Weierstrass}
  Every bounded sequence in $\mathbb{R}$
  has a convergent subsequence.
\end{thm}

\begin{rem}
  The Bolzano-Weierstrass theorem generalizes
  to finite-dimensional Euclidean spaces.
  Its generalization to infinite-dimensional metric spaces
  leads to the concept of sequential compactness
  in Section \ref{sec:sequentialCompactness}. 
\end{rem}

\begin{thm}%[Cauchy criterion]
  \label{thm:realCauchySeqConverges}
  Every Cauchy sequence in $\mathbb{R}$
  converges to a limit in $\mathbb{R}$.
\end{thm}
\begin{proof}
  By Lemma \ref{lem:CauchySeqIsBounded},
  the Cauchy sequence $(a_n)$ is bounded.
  Theorem \ref{thm:Bolzano-Weierstrass}
  implies that $(a_n)_{n\in \mathbb{N}}$ has a convergent
  subsequence $(a_{n_k})_{k\in \mathbb{N}}$.
  Then Lemma \ref{lem:CauchySubseqConvergence}
  completes the proof.
\end{proof}

\begin{thm}[Completeness of $\mathbb{R}$]
  \label{thm:completenessOfRealNumbers}
  A sequence of real numbers is Cauchy
  if and only if it is convergent.
\end{thm}
\begin{proof}
  This is a summary of
  Lemma \ref{lem:convergentSeqIsCauchy}
  and Theorem \ref{thm:realCauchySeqConverges}.
\end{proof}

\begin{thm}
  \label{thm:interchangeOfLimits}
  Consider $b_{ij}\ge 0$ in
  $\mathbb{R}^*:=\mathbb{R}\cup\{+\infty, -\infty\}$.
  If $b_{ij}$ is monotone increasing in $i$ for each $j$
  and is monotone increasing in $j$ for each $i$,
  then we have
  \begin{equation}
    \label{eq:interchangeOfLimits}
    \lim_{i=0}^{\infty} \lim_{j=0}^{\infty} b_{ij}
    = \lim_{j=0}^{\infty} \lim_{i=0}^{\infty} b_{ij}
  \end{equation}
  with all the indicated limits existing in $\mathbb{R}^*$.
\end{thm}


\begin{defn}
  Let $\epsilon>0$ be a real number.
  Two real numbers $x,y$ are said to be \emph{$\epsilon$-close}
  iff $|x-y|\le \epsilon$.
\end{defn}

\begin{defn}
  A real number $x$ is said to be \emph{$\epsilon$-adherent}
  to a sequence $(a_n)_{n=m}^{\infty}$ of real numbers
  iff there exists an $n\ge m$ such that
  $a_n$ is $\epsilon$-close to $x$.
  $x$ is \emph{continually $\epsilon$-adherent} to
  $(a_n)_{n=m}^{\infty}$
  iff it is $\epsilon$-adherent to $(a_n)_{n=N}^{\infty}$
  for every $N\ge m$.
\end{defn}

\begin{defn}
  \label{def:limitPointInR}
  A real number $x$ is a \emph{limit point} or \emph{adherent point}
  of a sequence $(a_n)_{n=m}^{\infty}$ of real numbers
  if it is continually $\epsilon$-adherent to
  $(a_n)_{n=m}^{\infty}$ for every $\epsilon\ge 0$.
\end{defn}

\begin{rem}
  Two special types of limit points are the limit superior
  and the limit inferior.
\end{rem}

\begin{defn}
  The \emph{limit superior} of a sequence $(a_n)_{n=m}^{\infty}$ of
  real numbers
  is
  \begin{equation}
    \label{eq:limsup}
    \lim \sup_{n\rightarrow \infty} a_n := \inf (a_N^+)_{N=m}^{\infty}, 
  \end{equation}
  where $a_N^+=\sup(a_n)_{n=N}^{\infty}$.
  The \emph{limit inferior} of $(a_n)_{n=m}^{\infty}$ is
  \begin{equation}
    \label{eq:liminf}
    \lim \inf_{n\rightarrow \infty} a_n := \sup (a_N^-)_{N=m}^{\infty}, 
  \end{equation}
  where $a_N^-=\inf(a_n)_{n=N}^{\infty}$.
\end{defn}

\begin{rem}
  Informally, $a_N^+$ and $a_N^-$ are the supremum and infimum
  of all elements in the sequence from $a_N$ onwards.
  Each of them yields a sequence with respect to $N$.
\end{rem}

\begin{exm}
  Let $(a_n)_{n=m}^{\infty}$ be the sequence
  \begin{displaymath}
    1.1, -1.01, 1.001, -1.0001, 1.00001, \ldots
  \end{displaymath}
  Then $(a_n^+)_{n=N}^{\infty}$ is the sequence
  \begin{displaymath}
    1.1, 1.001, 1.001, 1.00001, 1.00001, \ldots
  \end{displaymath}
  and $(a_n^-)_{n=N}^{\infty}$ is the sequence
  \begin{displaymath}
    -1.01, -1.01, -1.0001, -1.0001, -1.000001, -1.000001, \ldots
  \end{displaymath}
  Hence we have
  \begin{displaymath}
    \lim \sup_{n\rightarrow \infty} a_n = 1, \qquad
    \lim \inf_{n\rightarrow \infty} a_n = -1.
  \end{displaymath}
\end{exm}

\begin{lem}
  Let $(a_n)_{n=m}^{\infty}$ be a sequence of real numbers.
  For $L^+=\lim\sup a_n$ and $L^-=\lim\inf a_n$,
  we have
  \begin{enumerate}[(a)]
  \item For every $x>L^+$,
    elements of the sequence are eventually less than $x$:
    \begin{displaymath}
      \forall x>L^+, \exists N\ge m \text{ s.t. }
      \forall n\ge N, a_n<x.
    \end{displaymath}
    Similarly, for every $x<L^-$,
    elements of the sequence are eventually greater than $x$:
    \begin{displaymath}
      \forall x<L^-, \exists N\ge m \text{ s.t. }
      \forall n\ge N, a_n>x.
    \end{displaymath}
  \item For every $x<L^+$,
    there are an infinite number of elements in the sequence
    that are greater than $x$:
    \begin{displaymath}
      \forall x<L^+, \forall N\ge m,
      \exists n\ge N \text{ s.t. } a_n>x.
    \end{displaymath}
    Similarly,
    for every $x>L^-$,
    there are an infinite number of elements in the sequence
    that are less than $x$:
    \begin{displaymath}
      \forall x>L^-, \forall N\ge m,
      \exists n\ge N \text{ s.t. } a_n < x.
    \end{displaymath}
  \item $\inf(a_n)_{n=m}^{\infty}\le L^- \le L^+ \le
    \sup(a_n)_{n=m}^{\infty}$.
  \item Any limit point $c$ of $(a_n)_{n=m}^{\infty}$
    satisfies $L^-\le c \le L^+$.
  \item If $L^+$ (or $L^-$) is finite,
    then it is a limit point
    of $(a_n)_{n=m}^{\infty}$.
  \item $\lim_{n\rightarrow \infty} a_n = c$
    if and only if $L^+=L^-=c$.
    
  \end{enumerate}
\end{lem}

\begin{thm}[Squeeze test or the Sandwich Theorem]
  \label{thm:squeezeTest}
  Let $(a_n)_{n=m}^{\infty}$, $(b_n)_{n=m}^{\infty}$,
  and $(c_n)_{n=m}^{\infty}$ be sequences of real numbers
  that satisfy
  \begin{displaymath}
    \exists M\in \mathbb{N} \text{ s.t. }
    \forall n\ge M, a_n\le b_n \le c_n.
  \end{displaymath}
  Suppose $(a_n)_{n=m}^{\infty}$ and $(c_n)_{n=m}^{\infty}$
  both converge to the same limit $L$.
  Then $(b_n)_{n=m}^{\infty}$ also converges to $L$.
\end{thm}

\begin{rem}
  Let $(a_n)_{n=m}^{\infty}$ and $(b_n)_{n=m}^{\infty}$
  be two convergent sequences of real numbers
  that satisfy
  \begin{displaymath}
    \exists M\in \mathbb{N} \text{ s.t. }
    \forall n\ge M, a_n< b_n.
  \end{displaymath}
  Then we have
  $\lim_{n\rightarrow \infty} a_n\le \lim_{n\rightarrow  \infty} b_n$, 
  where ``$\le$'' cannot be strengthened to ``$<$.''
  In addition, replace ``$a_n< b_n$''
  with ``$a_n\le b_n$'' and we get the same conclusions.
  This is deeply connected to the difference between
  open and closed sets.
\end{rem}

\begin{ntn}[Asymptotic notation]
  \label{ntn:asymptoticNtn}
  For $g:\mathbb{R}^+\rightarrow\mathbb{R}^+$, 
  $f:\mathbb{R} \rightarrow\mathbb{R}$, 
  and $a\in [0,+\infty]$, 
  we write
  \begin{displaymath}
    f(x) = O(g(x)) \text{ as } x\to a %\in [0,+\infty]
  \end{displaymath}
  % iff there exist constants $C,C'>0$
  % and $x_0>0$ such that
  iff %if and only if
  \begin{displaymath}
    \lim\sup_{x\rightarrow a} \frac{|f(x)|}{g(x)} < \infty.
  \end{displaymath}
  In particular, we have
  \begin{displaymath}
    f(x) = o(g(x)) \text{ as } x\to a
    \ \Leftrightarrow\
    \lim_{x\rightarrow a} \frac{|f(x)|}{g(x)} = 0.
  \end{displaymath}
  We also write
  \begin{displaymath}
    f(x)=\Theta(g(x)) \text{ as } x\to a %\in [0,+\infty]
  \end{displaymath}
  iff
  \begin{displaymath}
    0< \lim\sup_{x\rightarrow a} \frac{|f(x)|}{g(x)} < \infty.
  \end{displaymath}
  When there is no danger of ambiguity,
  we often write $f(x)=O(g(x))$,
  $f(x)=o(g(x))$, and $f(x)=\Theta(g(x))$
  without the conditional $x \rightarrow a$
  in the above equations.
\end{ntn}

\begin{rem}
  Two typical cases of Notation \ref{ntn:asymptoticNtn}
  are $a=0$ and $a=+\infty$.
\end{rem}

\subsection{Series}
\label{sec:series}

\begin{defn}[Finite series]
  Let $m,n$ be integers
  and let $(a_i)_{i=m}^n$ be a finite sequence of real numbers.
  The \emph{finite series} or \emph{finite sum} associated with the sequence
  $(a_i)_{i=m}^n$ is the number $\sum_{i=m}^n a_i$
  given by the recursive formula
  \begin{equation}
    \label{eq:finiteSeries}
    \sum_{i=m}^n a_i :=
    \begin{cases}
      0 & \text{if } n< m;
      \\
      a_n + \sum_{i=m}^{n-1} a_i & \text{otherwise}.
    \end{cases}
  \end{equation}
\end{defn}

\begin{defn}[Formal infinite series]
  \label{def:seriesFromSequence}
  A (formal) \emph{infinite series} associated with an infinite sequence $\{a_n\}$
   is the expression $\sum_{n=0}^{\infty} a_n$.
%   the sum of all terms of the sequence.
\end{defn}

\begin{defn}
  The \emph{sequence of partial sums} $(S_n)_{n=0}^{\infty}$ associated
  with a formal infinite series
  $\sum_{i=0}^{\infty} a_i$
  is defined for each $n$ as the sum of the sequence $\{a_i\}$
  from $a_0$ to $a_n$
  \begin{equation}
    \label{eq:partialSum}
    S_n = \sum_{i=0}^{n} a_i.
  \end{equation}
\end{defn}

\begin{defn}
  A formal infinite series is said to be \emph{convergent}
  and \emph{converge} to $L$
  if its sequence of partial sums converges to some limit $L$.
  In this case we write $L=\sum_{n=0}^{\infty} a_n$
  and call $L$ the \emph{sum of the infinite series}.
\end{defn}

\begin{defn}
  A formal infinite series is said to be \emph{divergent}
  if its sequence of partial sums diverges.
  In this case we do not assign any real number value to this series.
\end{defn}

\begin{lem}[Cauchy criterion]
  \label{lem:seriesConvergeCauchy}
  An infinite series $\sum_{n=0}^{\infty} a_n$ of real numbers
  is convergent if and only if
  \begin{equation}
    \label{eq:seriesConvergeCauchy}
    \forall \epsilon>0,\ \exists N\in \mathbb{N} \text{ s.t. }
    \forall p,q\ge N, \ \left|\sum_{n=p}^q a_n\right| \le \epsilon.
  \end{equation}
\end{lem}

\begin{thm}[Comparison test]
  \label{thm:comparisonTest}
  Let $\sum a_n$ be a series where $a_n\ge 0$ for all $n$.
  \begin{enumerate}[(i)]\itemsep0em
  \item If $\sum a_n$ converges and $|b_n|\le a_n$ for all $n$,
    then $\sum b_n$ converges.
  \item If $\sum a_n=+\infty$ and $b_n \ge a_n$ for all $n$,
    then \mbox{$\sum b_n=+\infty$}.
  \end{enumerate}
\end{thm}

\begin{defn}
  \label{def:absoluteConvergence}
  An infinite series $\sum_{n=0}^{\infty} a_n$
  is \emph{absolutely convergent}
  iff the series $\sum_{n=0}^{\infty} |a_n|$ is convergent.
\end{defn}

\begin{lem}
  \label{lem:absolutelyConvergenceSeriesConverge}
  An infinite series that is absolutely convergent
  is convergent.
\end{lem}

\begin{thm}[Ratio test]
  \label{thm:ratioTest}
  A series $\sum a_n$ of nonzero terms
  \begin{enumerate}[(i)]\itemsep0em
  \item converges absolutely if ${\lim\sup}_{n\rightarrow\infty}|a_{n+1}/a_n|<1$;
  \item diverges if ${\lim\inf}_{n\rightarrow\infty}|a_{n+1}/a_n|>1$,
  \end{enumerate}
  Otherwise, this test gives no information
  about the convergence of $\sum a_n$.
\end{thm}

\begin{thm}[Root test]
  \label{thm:rootTest}
  Let $\sum a_n$ be a series and $\alpha={\lim\sup}_{n\rightarrow\infty}|a_n|^{1/n}$.
  The series $\sum a_n$
  \begin{enumerate}[(i)]\itemsep0em
  \item converges absolutely if $\alpha<1$;
  \item diverges if $\alpha>1$,
  \end{enumerate}
  Otherwise, this test gives no information
  about the convergence of $\sum a_n$.
\end{thm}
\begin{proof}
  Choose $\epsilon>0$ such that $\alpha+\epsilon\in (0,1)$.
  Then 
  \begin{displaymath}
    \exists N\in \mathbb{N}^+ \text{ s.t. }
    \forall n\ge N, \ \sup_{n\ge N}\{ a_n^{\frac{1}{n}}\} < \alpha+\epsilon.
  \end{displaymath}
  Hence $a_n<(\alpha+\epsilon)^n$ for all $n\ge N$.
  The comparison test with the geometric series yields (i).
  (ii) can be shown via proof by contradiction.
\end{proof}

\begin{rem}
  The root test applies whenever the ratio test applies.
  However, the converse is not true.
\end{rem}

\begin{thm}[Integral test]
  \label{thm:integralTest}
  Let $f:[0, \infty)\rightarrow \mathbb{R}$
   be a monotone decreasing function 
   which is non-negative.
%   such that $f(x)\ge 0$ for all $x\ge 0$.
  Then the series $\sum_{n=0}^{\infty}f(n)$ is convergent if and only if
   $\sup_{N>0}\int_0^N f$ is finite.
\end{thm}


\section{Scalar functions
  $\mathbb{R}\rightarrow\mathbb{R}$}
\label{sec:cont-funct-on-R}

\begin{defn}
  A \emph{scalar function} is a function
   whose range is a subset of $\mathbb{R}$.
\end{defn}

\begin{defn}[Limit of a scalar function]
  \label{def:limitOfAScalarFunc}
  Consider a function $f: I\rightarrow \mathbb{R}$
   with $I(c,r)=(c-r,c)\cup (c,c+r)$.
  The \emph{limit} of $f(x)$  exists
   as $x$ approaches $c$, written
     $\lim_{x\rightarrow c} f(x) = L$,
  iff 
  \begin{equation}
 \forall \epsilon>0, \exists \delta>0, \text{ s.t. }
    \forall x\in I(c,\delta),\ |f(x)-L|<\epsilon.
  \end{equation}
\end{defn}

\begin{rem}
  The notation in Definition \ref{def:limitOfAScalarFunc}
  reads ``as $x$ gets closer to $c$,
 $f(x)$ gets closer to $L$.''
How close is close?
As close as you wish.
This idea is packaged in the $\epsilon-\delta$ technique.
\end{rem}

\begin{exm}
  \label{exm:limitForShowingEpsDelta}
   We show $\lim_{x\rightarrow 2}\frac{1}{x}= \frac{1}{2}$ as follows.
   If $\epsilon\ge \frac{1}{2}$,
   choose $\delta=1$.
   Then $x\in(1,3)$ implies
   $\left|\frac{1}{x}-\frac{1}{2}\right|<\frac{1}{2}$
   since $\frac{1}{x}-\frac{1}{2}$
   is a monotonically decreasing function
   with its supremum at $x=1$.

  If $\epsilon\in (0,\frac{1}{2})$,
   choose $\delta=\epsilon$.
  Then $x\in (2-\epsilon,2+\epsilon)\subset
  (\frac{3}{2},\frac{5}{2})$.
  Hence
  $\left|\frac{1}{x}-\frac{1}{2}\right|=\frac{|2-x|}{|2x|}<|2-x|<\epsilon$.
  The proof is completed by Definition \ref{def:limitOfAScalarFunc}.
\end{exm}

\begin{rem}
  The philosophy in Example \ref{exm:limitForShowingEpsDelta}
  is that two functions have the same limit
  if their difference can be shown to be as small as you wish.
\end{rem}

% \begin{defn}
%   The \emph{open ball} centered at $P_0\in \mathbb{R}^n$
%    with radius $r>0$
%    is the point set
%    \begin{equation}
%      \label{eq:unitBall}
%      {\cal B}(P_0, r) = 
%      \left\{P \ \bigl|\ |P-P_0| < r\bigr.\right\}.
%    \end{equation}
%   It is an \emph{open interval} in 1D
%    and an \emph{open disk} in 2D.

%   The open ball without the center is denoted by
%    \begin{equation}
%      {\cal B}_0(P_0, r) = {\cal B}(P_0, r) \setminus
%      \{P_0\}.
%    \end{equation}
% \end{defn}

% \begin{defn}
%   \label{def:extremum}
%   $f: \mathbb{R}^n\rightarrow\mathbb{R}$ has a
%   \emph{local maximum} at $P_0\in \mathbb{R}^n$
%   iff
%   \begin{equation}
%     \label{eq:localMax}
%     \exists r>0, \text{ s.t. } \forall P \in{\cal B}(P_0, r),
%     \qquad f(P)\le f(P_0).
%   \end{equation}
%   Changing $\le$ to $\ge$ in (\ref{eq:localMax})
%    yields a \emph{local minimum}.\\
%   An \emph{extremum} is either a maximum or minimum.
% \end{defn}

% \begin{defn}
%   $P_0$ is
%   a \emph{boundary point} of a point set ${\cal U}$
%   iff $\forall r>0$, $\exists P\in {\cal B}(P_0,r)$
%   s.t. $P\not\in {\cal U}$.
%  \end{defn}

%  \begin{defn}
%   A set is an \emph{open set} if it contains
%    none of its boundary points.
%  \end{defn}

%  \begin{defn}
%    A set is a \emph{closed set} if it contains
%     all of its boundary points.
%  \end{defn}

%  \begin{defn}
%    A point set ${\cal U}\subseteq \mathbb{R}^n$
%     is \emph{bounded}
%     iff ${\cal U}\subseteq {\cal B}(P_0, r)$
%     for some $P_0\in \mathbb{R}^n$ and $r>0$.
%  \end{defn}

% \begin{defn}[Limit of a scalar function with multiple variables]
% \label{def:limit}
%   The \emph{limit} of a function
%   $f: {\cal B}_0(P_0,r)\rightarrow \mathbb{R}$
%   exists
%   as $P$ approaches $P_0$, written
% $     \lim_{P\rightarrow P_0} f(P) = L$,
%   iff
%    \begin{equation}
%      \begin{split}
%    \forall \epsilon>0,\ 
%    \forall \text{ paths } P\rightarrow P_0,\ 
%    \exists \delta>0, \text{ s.t. }\\
%    \forall P\in {\cal B}(P_0,\delta),\qquad
%    |f(P)-L|<\epsilon.
%      \end{split}
%    \end{equation}
% \end{defn}

% \begin{frm}
%   The limit of $f: {\cal B}_0(P_0,r)\rightarrow \mathbb{R}$
%    does not exist at $P_0$
%    if there exists two different paths $P\rightarrow P_0$
%    and $P \leadsto P_0$ s.t.
%    \begin{equation}
%      \bigl(\lim_{P\rightarrow P_0} f(P) = L_1\bigr)
%      \ne \bigl(L_2 = \lim_{P\leadsto P_0} f(P)\bigr).
%    \end{equation}
% \end{frm}

% \begin{thm}[The squeeze theorem]
%   If $\exists r>0$ s.t.\\
%   $\forall P\in{\cal B}_0(P_0,r)$,
%    $f(P)\le g(P)\le h(P)$,
%    then
%    \begin{equation}
%      \left(\lim_{P\rightarrow P_0}f(P)=\lim_{P\rightarrow P_0}h(P)=L\right)
%      \ \Rightarrow \ 
%      \lim_{P\rightarrow P_0} g(P) = L.
%    \end{equation}
% \end{thm}

% \begin{defn}
%   \emph{Big O notation} describes the limiting behavior
%   of a function
%   in terms of another function.
%   Given  $f,g: \mathbb{R}\rightarrow \mathbb{R}$,
%   \begin{subequations}
%     \begin{align}
%       f(h) = O(g(h))\ &\Leftrightarrow\ 
%       \lim_{h\rightarrow 0}\frac{f(h)}{g(h)} = L \ne 0
%       \\
%       f(h) = o(g(h))\ &\Leftrightarrow\ 
%       \lim_{h\rightarrow 0}\frac{f(h)}{g(h)} = 0
%     \end{align}
%   \end{subequations}
% \end{defn}

% \begin{exm}
%   We will be dealing with polynomial of $h$
%   and determining convergence is very easy in this case.
% \end{exm}

\subsection{Continuous scalar functions}
\label{sec:cont-scal-funct}

\begin{defn}
  \label{def:continuousScalarFunc}
  $f: \mathbb{R}\rightarrow \mathbb{R}$
  is \emph{continuous} at $c$
  iff
   \begin{equation}
     \label{eq:continuous}
     \lim_{x\rightarrow c} f(x) = f(c).
   \end{equation}
\end{defn}

\begin{defn}
  \label{def:continuousFuncOnR}
  A scalar function $f$ is \emph{continuous on} $(a,b)$,
     written \mbox{$f\in{\cal C}(a,b)$}, 
     if (\ref{eq:continuous}) holds  $\forall x\in (a,b)$.
\end{defn}

% \begin{defn}
%   $f: \mathbb{R}^n\rightarrow \mathbb{R}$
%   is \emph{continuous} at $Q$
%   iff
%    \begin{equation}
%      \label{eq:continuous}
%      \lim_{P\rightarrow Q} f(P) = f(Q).
%    \end{equation}
%    $f$ is \emph{continuous on} a point set ${\cal U}$
%     if (\ref{eq:continuous}) holds  $\forall Q\in {\cal U}$.
% \end{defn}

\begin{thm}[Extreme values]
  \label{thm:extremeValues}
  A continuous function $f:[a,b]\rightarrow\mathbb{R}$
  attains its maximum at some point \mbox{$x_{\max}\in [a,b]$}
  and its minimum at some point $x_{\min}\in [a,b]$.
\end{thm}

\begin{thm}[Intermediate value]
  \label{thm:intermediateValue}
  A scalar function $f\in {\cal C}[a,b]$ satisfies
  \begin{equation}
    \label{eq:intermediateValue}
    \forall y\in \left[m, M \right],\ 
    \exists \xi\in[a,b], \text{ s.t. }
    y=f(\xi)
  \end{equation}
  where  $m=\inf_{x\in[a,b]} f(x)$ and
   $M=\sup_{x\in[a,b]} f(x)$.
\end{thm}

\begin{rem}
Theorem \ref{thm:intermediateValue}
 states that a continuous function assumes
 all values between $f(a)$ and $f(b)$
 on a closed interval $[a,b]$.
\end{rem}

\begin{defn}
  \label{def:uniformlyContinuousScalar}
  Let $I=(a,b)$. A function $f: I\rightarrow \mathbb{R}$
  is \emph{uniformly continuous} on $I$
  iff
   \begin{equation}
     \label{eq:uniformlyContinuous}
     \begin{array}{l}
     \forall \epsilon>0, \exists \delta>0,\text{ s.t. }
     \\
     \forall x,y\in I,\ 
     |x-y|<\delta \Rightarrow |f(x)-f(y)|<\epsilon.
     \end{array}
   \end{equation}
\end{defn}

\begin{exc}
  Show that, on $(a,\infty)$,
  $f(x)=\frac{1}{x}$ is uniformly continuous
   if $a>0$ 
   and is not so if $a=0$.
\end{exc}
\begin{proof}
  If $a>0$, then
  $|f(x)-f(y)|=\frac{|x-y|}{xy} %< \frac{\delta}{xy}\
  <\frac{|x-y|}{a^2}$.\\
  Hence 
  $\forall \epsilon>0, \exists \delta=a^2 \epsilon$,
  s.t. \\$|x-y|<\delta \Rightarrow
  |f(x)-f(y)|<\frac{|x-y|}{a^2}<\frac{a^2 \epsilon}{a^2}=\epsilon$.

  If $a=0$, negating the condition
   of uniform continuity,
   i.e. eq. (\ref{eq:uniformlyContinuous}), yields
  $\exists \epsilon>0$ s.t. $\forall\delta>0$
   $\exists x,y>0$ s.t.
   $(|x-y|<\delta) \wedge (|\frac{1}{x}-\frac{1}{y}|\ge \epsilon)$.

   We prove a stronger version:
  $\forall \epsilon>0$, $\forall\delta>0$
   $\exists x,y>0$ s.t.
   $(|x-y|<\delta) \wedge (|f(x)-f(y)|\ge \epsilon)$.

%  Set $\epsilon=\frac{1}{3}$.
  If $\delta\ge \frac{1}{2\epsilon}$,
   choose $x=\frac{1}{2\epsilon}$, $y=\frac{1}{4\epsilon}$.
  This choice satisfies $|x-y|<\delta$
   since $x-y=\frac{1}{4\epsilon}<\frac{1}{2\epsilon}\le \delta$.
  However, $|f(x)-f(y)|=\frac{|x-y|}{xy}=2\epsilon>\epsilon$.

  If $\delta< \frac{1}{2\epsilon}$,
   then $2\epsilon\delta<1$.
  Choose $x\in (0, \epsilon\delta^2)$
   and $y\in (2\epsilon\delta^2, \delta)$.
  This choice satisfies $|x-y|<\delta$
   and $|x-y|>\epsilon\delta^2$.
  However, 
   $|f(x)-f(y)|=\frac{|x-y|}{xy}>\frac{\epsilon\delta^2}{xy}
   > \frac{1}{y} > \frac{1}{\delta} > 2\epsilon> \epsilon$.
\end{proof}

\begin{exc}
  On $(a,\infty)$,
  $f(x)=\frac{1}{x^2}$ is uniformly continuous
   if $a>0$ 
   and is not so if $a=0$.
\end{exc}
\begin{solution}
  Case 1 ($a>0$):
  The function $f(x)=\frac{1}{x^2}$ is uniformly continuous
  on $(a,\infty)$ iff
  \begin{align*}
    \forall\epsilon>0,&\exists\delta>0\text{ s.t. }
                        \forall x,y \in(a,\infty),\\
    &|x-y|<\delta\Rightarrow|f(x)-f(y)|<\epsilon.
  \end{align*}

  Since $a>0$ and $x,y \in (a,\infty)$,
  we have $|x|=x>0$ and $|y|=y>0$.  Thus,
\[
|f(x)-f(y)|
=|x-y|\left(\frac{1}{x^2y}+\frac{1}{xy^2}\right).
\]

For any given $\epsilon>0$,
choose $\delta=\frac{a^3\epsilon}{2}$.
Then we have
\begin{displaymath}
  x>a\Rightarrow\frac{1}{x}<\frac{1}{a};\ \ 
  y>a\Rightarrow\frac{1}{y}<\frac{1}{a}.
\end{displaymath}

It follows that
\[
 |x-y|<\delta \ \Rightarrow\ 
|f(x)-f(y)|
<\delta\left(\frac{1}{a^3}+\frac{1}{a^3}\right)
=\frac{2\delta}{a^3}=\epsilon.
\]
Hence $f(x)=\frac{1}{x^2}$ is uniformly continuous on $(a,\infty)$ if
$a>0$.

Case 2 ($a=0$):
The function $f(x)=\frac{1}{x^2}$ is not uniformly continuous
on $(0,\infty)$ iff
\begin{align*}
  \exists\epsilon>0\text{ s.t. } & \forall\delta>0,
  \exists x,y \in (0,\infty)\text{ s.t. }\\
  &|x-y|<\delta\Rightarrow |f(x)-f(y)|\geq\epsilon.
\end{align*}

We prove a stronger statement:
\begin{align*}
  \forall\epsilon>0 & \forall\delta>0,\exists x,y \in (0,\infty)
  \text{ s.t. } \\
  &|x-y|<\delta\Rightarrow |f(x)-f(y)|\geq\epsilon.
\end{align*}

If $\delta\geq\frac{1}{2\sqrt{\epsilon}}$,
 choose
 $x=\frac{1}{\sqrt{4\epsilon}}=\frac{1}{2\sqrt{\epsilon}}$ and $y=\frac{1}{\sqrt{9\epsilon}}=\frac{1}{3\sqrt{\epsilon}}$.
This choice satisfies $|x-y|< \delta$.
However,
\begin{align*}
  |f(x)-f(y)| &=
                \frac{|x-y|}{xy}\left(\frac{1}{x}+\frac{1}{y}\right)\\
 &= \frac{\frac{1}{6\sqrt{\epsilon}}}{\left(\frac{1}{2\sqrt{\epsilon}}\right)\left(\frac{1}{3\sqrt{\epsilon}}\right)}(2\sqrt{\epsilon}+3\sqrt{\epsilon})
  \\
&= \frac{1}{\sqrt{\epsilon}}\,\epsilon(5\sqrt{\epsilon})
= 5\epsilon >\epsilon
\end{align*}

If $\delta<\frac{1}{2\sqrt{\epsilon}}$,
 choose $x$ and $y$ so that $0<x<\sqrt{\epsilon}\delta^2$
 and $2\sqrt{\epsilon}\delta^2<y<\delta$.
Then $0<x<y<\delta$ implies $\frac{1}{x^2y^2}>\frac{1}{\delta^4}$
 and $x+y> 2\sqrt{\epsilon}\delta^2$.
Also, $2\sqrt{\epsilon}\delta<1$ and
 $2\sqrt{\epsilon}\delta^2-\sqrt{\epsilon}\delta^2<|x-y|<\delta \Rightarrow \sqrt{\epsilon}\delta^2<|x-y|<\delta$.
Altogether, these imply
\begin{align*}
|f(x)-f(y)| = \frac{|x-y|(x+y)}{x^2y^2}
> \frac{\sqrt{\epsilon}\delta^2 2\sqrt{\epsilon}\delta^2}{\delta^4}
=2\epsilon>\epsilon.
\end{align*}
Hence $f(x)=\frac{1}{x^2}$ is not uniformly continuous on $(a,\infty)$ if
$a=0$.
\end{solution}

\begin{thm}
  Uniform continuity implies continuity
   but the converse is not true.
\end{thm}
\begin{proof}
  exercise.
\end{proof}

\begin{lem}
  \label{lem:uniformContExtension}
  $f: \mathbb{R}\rightarrow \mathbb{R}$
  is uniformly continuous on $(a,b)$
  iff it can be extended
  to a continuous function $\tilde{f}$ on $[a,b]$.
\end{lem}

\begin{thm}[Uniform continuity]
  \label{thm:uniformContinuityScalarFunc}
  A continuous function $f: [a,b]\rightarrow \mathbb{R}$
  is uniformly continuous.
\end{thm}


\subsection{Differentiation of scalar functions}
\label{sec:diff-funct}

\begin{defn}
  \label{def:derivative}
  The \emph{derivative}
   of a function $f: \mathbb{R}\rightarrow \mathbb{R}$
   at $a$ is the limit
   \begin{equation}
     \label{eq:derivative}
     f'(a)=\lim_{h\rightarrow 0} \frac{f(a+h)-f(a)}{h}.
   \end{equation}
  If the limit exists, $f$ is \emph{differentiable} at $a$.
\end{defn}

\begin{exm}
  For the power function \mbox{$f(x)=x^{\alpha}$}, 
   we have $f'=\alpha x^{\alpha-1}$
   due to
   Newton's generalized binomial theorem, 
  \begin{equation*}
    (a+h)^{\alpha}=\sum_{n=0}^{\infty}{\alpha \choose n}a^{\alpha-n}h^n.
  \end{equation*}
\end{exm}

\begin{rem}
  Think about the three cases $\alpha=\frac{1}{2}$,
  $\alpha=1$, $\alpha=2$.
  If $x$ is time and $f(x)$ measures how knowledgeable you are.
  You probably want to have $f(x)=x^2$ rather than
  $f(x)=x^{\frac{1}{2}}$.
  The reason $f(x)=x^2$ is much better is that
  its rate of increase also increases.
\end{rem}

\begin{defn}
  A function $f(x)$ is
   $k$ times \emph{continuously differentiable}
   on $(a,b)$
   iff $f^{(k)}(x)$ exists on $(a,b)$ and is itself continuous.
  The set or space of all such functions on $(a,b)$
   is denoted by ${\cal C}^k(a,b)$.
  In comparison,
   ${\cal C}^k[a,b]$ is the space of functions $f$
   for which $f^{(k)}(x)$ is bounded and uniformly continuous on $(a,b)$.
\end{defn}

\begin{thm}
  \label{thm:continuousImpliesBoundedness}
  A scalar function $f$ is bounded on $[a,b]$
   if $f\in {\cal C}[a,b]$.
\end{thm}

\begin{thm}
  \label{thm:Fermat}
  If $f:(a,b)\rightarrow\mathbb{R}$
   assumes its maximum or minimum at $x_0\in(a,b)$
   and $f$ is differentiable at $x_0$,
   then $f'(x_0)=0$.
\end{thm}
\begin{proof}
  Suppose $f'(x_0)>0$. Then we have
  \begin{displaymath}
    f'(x_0)=\lim_{x\rightarrow x_0} \frac{f(x)-f(x_0)}{x-x_0}>0.
  \end{displaymath}
  The definition of a limit implies
  \begin{displaymath}
   \exists \delta>0 \text{ s.t. } 
   a<x_0-\delta<x_0+\delta<b,
  \end{displaymath}
   which, together with $|x-x_0|<\delta$, 
   implies
   $\frac{f(x)-f(x_0)}{x-x_0}>0$.
  This is a contradiction to $f(x_0)$ being a maximum
  when we choose $x\in(x_0,x_0+\delta)$.
\end{proof}

\begin{thm}[Rolle's]
  \label{thm:Rolles}
  If a function $f:\mathbb{R}\rightarrow\mathbb{R}$
   satisfies
   \begin{enumerate}[(i)]\itemsep0em
   \item $f\in {\cal C}[a,b]$ and $f'$ exists on $(a,b)$,
   \item $f(a)=f(b)$,
   \end{enumerate}
  then $\exists x\in (a,b)$ s.t. $f'(x)=0$.
\end{thm}
\begin{proof}
  % Theorem 18.1 in Ross states that
  %  if $f$ is continuous on a closed interval,
  %  then $f$ is bounded
  %  and assumes its maximum and minimum values
  %  on $[a,b]$.
  % By Theorem \ref{thm:continuousImpliesBoundedness},
  %  $f$ is bounded.
  By Theorem \ref{thm:intermediateValue},
   all values between $\sup f$ and $\inf f$
   will be assumed.
  If $f(a)=f(b)=\sup f=\inf f$,
   then $f$ is a constant on $[a,b]$
   and thus the conclusion holds.
  Otherwise, Theorem \ref{thm:Fermat} completes the proof.
\end{proof}

\begin{rem}
Theorem \ref{thm:intermediateValue} is about a closed interval
 and Theorem \ref{thm:Fermat} an open interval.
Thus in the proof of Theorem \ref{thm:Rolles}
 we must treat the special case of $f$ being a constant.
\end{rem}

\begin{thm}[Mean value]
  \label{thm:meanValue}
  If $f\in {\cal C}[a,b]$ and if $f'$ exists on $(a,b)$,
  then $\exists \xi\in(a,b)$
  s.t. $f(b)-f(a)=f'(\xi)(b-a)$.
\end{thm}
\begin{proof}
  Construct a linear function
   $L:[a,b]\rightarrow\mathbb{R}$ such that
   $L(a)=f(a)$, $L(b)=f(b)$,
   then $\forall x\in (a,b)$,
   we have $L'(x)=\frac{f(b)-f(a)}{b-a}$.
  Consider $g(x)=f(x)-L(x)$ on $[a,b]$.
   $g(a)=0$, $g(b)=0$.
  By Theorem \ref{thm:Rolles}, $\exists \xi\in[a,b]$
   such that $g'(\xi)=0$,
   which completes the proof.
\end{proof}


\subsection{Taylor series}
\label{sec:Taylorseries}

\begin{defn}
  \label{def:powerSeries}
  A \emph{power series} centered at $c$
   is a series of the form
   \begin{equation}
     \label{eq:powerSeries}
     p(x)=\sum_{n=0}^{\infty} a_n (x-c)^n,
   \end{equation}
   where $a_n$'s are the \emph{coefficients}.
  The \emph{interval of convergence}
   is the set of values of $x$ for which the series converges:
   \begin{equation}
     \label{eq:intervalOfConvergence}
     I_c(p) = \{x\ |\ p(x) \text{ converges} \}.
   \end{equation}
\end{defn}

\begin{defn}
  \label{def:TaylorPolynomial}
  If the derivatives $f^{(i)}(x)$ with $i=1,2,\ldots,n$ exist for a function
  $f: \mathbb{R}\rightarrow \mathbb{R}$ at $x=c$,
   then
   \begin{equation}
     \label{eq:partialSumTaylor}
     T_n(x) = \sum_{k=0}^{n}\frac{f^{(k)}(c)}{k!}(x-c)^k
   \end{equation}
   is called the $n$th \emph{Taylor polynomial} for $f(x)$ at $c$.
\\  In particular, 
the \emph{linear approximation} for $f(x)$ at $c$ is
   \begin{equation}
     \label{eq:linearTaylor}
     T_1(x) = f(c) + f'(c)(x-c).
   \end{equation}
 \end{defn}

 \begin{exm}
   \label{exm:TaylorPoly}
  If $f\in {\cal C}^{\infty}$, then $\forall n\in\mathbb{N}$,
   we have
  \begin{equation*}
    T_n^{(m)}(x) = \left\{
    \begin{array}{ll}
      \sum_{k=m}^n\frac{f^{(k)}(c)}{(k-m)!}(x-c)^{k-m}, &
      m\in\mathbb{N}, m\le n;
      \\
      0, & m\in\mathbb{N}, m> n.
    \end{array}
    \right.
  \end{equation*}
This can be proved by induction.
In the inductive step, we regroup the summation
 into a constant term and another shifted summation.
\end{exm}

\begin{defn}
  \label{def:TaylorSeries}
The \emph{Taylor series} (or Taylor expansion) 
   for $f(x)$ at $c$ is
 \begin{equation}
     \label{eq:TaylorSeries}
     %\lim_{n\rightarrow\infty}T_n(x)= 
     \sum_{k=0}^{\infty}\frac{f^{(k)}(c)}{k!}(x-c)^k.
   \end{equation}
\end{defn}

\begin{defn}
The \emph{remainder} of the $n$th \emph{Taylor polynomial}
 in approximating $f(x)$ is
  \begin{equation}
    \label{eq:TaylorSeriesRemainder}
    E_n(x) = f(x) - T_n(x).
  \end{equation}
\end{defn}

\begin{thm}
  \label{thm:limitTaylorRemainder}
  Let $T_n$ be the $n$th Taylor polynomial for $f(x)$ at $c$.
  \begin{equation}
    \lim_{n\rightarrow\infty} E_n(x) = 0
    \ \Leftrightarrow\
%    f(x) = \sum_{k=0}^{\infty}\frac{f^{(k)}(a)}{k!}(x-a)^k
    \lim_{n\rightarrow\infty}T_n(x) = f(x).
  \end{equation}
\end{thm}

\begin{lem}
  \label{lem:remainder}
  $\forall m=0,1,2,\ldots,n$, $E_n^{(m)}(c)=0$.
\end{lem}
\begin{proof}
  This follows from Definition \ref{def:TaylorPolynomial}
   and Example \ref{exm:TaylorPoly}.
\end{proof}

 \begin{thm}[Taylor's theorem with Lagrangian form]
   \label{thm:TaylorLagrangianForm}
   Consider a function
    $f: \mathbb{R} \rightarrow \mathbb{R}$.
   If $f\in {\cal C}^n[c-d,c+d]$
    and $f^{(n+1)}(x)$ exists on $(c-d, c+d)$,
   then
    $\forall x\in [c-d,c+d]$,
    there exists some $\xi$ between $c$ and $x$
    such that
    \begin{equation}
      \label{eq:TaylorTheorem}
      E_n(x) = \frac{f^{(n+1)}(\xi)}{(n+1)!}(x-c)^{n+1}.
    \end{equation}
 \end{thm}
 \begin{proof}
   Fix $x\ne c$,
    let $M$ be the unique solution of 
   \begin{equation*}
%     \label{eq:Mequality}
     E_n(x) = f(x) - T_n(x) = \frac{M(x-c)^{n+1}}{(n+1)!}.
   \end{equation*}
   Consider the function
   \begin{equation}
     \label{eq:TaylorLagrangianFormProof1}
     g(t) := E_n(t) - \frac{M(t-c)^{n+1}}{(n+1)!}.
   \end{equation}
   Clearly $g(x)=0$.
   By Lemma \ref{lem:remainder},
    $g^{(k)}(c)=0$ for each $k= 0,1,\ldots,n$.
   Then Rolle's theorem implies that
   \begin{equation*}
    \exists x_1\in (c, x) \text{ s.t. } g'(x_1)=0. 
   \end{equation*}
   If $x<c$, change $(c,x)$ above to $(x,c)$.
   Apply Rolle's theorem to $g'(t)$ on $(c,x_1)$
   and we have
   \begin{equation*}
    \exists x_2\in (c, x_1) \text{ s.t. } g^{(2)}(x_2)=0. 
   \end{equation*}
   Repeatedly using Rolle's theorem,
   \begin{equation}
     \label{eq:TaylorLagrangianFormProof2}
    \exists x_{n+1}\in (c, x_n) \text{ s.t. } g^{(n+1)}(x_{n+1})=0.
  \end{equation}
  Since $T_n$ is a polynomial of degree $n$,
   we have $T_n^{(n+1)}(t)=0$, 
   which, together with
   (\ref{eq:TaylorLagrangianFormProof2})
   and (\ref{eq:TaylorLagrangianFormProof1}),
   yields 
   \begin{equation*}
     f^{(n+1)}(x_{n+1})-M=0.
   \end{equation*}
   The proof is completed
    by identifying $\xi$ with $x_{n+1}$.
\end{proof}

\begin{exm}
  \label{exm:approximatingExponentialFunction}
  How many terms are needed to compute $e^2$ correctly to four decimal
  places?

  The requirement of four decimal places means
  an accuracy of at least $\epsilon=10^{-5}$.
By Definition \ref{def:TaylorSeries}, 
   the Taylor series of $e^x$ at $c=0$ is
  \begin{equation*}
    e^x = \sum_{n=0}^{+\infty}\frac{x^n}{n!}.
  \end{equation*}
  By Theorem \ref{thm:TaylorLagrangianForm}, we have
  \begin{equation*}
    \exists \xi\in [0,2] \text{ s.t. }
    E_n(2)=e^{\xi}2^{n+1}/(n+1)! < e^22^{n+1}/(n+1)!
  \end{equation*}
  Then $e^22^{n+1}/(n+1)!\le \epsilon$ yields $n\ge 12$, i.e., 13 terms.
\end{exm}


\subsection{Riemann integral}
\label{sec:RiemannIntegral}

\begin{defn}
  \label{def:partitionOfInterval}
  A \emph{partition of an interval} $I=[a, b]$
  is a totally-ordered finite subset $P_n\subseteq I$ of the form
  %a finite sequence of numbers of the form
  \begin{equation}
    \label{eq:partition}
    P_n(a,b) = \{a=x_0 < x_1 < \cdots < x_n=b\}.
  \end{equation}
  The interval $I_i=[x_{i-1}, x_{i}]$ is the $i$th
   \emph{subinterval} of the partition.
%  The \emph{mesh} or 
  The \emph{norm} of the partition
  is the length of the longest subinterval,
  \begin{equation}
    h_n= h(P_n) = \max(x_i-x_{i-1}),\qquad i=1, 2, \ldots, n.
  \end{equation}
\end{defn}

\begin{defn}
  \label{def:RiemannSum}
  The \emph{Riemann sum} of %a function
  $f: \mathbb{R}\rightarrow \mathbb{R}$
  over a partition $P_n$ is
  \begin{equation}
    \label{eq:RiemannSum}
    S_n(f) = \sum_{i=1}^n f(x_i^*) (x_{i}-x_{i-1}),
  \end{equation}
  where $x_i^*\in I_i$ 
%  $x_i^*\in [x_{i-1}, x_i]$ 
  is a \emph{sample point} % or a \emph{tag}
  of the $i$th subinterval.
  In particular, the \emph{upper Riemann sum}
  and \emph{lower Riemann sum} over $P_n$ are
  \begin{align}
    \label{eq:upperRiemannSum}
    M_n(f) = \sum_{i=1}^n (x_{i}-x_{i-1}) \sup f(I_i), 
    \\
    \label{eq:lowerRiemannSum}
    m_n(f) = \sum_{i=1}^n (x_{i}-x_{i-1}) \inf f(I_i). 
  \end{align}
\end{defn}

\begin{defn}
  \label{def:RiemannIntegrable}
  A function $f: \mathbb{R}\rightarrow \mathbb{R}$
  is  %\emph{integrable} (or more precisely
  \emph{Riemann integrable} on  $[a, b]$
  iff %there exists $L\in \mathbb{R}$ such that
% the limit of its Riemann sum exists as $n\rightarrow \infty$
  \begin{align}
    \nonumber
    &\exists L\in\mathbb{R}, \text{ s.t. }
      \forall \epsilon>0,\ \exists \delta>0 \text{ s.t. }
    \\ \label{eq:RiemannIntegrable}
    & \forall P_n(a,b) \text{ with } h(P_n)<\delta,\ 
      |S_n(f)-L|<\epsilon.
  \end{align}
  In this case we write $L=\int_a^b f(x) \dif x$
  and call it the \emph{Riemann integral} of $f$ on $[a,b]$.
\end{defn}

  % Bind each number to the picture that illustrates
  % Riemann integral.

\begin{rem}
  A logical statement is either true or false.
  As one way to prove it,
   imagine that you are debating with someone
   that believes it is false.
  Your opponent will focus on the $\forall$ variables,
   i.e. the variables quantified by the
   $\forall$ quantifiers, 
   and try to instantiate some combination of their values 
   so that the formula fails.
  Your job is to focus on the $\exists$ variables
   so that for any possible combination value of the $\forall$ variables
   you can name a corresponding combination value
   for the $\exists$ variables so that the formula holds.
  From this viewpoint, a proof is simply a formula
   for generating values of the ``$\exists$'' variables
   according to the ``$\forall$'' variables.

  To show a logical statement is false,
   we could first negate the logical statement
   and then use the aforementioned method to show the negation holds.
\end{rem}

\begin{exm}
  The following function $f:[a,b]\rightarrow \mathbb{R}$
   is not Riemann integrable.
   \begin{equation}
     \label{eq:rationalVsIrr}
     f(x)=
     \begin{cases}
       1 & \text{if $x$ is rational};
       \\
       0 & \text{if $x$ is irrational}.
     \end{cases}
   \end{equation}

  To see this, we first
   negate the logical statement in (\ref{eq:RiemannIntegrable})
   to get
  \begin{align*}
    &\forall L\in\mathbb{R},
      \exists \epsilon>0, \text{ s.t. } \forall \delta>0 
    \\
    & \exists P_n(a,b) \text{ with } h(P_n)<\delta,\ 
      \text{ s.t. } |S_n(f)-L|\ge \epsilon.
  \end{align*}
  
  If $|L|< \frac{b-a}{2}$,
  we choose all $x_i^*$'s to be rational
  so that $f(x_i^*)\equiv 1$;
  then (\ref{eq:RiemannSum}) yields $S_n(f)=b-a$.
  For $\epsilon=\frac{b-a}{4}$,
   the formula $|S_n(f)-L|\ge \epsilon$ clearly holds.
  
  If $|L|\ge \frac{b-a}{2}$,
  we choose all $x_i^*$'s to be irrational
  so that $f(x_i^*)\equiv 0$;
  then (\ref{eq:RiemannSum}) yields $S_n(f)=0$.
  For $\epsilon=\frac{b-a}{4}$,
   the formula $|S_n(f)-L|\ge \epsilon$ clearly holds.
\end{exm}

\begin{rem}
  The difficulty in proving many statements of the $\epsilon$-$\delta$ type
  lies in how to classify values of the $\forall$ variables
  for instantiating values of the $\exists$ variables.
\end{rem}

\begin{defn}
  \label{def:definiteIntegral}
  If $f: \mathbb{R}\rightarrow \mathbb{R}$
   is integrable on $[a, b]$,
   then the limit of the Riemann sum of $f$
   is called the \emph{definite integral}
   of $f$ on $[a, b]$:
%, written,
   \begin{equation}
     \label{eq:definiteIntegral}
     \int_a^b f(x)\dif x
     = \lim_{h_n\rightarrow 0} S_n(f).
%\sum_{i=1}^n f(x_i^*) (x_{i}-x_{i-1}).
   \end{equation}
\end{defn}

\begin{thm}
  \label{thm:upperLowerRiemannIntegrals}
  A bounded function $f:[a,b]\rightarrow\mathbb{R}$
  is Riemann integrable if and only if
  the upper and lower Riemann integrals of $f$ are equal: 
  \begin{equation}
    \label{eq:upperLowerRiemannIntegrals}
    \overline{\int}_a^b f \dif x :=
    \inf_{h_n\rightarrow 0} M_n(f), \quad
%    \sum_{i=1}^n(x_i-x_{i-1}) \sup f(I_i),
    \underline{\int}_a^b f \dif x :=
    \sup_{h_n\rightarrow 0} m_n(f), 
%    \sum_{i=1}^n(x_i-x_{i-1}) \inf f(I_i),
  \end{equation}
  where $M_n(f)$ and $m_n(f)$ are respectively defined in
  (\ref{eq:upperRiemannSum}) and (\ref{eq:lowerRiemannSum}). 
\end{thm}

\begin{thm}
  A scalar function $f$ is integrable on $[a, b]$
   if $f\in {\cal C}[a,b]$.
\end{thm}

\begin{defn}
  \label{def:monotonicFunctions}
  A \emph{monotonic} function is a function between ordered sets
   that either preserves or reverses the given order.
  In particular,
   $f:\mathbb{R}\rightarrow\mathbb{R}$
   is \emph{monotonically increasing}
   if $\forall x,y$, $x \le y \Rightarrow f(x)\le f(y)$;
   $f:\mathbb{R}\rightarrow\mathbb{R}$
   is \emph{monotonically decreasing}
   if $\forall x,y$, $x \le y \Rightarrow f(x)\ge f(y)$.
\end{defn}

\begin{rem}
  The ordered sets in Definition \ref{def:monotonicFunctions}
  can be posets,
  but we will limit our attention to chains.
\end{rem}

\begin{thm}
  A scalar function is integrable on $[a, b]$
   if it is monotonic on $[a, b]$.
\end{thm}

\begin{exc}
  True or false:
   a bijective function is either order-preserving
    or order-reversing?
\end{exc}
\begin{solution}
  False; missing continuity.
  In other words,
   a continuous bijective function is either order-preserving
    or order-reversing.
\end{solution}

\begin{thm}[Integral mean value]
  \label{thm:integralMeanValue}
  Let $w:[a,b]\rightarrow \mathbb{R}^+$
  be integrable on $[a,b]$. For $f\in {\cal C}[a,b]$,
  $\exists \xi\in[a,b]$ s.t.
  \begin{equation}
    \label{eq:integralMeanValue}
    \int_a^b w(x)f(x)\dif x = f(\xi) \int_a^b w(x)\dif x.
  \end{equation}
\end{thm}
\begin{proof}
%    see page 19 on the text.
  Denote $m=\inf_{x\in[a,b]} f(x)$, $M=\sup_{x\in[a,b]} f(x)$,
   and $I=\int_a^b w(x)\dif x$.
  Then $m w(x)\le f(x)w(x) \le M w(x)$ and
  \begin{equation*}
    m I \le \int_a^b w(x) f(x) \dif x \le M I.
  \end{equation*}
  $w>0$ implies $I\ne 0$, hence
  \begin{equation*}
    m \le \frac{1}{I}\int_a^b w(x) f(x) \dif x \le M.
  \end{equation*}
  Applying Theorem \ref{thm:intermediateValue}
   completes the proof.
\end{proof}

\begin{thm}[First fundamental theorem of calculus]
  \label{thm:fundamentalThmCalculus1}
  Let $a<b$ be real numbers.
  For a continuous function $f: [a,b]\rightarrow \mathbb{R}$
  that is Riemann integrable, 
  define a function $F: [a,b]\rightarrow \mathbb{R}$ by
  \begin{equation}
    \label{eq:integralOfRiemannIntF}
    F(x) := \int_a^x f(y) \dif y.
  \end{equation}
  Then $F$ is differentiable and
  \begin{equation}
    \label{eq:fundamentalThmCalculus1}
    \forall x_0\in [a,b], \quad
    F'(x_0) = f(x_0).
  \end{equation}
\end{thm}

\begin{thm}[Second fundamental theorem of calculus]
  \label{thm:fundamentalThmCalculus2}
  Let $a<b$ be real numbers
  and let $f: [a,b]\rightarrow \mathbb{R}$
  be a Riemann integrable function.
  If $F:[a,b]\rightarrow\mathbb{R}$ is the antiderivative of $f$,
  i.e. $F'(x)=f(x)$,
  then
  \begin{equation}
    \label{eq:fundamentalThmCalculus2}
    \int_a^b f = F(b) - F(a).
  \end{equation}
\end{thm}


\section{Vector calculus}
\label{sec:sever-vari-diff}

\begin{rem}
  This section follows \emph{Analysis II} by Tao, Chapter 6.
\end{rem}

\begin{rem}
  \label{rem:singleVarDiff2MultiVarDiff}
  In single variable calculus,
  the derivative of a function $f$ defined in (\ref{eq:derivative})
  is equivalent to 
  \begin{displaymath}
    f'(x_0)=\lim_{x\rightarrow x_0, x\ne x_0} \frac{f(x)-f(x_0)}{x-x_0}.
  \end{displaymath}
  In the several variable case $f: E\rightarrow \mathbb{R}^m$
  where $E$ is now a subset of $\mathbb{R}^n$,
  the nominator $f(x)-f(x_0)$ is a vector in $\mathbb{R}^m$
  and the denominator $x-x_0$ a vector in $\mathbb{R}^n$.
  It does not make sense to divide the two vectors and
   taking norms of the two vectors would also be pointless.
  Therefore a straightforward generalization
  of the above definition of $f'(x_0)$ would not work.

  The solution is to rewrite the concept of derivative
  in one dimension so that generalization to multiple dimensions
  would be much easier.
  The following lemma is the first step.
\end{rem}

\begin{lem}
  \label{lem:derivativeEquiv}
  For $E\subset \mathbb{R}$, $f: E\rightarrow \mathbb{R}$,
  $x_0\in E$, and $L\in \mathbb{R}$,
  the following two statements are equivalent,
  \begin{enumerate}[(a)]\itemsep0em
  \item $f$ is differentiable at $x_0$ and $f'(x_0)=L$;
  \item $\lim_{x\rightarrow x_0, x\in E\setminus\{x_0\}} \frac{|f(x)-
      f(x_0) - L(x-x_0)|}{|x-x_0|}=0$.
  \end{enumerate}
\end{lem}

\begin{exc}
  Prove Lemma \ref{lem:derivativeEquiv}.
\end{exc}

\begin{rem}
  By Lemma \ref{lem:derivativeEquiv},
  the derivative can be interpreted as the number $L$
  for which $|f(x)- f(x_0)-L(x-x_0)|$ is small when $x$ tends to $x_0$,
  even if we divide it by the very small number $|x-x_0|$.
  In other words, the derivative is the quantity $L$
  that makes the approximation $f(x)-f(x_0) \approx L(x-x_0)$
  very accurate.
\end{rem}

\begin{defn}[Total derivative]
  \label{def:derivativeNdim}
  For $E\subset \mathbb{R}^n$, $f: E\rightarrow \mathbb{R}^m$,
  $x_0\in E$, $f$ is \emph{differentiable at $x_0$ with derivative}
  $L: \mathbb{R}^n \rightarrow \mathbb{R}^m$
  if
  \begin{equation}
    \label{eq:derivativeNdim}
    \lim_{x\rightarrow x_0, x\in E\setminus\{x_0\}} \frac{\|f(x)-
      f(x_0) - L(x-x_0)\|_2}{\|x-x_0\|_2}=0.
  \end{equation}
  We denote the derivative of $f$ with $f'(x_0)=L$
  and also call it the \emph{total derivative of $f$}.
\end{defn}

\begin{rem}
  Notice the similarity between (\ref{eq:derivativeNdim})
  and the statement (b) in Lemma \ref{lem:derivativeEquiv}.
  The differences are that the derivative is now a linear
  transformation instead of a real number
  and that the distance is now measured by the 2-norm of a vector
  instead of the absolute value of a real number.  
\end{rem}

\begin{exm}
  For $f: \mathbb{R}^2\rightarrow \mathbb{R}^2$
  and $L: \mathbb{R}^2\rightarrow \mathbb{R}^2$,
  \begin{equation}
    \label{eq:fSqaureComp}
    f(x,y):=(x^2, y^2), \ \ L(x,y):=(2x, 4y), 
  \end{equation}
  we claim that $f$ is differentiable at $(1,2)$
  with \mbox{$f'(x_0)=L$} by computing
  \begin{align*}
    &\lim_{\stackrel{(x,y)\rightarrow (1,2)}{(x,y)\ne(1,2)}}
    \frac{\|f(x,y)- f(1,2) - L((x,y)-(1,2))\|_2}{\|(x,y)-(1,2)\|_2}
      \\
    =&\lim_{\stackrel{(a,b)\rightarrow (0,0)}{(a,b)\ne(0,0)}}
    \frac{\|f(1+a,2+b)- f(1,2) - L(a,b)\|_2}{\|(a,b)\|_2}
      \\
    =&\lim_{\stackrel{(a,b)\rightarrow (0,0)}{(a,b)\ne(0,0)}}
    \frac{\|\left((1+a)^2,(2+b)^2\right)- (1,4) - (2a,4b)\|_2}{\|(a,b)\|_2}
      \\
    =&\lim_{\stackrel{(a,b)\rightarrow (0,0)}{(a,b)\ne(0,0)}}
    \frac{\|(a^2, b^2)\|_2}{\|(a,b)\|_2}
    \le \lim_{\stackrel{(a,b)\rightarrow (0,0)}{(a,b)\ne(0,0)}}
          \left(\frac{\|(a^2, 0)\|_2}{\|(a,b)\|_2}
          +  \frac{\|(0, b^2)\|_2}{\|(a,b)\|_2}\right)
      \\
    = &\lim_{\stackrel{(a,b)\rightarrow (0,0)}{(a,b)\ne(0,0)}}
        \sqrt{a^2 + b^2}
    = 0.
  \end{align*}
\end{exm}

\begin{lem}
  \label{lem:uniquenessOfDerivatives}
  Let $E$ be a subset of $\mathbb{R}^n$,
  $f: E\rightarrow \mathbb{R}^m$ a function,
  and $x_0\in E$ an interior point of $E$.
  Suppose $f$ is differentiable at $x_0$ with derivative $L_1$
  and also differentiable at $x_0$ with derivative $L_2$.
  Then $L_1=L_2$.
\end{lem}

\begin{exc}
  Prove Lemma \ref{lem:uniquenessOfDerivatives}.
\end{exc}

\begin{rem}
  Thanks to Lemma \ref{lem:uniquenessOfDerivatives},
  we have \emph{the} derivative of $f$ at an interior point $x_0$.
  As another consequence of Lemma \ref{lem:uniquenessOfDerivatives},
  if $f(x)=g(x)$ for all $x\in E$
  and $f,g$ are both differentiable at an interior point $x_0$ of $E$,
  then $f'(x_0)=g'(x_0)$.
\end{rem}

\begin{defn}[Directional derivative]
  \label{def:directionalDerivative}
  Let $E$ be a subset of $\mathbb{R}^n$,
  $f: E\rightarrow \mathbb{R}^m$ a function,
  $x_0\in E$ an interior point of $E$,
  and $\mathbf{v}\in \mathbb{R}^n$ a vector.
  If the limit
  \begin{displaymath}
    \lim_{t\rightarrow 0; t>0, x_0+t\mathbf{v}\in E} \frac{f(x_0+t\mathbf{v}) - f(x_0)}{t}
  \end{displaymath}
  exists, we say that $f$ is \emph{differentiable in the direction $\mathbf{v}$
    at $x_0$},
  and we denote this limit as
  \begin{equation}
    \label{eq:directionalDerivative}
    D_\mathbf{v} f(x_0) := \lim_{t\rightarrow 0; t>0, x_0+t\mathbf{v}\in E}
    \frac{f(x_0+t\mathbf{v}) - f(x_0)}{t}.
  \end{equation}
\end{defn}

\begin{exm}
  For $\mathbf{v}=(3,4)$ and $f: \mathbb{R}^2\rightarrow\mathbb{R}^2$ defined in
  (\ref{eq:fSqaureComp}),
  we have $D_{\mathbf{v}}f(1,2)=(6,16)$.
\end{exm}

\begin{exm}
  For $f: \mathbb{R}\rightarrow\mathbb{R}$,
  $D_{+1} f(x)$ is the right derivative of $f$ at $x$ (if it exists),
  and similarly $D_{-1}f(x)$ is the left derivative of $f$ at $x$ (if it exists).
\end{exm}

\begin{rem}
  The following lemma connects total derivatives and directional derivatives.
\end{rem}

\begin{lem}
  \label{lem:totalDiffImpliesDirectionalDiff}
  Let $E$ be a subset of $\mathbb{R}^n$,
  $f: E\rightarrow \mathbb{R}^m$ a function,
  $x_0\in E$ an interior point of $E$,
  and $\mathbf{v}\in \mathbb{R}^n$ a vector.
  If $f$ is differentiable at $x_0$,
  then $f$ is also differentiable in the direction $\mathbf{v}$ at $x_0$, and
  \begin{equation}
    \label{eq:totalDiffImpliesDirectionalDiff}
    D_\mathbf{v} f(x_0) = f'(x_0) \mathbf{v}.
  \end{equation}
\end{lem}

\begin{rem}
  The converse of Lemma \ref{lem:totalDiffImpliesDirectionalDiff} is
  false.
\end{rem}

\begin{defn}[Partial derivative]
  \label{def:partialDerivative}
  Let $E$ be a subset of $\mathbb{R}^n$,
  $f: E\rightarrow \mathbb{R}^m$ a function,
  $x_0\in E$ an interior point of $E$,
  and $1\le j \le n$.
  The \emph{partial derivative of $f$ with respect to the $x_j$
    variable at} $x_0$ is defined by
  \begin{equation}
    \label{eq:partialDerivative}
    \begin{array}{ll}
    \frac{\partial f}{\partial x_j}(x_0) &:=
    \lim_{t\rightarrow 0; t>0, x_0+te_j\in E} \frac{f(x_0+te_j) -
      f(x_0)}{t} \\
      &= \frac{\dif}{\dif t} \left. f(x_0+te_j)\right|_{t=0}
    \end{array}
  \end{equation}
  provided that the limit exists.
  Here $e_j$ is the $j$th standard basis vector of $\mathbb{R}^n$.
\end{defn}

\begin{rem}
  By Lemma \ref{lem:totalDiffImpliesDirectionalDiff},
  one can write directional derivatives
  in terms of partial derivatives
  if the function is actually differentiable at that point.
  More precisely, for the vector
  $\mathbf{v}=(v_1, v_2, \ldots, v_n)^T=\sum_j v_j e_j$,
  we have
  \begin{displaymath}
    D_{\mathbf{v}} f(x_0) = f'(x_0) \sum_j v_j e_j = \sum_j v_j
    f'(x_0) e_j = \sum_j v_j \frac{\partial f}{\partial x_j}(x_0), 
  \end{displaymath}
  where the first step follows from Lemma
  \ref{lem:totalDiffImpliesDirectionalDiff},
  the second step from $f'(x_0)$ being a linear transformation,
  and the last step from (\ref{eq:partialDerivative}),
  (\ref{eq:totalDiffImpliesDirectionalDiff}), and
  (\ref{eq:directionalDerivative}).
  In particular, for $m=1$ we have the familar formula
  \begin{displaymath}
    D_{\mathbf{v}} f(x_0) = \mathbf{v}\cdot \nabla f(x_0),
  \end{displaymath}
  where the \emph{gradient} of the function
  $f: \mathbb{R}^n\rightarrow \mathbb{R}$ is defined as
  \begin{displaymath}
    \nabla f := \left(
      \frac{\partial f}{\partial x_1}, 
      \frac{\partial f}{\partial x_2}, 
      \ldots,
      \frac{\partial f}{\partial x_n}
      \right).
  \end{displaymath}
\end{rem}

\begin{exc}
  Show that the existence of partial derivatives at $x_0$
  does not imply that the function is differentiable at $x_0$
  by considering the differentiability of
  the following function $f: \mathbb{R}^2\rightarrow \mathbb{R}$
  at (0,0). 
  \begin{displaymath}
    f(x,y) =
    \begin{cases}
      \frac{x^3}{x^2+y^2} & \text{if } (x,y)\ne (0,0);
      \\
      0 & \text{if } (x,y)= (0,0).
    \end{cases}
  \end{displaymath}
\end{exc}
\begin{solution}
The function is not differentiable at (0,0),
  although it is differentiable in every direction $\mathbf{v}\in
  \mathbb{R}^2$ at $(0,0)$.
  Indeed, for $\mathbf{v}=(1,0)$, we have
  \begin{displaymath}
    D_{\mathbf{v}} f (0,0) = \lim \frac{f(t,0)-f(0,0)}{t}
    = 1
  \end{displaymath}
  while for $\mathbf{u}=(0,1)$, we have
  \begin{displaymath}
    D_{\mathbf{u}} f (0,1) = \lim \frac{f(0,t)-f(0,0)}{t}
    = 0.
  \end{displaymath}
  Hence the limit in (\ref{eq:derivativeNdim}) does not exist.
\end{solution}


\begin{thm}
  \label{thm:continuousParitialDerivativesImplyDifferentiability}
  Let $E$ be a subset of $\mathbb{R}^n$,
  $f: E\rightarrow \mathbb{R}^m$ a function,
  $F$ a subset of $E$,
  and $x_0\in E$ an interior point of $F$.
  If all the partial derivatives $\frac{\partial f}{\partial x_j}$
  exist on $F$ and are continuous at $x_0$,
  then $f$ is differentiable at $x_0$,
  and the linear transformation
  $f'(x_0): \mathbb{R}^n \rightarrow \mathbb{R}^m$ is defined by
  \begin{equation}
    \label{eq:continuousParitialDerivativesImplyDifferentiability}
    f'(x_0) (\mathbf{v})
    = \sum_{j=1}^n v_j \frac{\partial f}{\partial x_j} (x_0).
  \end{equation}
\end{thm}


\begin{defn}
  The \emph{derivative matrix} or \emph{differential matrix}
  or \emph{Jacobian matrix} of a differentiable function
  \mbox{$f: \mathbb{R}^n\rightarrow\mathbb{R}^m$}
  is a $m\times n$ matrix,
  \begin{equation}
    \label{eq:JacobianMatrix}
    Df :=
    \begin{pmatrix}
      \frac{\partial f_1}{\partial x_1}
      &
      \frac{\partial f_1}{\partial x_2}
      & \cdots &
      \frac{\partial f_1}{\partial x_n}
      \\
      \frac{\partial f_2}{\partial x_1}
      &
      \frac{\partial f_2}{\partial x_2}
      & \cdots &
      \frac{\partial f_2}{\partial x_n}
      \\
      \vdots & \vdots & \ddots & \vdots &
      \\
      \frac{\partial f_m}{\partial x_1}
      &
      \frac{\partial f_m}{\partial x_2}
      & \cdots &
      \frac{\partial f_m}{\partial x_n}
    \end{pmatrix}.
  \end{equation}
\end{defn}

\begin{rem}
  For a row vector $\mathbf{v}\in \mathbb{R}^n$,
  we have
  \begin{displaymath}
%    \label{eq:relationsOfMatrixAndDerivatives}
    (D_{\mathbf{v}} f(x_0))^T = (f'(x_0)\mathbf{v})^T = D f(x_0) \mathbf{v}^T.
  \end{displaymath}
  We can also write $D f$ as
  \begin{align*}
    Df(x_0) &= \left(
      \frac{\partial f_1}{\partial x_1},
      \frac{\partial f_1}{\partial x_2},
      \ldots, 
      \frac{\partial f_1}{\partial x_n}\right)
    \\
    &=
      \begin{pmatrix}
        \nabla f_1(x_0)
        \\
        \nabla f_2(x_0)
        \\
        \vdots
        \\
        \nabla f_m(x_0)
      \end{pmatrix}.
  \end{align*}
\end{rem}

\begin{thm}[Implicit function theorem]
  \label{thm:implicitFunction}
  Suppose a ${\cal C}^1$ function $\mathbf{g}: \mathbb{R}^m\times
  \mathbb{R}^n\to \mathbb{R}^n$
  satisfies
  \begin{enumerate}[(i)]\itemsep0em
  \item
    $\mathbf{g}(\mathbf{x}_0, \mathbf{y}_0)=\mathbf{0}$
    where $(\mathbf{x}_0, \mathbf{y}_0)\in \mathbb{R}^m\times
    \mathbb{R}^n$; 
  % \item
  %   $\mathbf{g}$ is $\mathcal{C}^r$ in a neighborhood of $(\mathbf{x}_0, \mathbf{y}_0)$;
  \item
    the Jacobian matrix $\frac{\partial \mathbf{g}}{\partial \mathbf{y}}(\mathbf{x}_0, \mathbf{y}_0)$ is invertible.
  \end{enumerate}
  Then there is a neighborhood $U$ of $\mathbf{x}_0$ and
  a unique ${\cal C}^1$ function $\mathbf{f}: U\to \mathbb{R}^n$
  such that
  $\mathbf{f}(\mathbf{x}_0) = \mathbf{y}_0$
  and $\mathbf{g}(\mathbf{x}, \mathbf{f}(\mathbf{x}))=\mathbf{0}$
  for all $\mathbf{x}\in U$.
  Furthermore, if $\mathbf{g}$ is analytic or ${\cal C}^p$,
  then $\mathbf{f}$ is analytic or ${\cal C}^p$.
\end{thm}


%%% Local Variables:
%%% mode: latex
%%% TeX-master: "../notesNumAnal"
%%% End:

% LocalWords:  Lipschitz pointwise boundedness integrable bijective
